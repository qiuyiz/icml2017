\section{Earnshaw's Theorem and Harmonic Potentials} 
%
When running gradient descent on non-convex loss functions, we often
can and do get stuck at a local minima. In this section, we use
second-order information to deduce that the local
minima for these certain classes of potentials occur only when certain convergence statements hold. In these section, these potentials are often {\it non-smooth, unbounded and un-realizable}. However, in the later sections, we apply insights developed here to derive similar convergence results for approximations of these potentials.
%
\subsection{Earnshaw's Theorem}
%
Earnshaw's theorem in electrodynamics shows that there is no stable
local minima for electron-proton dynamics. This hinges on the property
that the electric potential
$\Phi(\theta,w) = \|\theta-w\|^{2-d}, d \neq 2$ is harmonic, with
$d = 3$ in natural setting. If $d = 2$, we instead have
$\Phi(\theta, w) = - \ln(\|\theta - w\|)$. First, we notice that this
is a symmetric loss, and our usual loss in \eqref{errLoss} has
constant terms that can be dropped to further simplify.
%
\begin{equation}\label{errSimp}
\overline{L}(a,\theta) =  2\sum_{i=1}^k\sum_{i < j} a_ia_j\Phi(\theta_i,\theta_j) + 2\sum_{i=1}^k\sum_{j=1}^ka_ib_j \Phi(\theta_i,w_j)
\end{equation} 
%
\begin{definition}
$\Phi(\theta,w)$ is a {\bf harmonic} potential on $\Omega$ if $\Delta_\theta \Phi(\theta,w) = 0$ for all $\theta \in \Omega$, except possibly at $\theta = w$.
\end{definition}

\begin{definition}
  Let $\Omega \subseteq \R^d$ and consider a function
  $f:\Omega \to \R$. A critical point $x^* \in \Omega$ is a {\bf local
    minimum} if there exists $\epsilon > 0$ such that
  $f(x^*+v) \geq f(x^*)$ for all $\|v\|\leq \epsilon$. It is a {\bf
    strict local minimum} if the inequality is strict for all
  $\norm{v} \le \epsilon.$
\end{definition} 
%
\begin{fact}
  % Let $\Omega \subseteq \R^d$ and consider a function
  % $f:\Omega \to \R$.  
  Let $x^*$ be a critical point of a function $f : \Omega \to \R$ such
  that $f$ is twice differentiable at $x^*.$ Then, if $x^*$ is a local
  minimum then $\lambda_{min}(\nabla^2 f(x^*)) \geq 0.$ Moreover, if
  $\lambda_{min}(\nabla^2 f(x^*)) > 0,$ then $x^*$ is a strict local minimum.
\end{fact}
%

Note that if $\lambda_{min}(\nabla^2 f(x^*)) < 0$ then moving along the direction of the corresponding eigenvector decreases $f$ locally. if $\Phi$ is harmonic then it can be shown the trace of its Hessian is $0$ so 
if there is any non zero eigenvalue then at least one eigenvalue is negative. This idea results in the following known theorem (see full proof in supplementary material) that is applicable to the electric potential  
function $1/r$ in $3$-dimensions since is harmonic. It implies that a configuration of $n$ electrons and $n$ protons cannot be in a strict local minimum even if one of the mobile charges is isolated
(however note that this potential function goes to $\infty$ at $r=0$ and may not be realizable).

\begin{restatable}{theorem}{earnshaw}
\emph{(Earnshaw's Theorem. See~\cite{arnold1985mathematical})}
\label{Earnshaw} 
Let $\mathcal{M} = \R^d$ and let $\Phi$ be harmonic and $L$
be as in $\eqref{errSimp}$. Then, $L$ admits no
differentiable strict local minima.
\end{restatable}
%

Note that a potential being harmonic does not mean that its Hessian has a strictly negative eigenvalue at everypoint as all eigenvalues may be $0$. To
avoid the latter possibility we will generalize the idea of harmonic potentials. 

\if{1}
\begin{corollary}
Let $\mathcal{M} = \R^d$, $\Phi(\theta,w) = \|\theta-w\|^{2-d}$, and $L$ be as in \eqref{errSimp}. If $\theta_i$ are all distinct and $(\boldsymbol{a,\theta})$ is a strict local minima, then $\theta$ has reached the global minima, i.e. $\theta_i = w_{\pi(i)}$ for some permutation $\pi$. 
\end{corollary}
\fi


%Intuitively, the trace of the Hessian being 0 implies every
%differentiable critical point has a direction of negative curvative
%(unless the Hessian is the zero matrix altogether, in which case some
%complex analysis still finds a direction of negative change
%\cite{arnold1985mathematical}). 

\iffalse
Such a convergence result is desired but we run into two main problems. First, the singularity of the natural harmonic potentials at $0$ disqualifies them from being a realizable potential. Furthermore, harmonic potentials are not robust to approximation and statistical error as the convergence guarantees hinge on $\Delta_\theta\Phi$ being exactly 0. 

Second, the lack of strict local minima {\it does not} imply
convergence to the global minima under gradient descent. Notice that
harmonic potentials can admit local minima, as the Hessian matrix can
be the zero matrix, and so gradient descent could converge to these
local minima. However, if our loss function admits no local minima
(other than the global minima), then we can guarantee that gradient
descent with small stepsize converges to the global minima.
\fi

\subsection{$\lambda$-Harmonic Potentials}
\iffalse
The first alternative to harmonic potentials is the natural
consideration of strictly subharmonic potentials, which have a
positive Laplacian value almost everywhere. Subharmonic potentials are
also difficult to realize; however their convergence properties are
more robust than harmonic potentials and are discussed in supplementary material. 
\fi

%
In order to relate our loss function with its Laplacian, we consider potentials that are non-negative eigenfunctions of the Laplacian operator. Since the zero eigenvalue case simply gives rise to harmonic potentials, we restrict our attention to positive eigenfunctions.
%
\begin{definition}
A potential $\Phi$ is {\bf$\lambda$-harmonic} on $\Omega$ if there exists $\lambda > 0$ such that for every $\theta \in \Omega$, $\Delta_\theta \Phi(\theta, w) = \lambda \Phi(\theta,w) $, except possibly at $\theta = w$.
\end{definition}

We note that there are realizable versions of these potentials; for
example $\Phi(a,b) = e^{-\|a-b\|_1}$ in $\R^1.$
% as in Table~\ref{table1}.
In the next section, we construct  realizable potentials that are 
$\lambda$-harmonic almost everywhere except when $\theta$ and $w$ are very close. 
%
\begin{restatable}{theorem}{eigStrict}
\label{EigStrict}
Let $\Phi$ be $\lambda$-harmonic and $L$ be as in \eqref{errLoss}. Then,
$L$ admits no local minima $\boldsymbol{(a,\theta)}$, except when
$L(\boldsymbol{a,\theta}) = L(0,\boldsymbol{\theta})$ or $\theta_i = w_j$ for some $i,j$. 
\end{restatable}
\begin{proof}
  Let $(\boldsymbol{a,\theta})$ be a critical point of $L.$ On the
  contrary, we assume that $\theta_i \neq w_j$ for all $i,j.$ WLOG, we
  can partition $[k]$ into $S_1,...,S_r$ such that for all $u \in S_i,
  v \in S_j$, we have $\theta_{u} = \theta_v$ iff $i=j$. 

Let $S_1 = \{ \theta_1, \ldots, \theta_{l}\}.$ 
% WLOG, let $\theta_1 =
% \theta_2 =... = \theta_l$ be non-distinct (with $l$ maximal) and
% $\theta_1 \neq w_j$ for any $j$. 
We consider changing all
$\theta_1, \ldots, \theta_{l}$ by the same $v$ and define 
%
\[H({\bf a}, v) = L({\bf a},\theta_1+v,...,\theta_l+v, \theta_{l+1}
\ldots, \theta_k).\]

The optimality conditions on ${\bf a}$ are 
\begin{align*}
0 = \pd{L}{a_i} = 2a_i  + 2\sum_{j\neq i} a_j \Phi(\theta_i,\theta_j)
  + 2\sum_{j=1}^k b_j \Phi(\theta_i,w_j).
\end{align*}
%
Next, by $\lambda$-harmonic definition, we may differentiate as $\theta_i \neq w_j$ and compute the Laplacian as 
\begin{align*}
\Delta_v H & = \lambda\sum_{i=1}^l a_i \left(2\sum_{j=1}^k b_j
  \Phi(\theta_i, w_j) + 2\sum_{j=l+1}^k a_j
  \Phi(\theta_i, \theta_j)\right) \\
& = \lambda\sum_{i=1}^l a_i \left(-2a_i\Phi(\theta_i, \theta_i) - 2
  \sum_{j = 1, j\neq i}^l  a_j \Phi(\theta_i,\theta_j)\right) \\
%
% & = -2\lambda\left(\sum_{i=1}^l a_i^2 \Phi(\theta_i, \theta_i)
%   +\sum_{i \neq j}^l a_i a_j \Phi(\theta_i, \theta_j)\right) \\
%
& = -2\lambda\left(\sum_{i=1}^l a_i^2
  +\sum_{i \neq j}^l a_i a_j \right) = -2 \lambda\left(\sum_{i=1}^l a_i\right)^2
\end{align*} 
%
If $\sum_{i=1}^l a_i \neq 0$, then we conclude that the Laplacian is
strictly negative, so we are not at a local minimum. 

Similarly, we can conclude that for each $S_i,$ 
% Therefore, if $\theta_i \neq w_j$ for all $i,j$, then we may partition $[k]$ into $S_1,...,S_r$ such that each $S_i$, $\theta_{u} = \theta_v$
% for $u, v \in S_i$ and
 $\sum_{u \in S_i} a_u = 0$. In this case, since $\sum_{i=1}^k a_i \sigma(\theta_i,x) = 0$, $L(\boldsymbol{a,\theta}) = L(0,\boldsymbol{\theta})$. 
\end{proof} 
%
\iffalse
\Snote{Rina, Richard: I think we should leave out this
  observation. Let's discuss tomorrow morning.}
\begin{observation}
We can initialize $\boldsymbol{a,\theta}$ such that $L(\boldsymbol{a,\theta}) < L(0,\boldsymbol{\theta})$ and continuous gradient descent will enforce the inequality as the $L$ is non-increasing. Initialization details are expounded in Observation~\ref{initialize}.
\end{observation}
\fi
%

%%% Local Variables:
%%% mode: latex
%%% TeX-master: "icmlpaper2017.tex"
%%% End: