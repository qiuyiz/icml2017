\section{Common Activations}
First, we consider the sign activation function. Under restrictions on the size of the input dimension or the number of hidden units, we can prove convergence results under the sign activation function, as it gives rise to a harmonic potential.

\Anote{ythis theorem is about $\theta\in R$, which is not of any practical interest. you should remove it.}
\begin{assumption}
\label{outputFixed}
All output weights $b_i = 1$ and therefore the output weights  $a_i = - b_i = -1$ are fixed throughout the learning algorithm. 
\end{assumption}

\begin{restatable}{lemma}{signcon}\label{signCon}
Let $\mathcal{M} = S^1$ and let Assumption \ref{outputFixed} hold. Let $L$ be as in \eqref{errSimp} and $\sigma$ is the sign activation function. Then $L$ admits no strict local minima, except at the global minima.
\end{restatable}


We cannot simply analyze the convergence of GD
on all $\theta_i$ simultaneously since as before, the pairwise
interaction terms between the $\theta_i$ present complications. To
derive a tighter control on $(\boldsymbol{a,\theta})$, we run a greedy
node-wise second-order gradient descent (SGD) algorithm to learn the hidden weights, i.e. we run a full SGD algorithm with respect to $(a_i,\theta_i)$ sequentially. The
main idea is that after running SGD with respect to $\theta_1$,
$\theta_1$ should be close to some $w_j$ for some $j$. Then, we can
carefully induct and show that $\theta_2$ must be some other $w_k$ for
$k\neq j$ and so on. 

In this section, we only consider the convergence guarantee of gradient descent on the first node, $\theta_1$, to some $w_j$, while the other nodes are inactive (i.e. $a_2,...,a_k = 0$). In essence, we are working with the following simplified loss function.
%
\begin{equation}\label{errLossUnit}
L(a_1,\theta_1) =  a_1^2 \Phi(\theta_1,\theta_1)  + 2\sum_{j=1}^k a_1b_j \Phi(\theta_1,w_j)
\end{equation}


\begin{restatable}{lemma}{signconv}
\label{SignConv}
Let $\mathcal{M} = S^1$ and $L$ be as in \eqref{errLossUnit} and $\sigma$ is the sign activation function. Then, almost surely over random choices of $b_1,...,b_k$, all local minima of $L$ are at $\pm w_j$. 
\end{restatable}
%
For the polynomial activation and potential functions, we also can show convergence under orthogonality assumptions on $w_j$. 

\begin{restatable}{theorem}{polystrict}
\label{PolyStrict}
Let $\mathcal{M} = S^{d-1}$. Let $w_1,...,w_k$ be orthonormal vectors in $\R^d$ and $\Phi$ is of the form $\Phi(\theta,w) = (\theta^Tw)^l$ for some fixed integer $l \geq 3$. Let $L$ be as in \eqref{errLossUnit}. Then, all critical points of $L$ are not local minima, except when $\theta_1 = w_j$ for some $j$.   
\end{restatable}


