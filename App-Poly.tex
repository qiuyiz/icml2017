
\section{Convergence of Polynomial Potentials}

\polystrict*

\begin{proof}
WLOG, we can consider $w_1,...,w_d$ to be the basis vectors $e_1,...,e_d$. Note that this is a manifold optimization problem, so our optimality conditions are given by introducing a Lagrange multiplier $\lambda$, as in \cite{GeHJY15}.
\[\pd{L}{a} = 2\sum_{i=1}^d ab_i (\theta_i)^l + 2a = 0\]
\[ (\nabla_\theta L)_i = 2ab_il(\theta_i)^{l-1}  -2\lambda \theta_i = 0 \]
where $\lambda$ is chosen that minimizes 
\[\lambda = \arg \min_\lambda \sum_i (ab_i l (\theta_i)^{l-1} - \lambda\theta_i)^2 = \sum ab_i l (\theta_i)^l \]
Therefore, either $\theta_i = 0$ or $b_i (\theta_i)^{l-2} = \lambda/(al)$. From \cite{GeHJY15}, we consider the constrained Hessian, which is a diagonal matrix with diagonal entry: 
\[(\nabla^2 L)_{ii} = 2a b_i l(l-1)(\theta_i)^{l-2} - 2 \lambda\]
Assume that there exists $\theta_i, \theta_j \neq 0$, then we claim that $\theta$ is not a local minima. First, our optimality conditions imply $b_i(\theta_i)^{l-2} = b_j (\theta_j)^{l-2} = \lambda/(al)$. So,
\[(\nabla^2 L)_{ii} = (\nabla^2L)_{jj} = 2a b_i l(l-1)(\theta_i)^{l-2} - 2 \lambda\]
\[ = 2(l-2)\lambda = -2(l-2)la^2\]
Now, there must exist a vector $v \in S^{d-1}$ such that $v_k = 0$ for $k \neq i,j$ and $v^T\theta = 0$, so $v$ is in the tangent space at $\theta$. Finally, $v^T(\nabla^2 L) v  = -2(l-2)l a^2 < 0$, implying $\theta$ is not a local minima when $a \neq 0$. Note that $a = 0$ occurs with probability 0 since our objective function is non-increasing throughout the gradient descent algorithm and is almost surely initialized to be negative with $a$ optimized upon initialization, as by observed before.
\end{proof}

When the output weights are variable, notice that our convergence results often rely on the optimality of the output weights. For this reason, we will optimize the output weights at every gradient descent step, by carefully choosing the stepsize. As our loss function \eqref{errLoss} is quadratic in $a$, we know that gradient descent will find the optimal $a$ efficiently. The pseudocode of our SGD algorithm is given in Algorithm \ref{NodeGDOpt}. 


%
\begin{theorem}\cite{nesterov2013introductory}\label{quadConverge}
  Let $x_0 \in \Omega = \{ x \in \R^d | \|x \| \leq \poly(d)\}$ and let
  $L(x) = x^TAx + b^Tx$ be a quadratic loss, where $A$ is a positive
  semi-definite matrix with maximum eigenvalue $\beta$. Then, running
  Algorithm \ref{GD} on $L$ with stepsize $\alpha = 1/\beta$ converges
  to $x_T$ such that
  \[L(x_T) - \min_{x \in \Omega} L(x) \leq \epsilon\]
  in $T = \poly(d, \beta, 1/\epsilon)$ iterations.
\end{theorem} 

%
\begin{algorithm}[tb]
 \caption{Node-wise Gradient Descent Algorithm with Output Weights Optimization}
   \label{NodeGDOpt}
\begin{algorithmic}
  \STATE {\bfseries Input:}
  $(\boldsymbol{a,\theta}) = (a_1,...,a_k,\theta_1,...,\theta_k), a_i
  \in\R, \theta_i\in\mathcal{M}$;
  $T\in \N$; $\widehat{L}$; $\alpha\in \R$; $\delta \in \R$;
  $\gamma \in R$; \vspace{0.1in} \STATE{\bf Initialize} $a = 0$
  \FOR{$i=1$ {\bfseries to} $k$} 
  \REPEAT \STATE Sample $\theta_i$
  uniformly from $\mathcal{M}$
  \UNTIL{$\left (\frac{\partial
        \widehat{L}(\boldsymbol{a,\theta})}{\partial a_i} \right)^2
    \geq \gamma$}
  \FOR{$j=1$ {\bfseries to} $T$} \STATE
  $a_i = a_i^* = GradientDescent \left(\widehat{L}_{a_i}, a_i, 1,
    \frac{\partial^2 \hat{L}}{\partial a_i^2} \right)$
  \STATE
  $\theta_i = SGD \left(\widehat{L}_{\theta_i}, \theta_i,1, \alpha,\delta \right)$
   \ENDFOR
   \STATE    $a =  GradientDescent \left(\widehat{L}_{a_1,..,a_i},
     (a_1,..,a_i), T , \alpha \right)$\;
   \ENDFOR
   \STATE {\bf return} $a = (a_1,...,a_k), \theta = (\theta_1,..., \theta_k)$
   \end{algorithmic}
\end{algorithm}


First, we need some control on the size of the squares of variable charges, $a_i^2$. Node-wise gradient descent allows us to maintain that control. We will work with \eqref{errEmp} and let $\widehat{\Phi}$ be the empirical potential function.

\begin{theorem}\label{nonDecrease} \Rnote{Can be simplified}
  Let $\widehat{\Phi}(\theta, w)$ be $\poly(d)$-Lipschitz in $\theta$,
  $|\widehat{\Phi}| \leq \poly(d)$ on $\mathcal{M}$, and $\sum_{j \neq i} |a_j| + \sum_j |b_j| \leq \poly(d)$. Also, assume that if $\theta_i$ is drawn uniformly on $\mathcal{M}$,
%
\[\expt_{\theta_i}\left[\left(  \sum_{j\neq i} a_j \widehat{\Phi}(\theta_i,\theta_j) + \sum_{j=1}^k b_j \widehat{\Phi}(\theta_i,w_j)\right)^2\right] \geq \epsilon \]
%
Then, with high probability, running Algorithm \ref{NodeGDOpt} on $\widehat{L}$ with stepsize at most $\alpha = 1/\poly(d)$ and $\gamma = \epsilon$ will enforce that $a_i^2 = \Omega(\epsilon)$ when running SGD on $\theta_i$. The sample complexity is $\poly(d,1/\epsilon)$
\end{theorem}

\begin{proof}
We want to show that $a_i^2 = \Omega(\epsilon)$. First, we analyze the random drawing of $\theta_i$.  Let 
%
\[  \widehat{C}(\theta_i) = \sum_{j\neq i} a_j \widehat{\Phi}(\theta_i,\theta_j) + \sum_{j=1}^k b_j \widehat{\Phi}(\theta_i,w_j)\]
%

By assumption, we know that $\expt[\widehat{C}(\theta_i)^2] \geq
\epsilon$. Since $a_i,b_i, \widehat{\Phi}$ is $\poly(d)$-bounded, then $\widehat{C}(\theta_i)$ is $\poly(d)$-bounded. From Hoeffding bounds, we see that after $\poly(d,1/\epsilon)$ samples, we know that $\hat{\expt}\widehat{C}(\theta_i)^2 \geq \epsilon/2$, where $\hat{\expt}$ denotes the empirical average over $\theta_i$. Therefore, we must have found some $\theta_i$ such that $\widehat{C}(\theta_i)^2 \geq \epsilon/2$.

Next, notice that $\hat{L}$ is a quadratic in $a_i$, which is a
scalar. Therefore, the gradient step on $a_i$ is chosen with the exact
stepsize that will optimize $a_i = a_i^*(\theta_i)$.  By optimality of
$a$,
%
\begin{equation}
0 = 2a_i \widehat{\Phi}(\theta_i,\theta_i) + 2\sum_{j\neq i} a_j \widehat{\Phi}(\theta_i,\theta_j) + 2\sum_{j=1}^k b_j \widehat{\Phi}(\theta_i,w_j)
\end{equation} 
%
Therefore, 
%
\begin{align*}
 a_i^* & = - \frac{1}{\widehat{\Phi}(\theta_i,\theta_i)} (\sum_{j\neq i} a_j \widehat{\Phi}(\theta_i,\theta_j) + \sum_{j=1}^k b_j \widehat{\Phi}(\theta_i,w_j)) \\
& = -\frac{1}{\widehat{\Phi}(\theta_i,\theta_i)}\widehat{C}(\theta_i) 
\end{align*}


As $a_i$ is initially $0$ when sampling $\theta_i$, we conclude that we can find $\theta_i$ such that
%
\[ \left( \pd{\hat{L}}{a_i} \right)^2 = 4 \widehat{C}(\theta_i)^2 \geq \epsilon.\]
Therefore, after $\poly(d,1/\epsilon)$ samples, our sampling algorithm will, with high probability, return $\theta_i$  such that $a_i^2 = a_i^*(\theta_i)^2 \geq \epsilon /4$. 

Next, we examine the SGD steps. Note
%
\begin{align*}
\nabla_{\theta_i} \widehat{C}(\theta_i) & =  \sum_{j\neq i} a_j \nabla_{\theta_i}\widehat{\Phi}(\theta_i,\theta_j) + \sum_{j=1}^k b_j \nabla_{\theta_i}\widehat{\Phi}(\theta_i,w_j) \\
& = \frac{1}{2a_i^*} \nabla_{\theta_i}\widehat{L}
\end{align*}

Now, we calculate the following gradient: 
%
\begin{equation}
\nabla_{\theta_i} (a_i^*)^2 = 2a_i^*\frac{-1}{\widehat{\Phi}(\theta_i,\theta_i)} \nabla_{\theta_i} \widehat{C}(\theta_i) = \frac{1}{\widehat{\Phi}(\theta_i,\theta_i)} (-\nabla_{\theta_i} \widehat{L})
\end{equation}

Therefore, let $\theta_{i}^{(j)}$ be the value of $\theta_i$ in the $j$-th iteration of SGD with stepsize $\alpha$. Since $\theta_i$ moves in the direction of the gradient of $(a_i^*)^2$ in expectation and $(a_i^*)^2$ is $\poly(d)$-Lipschitz in $\theta_i$, so we conclude by a standard analysis of SGD with stepsize $\alpha = 1/\poly(d)$ that $(a_i^*)^2$ is a supermartingale with a bounded difference of $1/\poly(d)$. Specifically, let $\widetilde{\nabla}_{\theta_i}\hat{L}$ be the stochastic gradient with $E[\widetilde{\nabla}_{\theta_i}\hat{L}] = \nabla_{\theta_i}\hat{L}$.
%
\[E[a_i^*(\theta_i^{(j+1)})^2 | \theta_{i}^{(j)}] = E[a_i^*(\theta_i^{(j)} - \alpha \widetilde{\nabla}_{\theta_i} \hat{L})^2 |\theta_{i}^{(j)}] \]
\[\geq a_i^*(\theta_i^{(j)})^2 - \alpha E[(\nabla_{\theta_i}(a_i^*)^2)^T\widetilde{\nabla}_{\theta_i}\hat{L}]\]
%
\[+ \poly(d)\alpha^2E[\|\widetilde{\nabla}_{\theta_i}\hat{L} - \nabla_{\theta_i}\hat{L}\|^2 - \|\nabla_{\theta_i}\hat{L}\|^2]\]
%
\[\geq a_i^*(\theta_i^{(j)})^2 + \alpha \|\nabla_{\theta_i}\hat{L}\|^2 - (\alpha/2) \|\nabla_{\theta_i}\hat{L}\|^2 \geq a_i^*(\theta_i^{(j)})^2  \]

Therefore, $X_j = a_i^*(\theta_i^{(j)})^2$ is a supermartingale and by Azuma's, that in $\poly(1/\epsilon,d,\log(1/\zeta))$ iterations, $(a_i^*)^2 = \Omega(\epsilon)$ with probability $1-\zeta$. By choosing $\zeta$ to be exponentially small, we may union bound over the $\poly(d,1/\epsilon)$ iterations of SGD to derive our conclusion.
\end{proof}

%

Under a node-wise SGD algorithm, we can show efficient convergence to global minima under orthogonality assumptions on $w_j$ for these polynomial activations/potentials.

\begin{restatable}{theorem}{polyConv}
\label{PolyConv}
Let $\mathcal{M} = S^{d-1}$. Let $w_1,...,w_d$ be orthonormal vectors in $\R^d$ and $\Phi$ is of the form $\Phi(\theta,w) = (\theta^Tw)^l$ for some fixed integer $l \geq 3$. Furthermore, $1 \leq |b_i|\leq \poly(d)$. 

Then, with high probability, running Algorithm \ref{NodeGDOpt} on \eqref{errLoss} converges to an $\epsilon$-neighborhood of the global minima in $\poly(d,1/\epsilon)$ time. The sample complexity is $\poly(d,1/\epsilon)$.
\end{restatable}


\begin{proof}
Without loss of generality let $w_1,...,w_k$ be the standard basis vectors $e_1,..,e_k$. and these basis vectors span the whole optimization space. Consider the algorithm on just the first node: $(a_1,\theta_1)$. For simplicity, we will drop the subscripts in the proof. 

First, consider the initialization of $\theta$, notice that upon initialization and optimization, if $\theta$ is drawn from a standard Gaussian, then by independence and $\expt[\theta_i] = 0$,
%
\[\expt[C(\theta)^2] = \expt\left[\left(\sum_{i=1}^k  b_i (\theta_i)^l\right)^2\right] = \sum_{i=1}^k b_i^2 \expt[\theta_i^{2l}]\]

There must exists $j$ such that $|b_j|\geq 1$ and since drawing $\theta_i$ uniformly on $S^{d-1}$ is just a $\poly(d)$ rescaling of a Gaussian, we conclude that $E[a^2]\geq1/\poly(d)$. By Theorem \ref{nonDecrease}, we conclude that $E[a^2] \geq 1/\poly(d)$ throughout the gradient descent algorithm. Next, to use Theorem \ref{strongConverge}, we check the regularity conditions. By assumption, we can choose $B, L, \rho$ to be $\poly(d)$ since $\Phi$ and the second and third partials of $\Phi$ are all bounded by $\poly(d)$ in $\Omega$. Furthermore, our activation function $\sigma$ and its derivatives are bounded in magnitude by $O(|x|^{l})$, where $l$ is fixed. By Theorem \ref{genErrBound}, with high probability, we can construct a stochastic oracle up to $\epsilon$ error with sample complexity $\poly(d,1/\epsilon)$.


Therefore, we conclude that we converge to $\theta \in \mathcal{M}_{\poly(\epsilon,1/d)}$ in $\poly(d,1/\epsilon)$ iterations. Note that this is a constrained optimization problem, so by introducing a Lagrange multiplier $\lambda$, the optimality conditions are:
%
\begin{align*}
|\pd{L}{a}| & = |2\sum_{i=1}^k b_i (\theta_i)^l + 2a| \leq \poly(\epsilon,1/d), \textrm{ and } \\
%
 |(\nabla_\theta L)_i| & = |\theta_i||2ab_il(\theta_i)^{l-2}  -2\lambda| \leq \poly(\epsilon,1/d) ,
\end{align*}
where $\lambda$ is chosen to minimize
\[\lambda = \arg \min_\lambda \sum_i (ab_i l (\theta_i)^{l-1} - \lambda\theta_i)^2 = \sum_i ab_i l (\theta_i)^l = -la^2 + \poly(\epsilon,1/d) \]

Therefore, either $|\theta_i| \leq \poly(\epsilon,1/d)$ or $|2ab_il(\theta_i)^{l-2} - 2 \lambda |\leq \poly(\epsilon,1/d)$. Next, the projected Hessian is a diagonal matrix with diagonal entry: 
%
\[(\nabla^2 L)_{ii} = 2a b_i l(l-1)(\theta_i)^{l-2} - 2 \lambda\]

Assume that there exists $\theta_i, \theta_j$ such that $|\theta_i|,|\theta_j| \geq \poly(\epsilon,1/d)$, then we conclude that the other inequality must hold for coordinates $i, j$. So,
%
\begin{align*}
(\nabla^2 L)_{ii} & = 2a b_i l(l-1)(\theta_i)^{l-2} - 2 \lambda \\
& \leq
  2(l-2)\lambda + (l-1)\epsilon \\
&  \leq -2(l-2)la^2 + \poly(\epsilon,1/d)
\end{align*}
%
Since $l-2 \geq 1$ and $a^2 \geq 1/\poly(d)$, we conclude that there
exists a vector with $v_k = 0$ for all $k\neq i, j$ such that
$v^T\theta = 0$ (in the tangent space) and
$v^T(\nabla^2 L) v = -2(l-2)l a^2 + \poly(\epsilon,1/d) <
-\poly(\epsilon,1/d)$.
This contradicts $\theta \in \mathcal{M}_{\poly(\epsilon,1/d)}$, so
$\theta$ is in some $\epsilon$ neighborhood of some
$w_j$. Furthermore, note that $|a_1| \leq \poly(d)$ and there exists
$b_j$ such that $|b_j| \geq 1$ as some $w_i$ has not yet been paired
with a $\theta$.

Now, we proceed with induction and repeat the same argument on $\theta_2$. We can simply treat $\theta_1$ as $w_{k+1}$ and so applying the same argument tells us that $\theta_2$ is close to some $w_j$ for some $j$. The issue is that $\theta_2$ could be in a $\poly(\epsilon,1/d)$-neighborhood of $w_{k+1} = \theta_1$ or $w_i$. We claim that this will not occur. First, since $w_i, w_{k+1}$ are in a $\poly(\epsilon,1/d)$-neighborhood of each other, we will assume WLOG that $\theta_2$ is close to $\theta_1$.

Now, by the optimality of $a_1$, we know that $L(a_1,\theta_1) \leq \min_{a \in \Omega} L(a,\theta_1) + O(\epsilon)$. We claim that if $\theta_2$ is close to $\theta_1$, then $L(a_1+a_2,\theta_1) \leq L(a_1,\theta_1) - \Omega(\epsilon)$. This, combined with the fact that $a_1 + a_2$ is bounded by $\poly(d)$, would lead to a contradiction.

First, notice that since $a_2$ is always optimal, we have
\begin{align*}
& L(a_1,\theta_1) - L(a_1,\theta_1,a_2,\theta_2) \\
& \qquad \qquad = a_2^2 + 2a_2 (\sum_{j < 2} \Phi(\theta_2,\theta_j) + \sum_{j=1}^k b_j \Phi(\theta_2,w_j)) \\
& \qquad \qquad = -a_2^2 = \Omega(\epsilon)
\end{align*}

Therefore, it suffices to show that $|L(a_1+a_2,\theta_1) - L(a_1,\theta_1,a_2,\theta_2) | \leq O(\epsilon)$. Since $L(a_1+a_2,\theta) = L(a_1,\theta_1,a_2,\theta_1)$, this follows immediately from the $\poly(d)$-Lipschitz of $\Phi$ and the fact that $\theta_2$ is in a $\poly(d,1/\epsilon)$-neighborhood of $\theta_1$. We conclude that $\theta_2$ cannot be in a $\poly(\epsilon,1/d)$-neighborhood of $\theta_1$ but converges to a point close to $w_{j}$, $j\neq i,k+1$. Therefore, the no two $\theta_i$ are matched to one $w_j$. 

By applying this logic to all $\theta_i$ through induction, we deduce that $\theta$ is within $\epsilon$ of the global minima.
\end{proof}

