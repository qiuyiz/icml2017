%%%%%%%%%%%%%% Use for definitions
\newcommand{\defeq}{\stackrel{\textup{def}}{=}}

%%%%%%%%%%%%%% Probability stuff
\DeclareMathOperator*{\pr}{\bf Pr}
\DeclareMathOperator*{\av}{\bf E}
\DeclareMathOperator*{\var}{\bf Var}

%%%%%%%%%%%%%% Matrix stuff
\newcommand{\tr}[1]{\mathop{\mbox{$\mathrm{Tr}$}}\left({#1}\right)}
\newcommand{\diag}[1]{{\bf Diag}\left({#1}\right)}

%% Notation for integers, natural numbers, reals, fractions, sets, cardinalities
%%and so on
\newcommand{\nfrac}[2]{\nicefrac{#1}{#2}}
\def\abs#1{\left| #1 \right|}
\newcommand{\norm}[1]{\ensuremath{\left\lVert #1 \right\rVert}}

\newcommand{\floor}[1]{\left\lfloor\, {#1}\,\right\rfloor}
\newcommand{\ceil}[1]{\left\lceil\, {#1}\,\right\rceil}

\newcommand{\pair}[1]{\left\langle{#1}\right\rangle} %for inner product

\newcommand\B{\{0,1\}}      % boolean alphabet  use in math mode
\newcommand\bz{\mathbb Z}
\newcommand\nat{\mathbb N}
\newcommand\rea{\mathbb R}
\newcommand\com{\mathbb{C}}
\newcommand\plusminus{\{\pm 1\}}
\newcommand\Bs{\{0,1\}^*}   % B star use in math mode

\newcommand{\V}[1]{\mathbf{#1}} % Used to denote bold commands
                                % e.g. vectors, matrices
\DeclareRobustCommand{\fracp}[2]{{#1 \overwithdelims()#2}}
\DeclareRobustCommand{\fracb}[2]{{#1 \overwithdelims[]#2}}
\newcommand{\marginlabel}[1]%
{\mbox{}\marginpar{\it{\raggedleft\hspace{0pt}#1}}}
\newcommand\card[1]{\left| #1 \right|} %cardinality of set S; usage \card{S}
\newcommand\set[1]{\left\{#1\right\}} %usage \set{1,2,3,,}
%\renewcommand\complement{\ensuremath{\mathsf{c}}}
\newcommand{\poly}{\mathrm{poly}}
 %\newcommand{\polylog}{{\mathrm{\mbox{polylog}}}}
%\newcommand\poly{\mathrm{poly}}  %usage \poly(n)
\newcommand{\comp}[1]{\overline{#1}}
\newcommand{\smallpair}[1]{\langle{#1}\rangle}
\newcommand{\ol}[1]{\ensuremath{\overline{#1}}\xspace}
\DeclareMathOperator*{\argmin}{arg\,min}
\DeclareMathOperator*{\argmax}{arg\,max}

%%%%%%%%%%%%%% Mathcal shortcuts
\newcommand\calF{\mathcal{F}}
\newcommand\calS{\mathcal{S}}
\newcommand\calG{\mathcal{G}}
\newcommand\calH{\mathcal{H}}
\newcommand\calC{\mathcal{C}}
\newcommand\calD{\mathcal{D}}
\newcommand\calI{\mathcal{I}}
\newcommand\calV{\mathcal{V}}
\newcommand\calK{\mathcal{K}}
\newcommand\calX{\mathcal{X}}
\newcommand\calU{\mathcal{U}}
\newcommand\calE{\mathcal{E}}

%%%%%%%%%%%%%% Author notes %%%%%%%%%%%%%

\definecolor{Mygray}{gray}{0.8}

 \ifcsname ifcommentflag\endcsname\else
  \expandafter\let\csname ifcommentflag\expandafter\endcsname
                  \csname iffalse\endcsname
\fi

\ifnum\showauthornotes=1
\newcommand{\todo}[1]{\colorbox{Mygray}{\color{red}#1}}
\else
\newcommand{\todo}[1]{#1}
\fi

\ifnum\showauthornotes=1
\newcommand{\Authornote}[2]{{\small\color{red}{[#1: #2]}}}
\else
\newcommand{\Authornote}[2]{}
\fi

%%%%%%%%%%%%%% Logical operators
\newcommand\true{\mbox{\sc True}}
\newcommand\false{\mbox{\sc False}}
\def\scand{\mbox{\sc and}}
\def\scor{\mbox{\sc or}}
\def\scnot{\mbox{\sc not}}
\def\scyes{\mbox{\sc yes}}
\def\scno{\mbox{\sc no}}

%% Parantheses
\newcommand{\paren}[1]{\left({#1}\right)}
\newcommand{\sqparen}[1]{\left[{#1}\right]}
\newcommand{\curlyparen}[1]{\left\{{#1}\right\}}
\newcommand{\smallparen}[1]{({#1})}
\newcommand{\smallsqparen}[1]{[{#1}]}
\newcommand{\smallcurlyparen}[1]{\{{#1}\}}

%% short-hands for relational simbols

\newcommand{\from}{:}
% \newcommand\xor{\oplus}
\newcommand\bigxor{\bigoplus}
\newcommand{\logred}{\leq_{\log}}
\def\iff{\Leftrightarrow}
\def\implies{\Rightarrow}
