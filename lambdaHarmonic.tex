
\section{Realizable Potentials with Convergence Guarantees}

{\color{red} Under some assumptions, we
will show that gradient descent can learn at least one of the $w_i's$
of the target network for certain activation functions. The algorithm
will try to learn a guess
$\widetilde{f}(x_j) = \sum_{i=1}^k a_i \sigma(x_j,\theta_i)$ for $f$
and then running gradient descent over the parameters $a_i, \theta_i$
will move them to $b_i, w_i$. We will prove that at convergence, at
least one $\theta_i$ is equal to (or close to) some $w_j$. Note that
we may end up with a many to one mapping of the learned hidden weights
to the true hidden weights, instead of a bijection.


% Next, we provide a partial theoretical justification for this phenomena with simplifying assumptions.
}
 
In this section, we derive convergence guarantees for a class of realizable potentials that closely approximate $\lambda$-harmonic potentials. First, we construct realizable potentials with corresponding activation functions that are approximately $\lambda$-harmonic, specifically they are $\lambda$-harmonic outside of a small neighborhood around the center. We will defer the details of the technical construction to the appendix. Then, we reason similarly about the Laplacian of our loss function to derive our convergence theorem. To make sure that $\|a\|$ remains controlled throughout the optimization process, we add a quadratic regularization term to $L$ and instead run SGD on $G = L + \beta\|a\|^2$. 

By reusing techniques in \cite{GeHJY15}, we show that SGD (with noise) can avoid all points that are have a negative curvature of at least $\epsilon$ in $\poly(1/\epsilon)$ iterations. Intuitively, this means that SGD will converge to points with small gradient and small negative curvature, converging to a point in $\mathcal{M}_{G, \epsilon}$, where 
%
%
\[\mathcal{M}_{G, \epsilon} = \left\{x\in \mathcal{M} \Big| \|\nabla G(x)\|
  \leq \epsilon \text{ and } \lambda_{min}(\nabla^2 G(x)) \geq
  -\epsilon\right\}\]
%
Then, we show that if $(\boldsymbol{a,\theta})$ is in $\mathcal{M}_{G, \epsilon}$ for $\epsilon$ small, then $\theta_i$ is close to $w_j$ for some $j$. Finally, we show how to initialize $(\boldsymbol{a^{(0)},\theta^{(0)}})$ and run SGD to converge to $\mathcal{M}_{G,\epsilon}$, proving our following theorem.
%
\begin{algorithm}[hb]
 \caption{$x = NoisyGD(\widehat{L}, x_0, T,\alpha,\epsilon)$}
   \label{SGD}
\begin{algorithmic}
   \STATE {\bfseries Input:} $\widehat{L}:\mathcal{M} \to \R$; $x_0 \in \mathcal{M}$; $T\in \N$; $\alpha \in \R$; $\epsilon\in\R$
   \vspace{.1in}
   \STATE {\bf Initialize} $x = x_0$
   \WHILE{{\it true}}
   \STATE $x_1 = x$
   \FOR{$i=1$ {\bfseries to} $T$}
   \STATE Sample $\omega$ uniformly on unit sphere.
   \STATE $x_{i+1} = x_i - \alpha(\nabla \widehat{L} (x_i)+\alpha\omega)$ 
%\STATE $x = \Pi_\mathcal{M} x$
   \ENDFOR
    \IF{$\widehat{L}(x_i) - \widehat{L}(x) < \alpha^2$ for all $i$}
   \STATE {\bf return} $x$
   \ELSE 
   \STATE Find $i$ such that $\widehat{L}(x_i) - \widehat{L}(x) \geq \alpha^2$ 
   \STATE $x = x_i$
   \ENDIF
   \ENDWHILE
\end{algorithmic}
\end{algorithm}
%
%
\begin{algorithm}[hb]
 \caption{$x = SecondGD(L, x_0, T,\alpha, \eta, \gamma)$}
   \label{SecondGD}
\begin{algorithmic}
   \STATE {\bfseries Input:} $L:\mathcal{M} \to \R$; $x_0 \in \mathcal{M}$; $T\in \N$; $\alpha \in \R$; $\epsilon\in\R$
   \vspace{.1in}
   \STATE {\bf Initialize} $x = x_0$
   \FOR{$i=1$ {\bfseries to} $T$}
   \IF{$\|\nabla L(x) \geq  \eta$}
   \STATE $x = GD(L, x, 1, \alpha)$
   \ELSE 
   \STATE Find unit eigenvector $v_{min}$ corresponding to $\lambda_{min}(\nabla^2 f(w_t))$ 
    \IF{$\lambda_{min}(\nabla^2 f(w_t)) \leq - \gamma$}
    \STATE $x = x - \alpha \gamma sgn(\nabla f(w_t)^Tv_{min}) v_{min}$
    \ELSE 
     \STATE {\bf return} $x$
    \ENDIF
   \ENDIF
   \ENDFOR
   \end{algorithmic}
\end{algorithm}
%

\begin{theorem}\label{almostHarmSGD}
  Let $\mathcal{M} = \R^{d}$. For all $\epsilon \in (0,1),$ there exists an activation $\sigma$ such that if $w_1,...,w_k \in \R^d$ with $w_i$ randomly chosen from $w_i \sim  \mathcal{N}({\bf 0}, O(d\log k){\bf I_{d\times d}})$ and $b_1,...,b_k$ be randomly chosen at uniform from $[-1,1]$, then with high probability, we can choose an initial point $(\boldsymbol{a^{(0)}, \theta^{(0)}})$ such that after running SecondGD (Algorithm \ref{SecondGD}) on the regularized objective $G(\boldsymbol{a,\theta})$ for at most $\poly(1/\beta,1/\lambda)d^{\sqrt{\lambda}}(d/\epsilon)^{O(d)}$ iterations, there exists an $i, j$ such that $\|\theta_i - w_j\| <  \epsilon$.
\end{theorem}


We start by stating a lemma concerning the construction of an approximately
$\lambda$-harmonic function on $\R^d.$ The construction is given in
Section~\ref{realizable} and is quite technical.
%
%
\begin{restatable}{lemma}{AlmostHarmReal}\label{almostHarmReal}
Let $\mathcal{M} = \R^d$ for $d \equiv 3 \mod 4$. Then, for any $1 > \epsilon > 0$, we can construct a realizable normalized radial potential $\Phi_\epsilon(r)$ that is $\lambda$-harmonic when $r \geq \epsilon$.

Furthermore, we have ${\Phi_\epsilon}^{(d-1)}(r) \geq 0$ for all $r  > 0$ and ${\Phi_\epsilon}^{(k)}(r) \geq 0$ and ${\Phi_\epsilon}^{(k+1)}(r)\leq 0$ for all $r > 0$ and $d - 3 \geq k \geq 0 $ even.

Lastly, $|{\Phi}_\epsilon^{(k)}(r)| \leq 3(2d + \sqrt{\lambda})^{2d} \epsilon^{-2d}e^{\sqrt{\lambda}}$ for all $0 \leq k \leq d-1$. And for $r \geq \epsilon$, $e^{-\sqrt{\lambda}r}r^{2-d}(2d+\sqrt{\lambda})^{-2d}\epsilon^{2d}/3\leq {\Phi}_\epsilon(r) \leq (1+r\sqrt{\lambda})^de^{\sqrt{\lambda}(1-r)}(r)^{2-d}$. Also for $r \geq \epsilon$, $ e^{-\sqrt{\lambda}r}r^{1-d}(2d+\sqrt{\lambda})^{-2d}\epsilon^{2d}/3 \leq |{\Phi}_\epsilon'(r)| \leq (d+\sqrt{\lambda}r)(1+ r\sqrt{\lambda})^de^{\sqrt{\lambda}(1- r)} r^{1-d}$
\end{restatable}
%
%
%
\begin{lemma}\label{almostHarmConv}
Let $\mathcal{M} = \R^d$ for $d \equiv 3 \mod 4$ and let $L$ be as in \eqref{errLoss} corresponding to the activation function $\sigma$ given by Lemma~\ref{almostHarmReal}. For any $\epsilon, \delta > 0$, we can construct $\sigma$ such that if $\boldsymbol{(a,\theta)} \in \mathcal{M}_{G,\delta/k}$, then for all $i$, either 1) there exists $j$ such that $\|\theta_i - w_j\| < \epsilon$ or 2) $a_i^2 < \delta/(\beta\lambda)$, (as long as $\beta \geq \delta/k$)
\end{lemma}
%
\begin{proof}
 The proof is similar to Theorem \ref{EigStrict}. Let $\Phi_\epsilon$ be the realizable potential in \ref{almostHarmReal} such that $\Phi_\epsilon(r)$ is $\lambda$-harmonic when $r \geq \epsilon/k$. Note that $\Phi_\epsilon(0) = 1$ is normalized. And let $\boldsymbol{(a,\theta)} \in \mathcal{M}_{G,\delta/k}$. 
 
WLOG, consider $\theta_1$ and a initial set $S_0 = \{ \theta_1\}$ containing it. For a finite set of points $S$ and a point $x$, define $d(x,S) = \min_{y \in S} \| x - y\|$. Then, we consider the following set growing process. If there exists $\theta_i, w_i \not \in S_j$ such that $d(\theta_i, S_j) < \epsilon/k$ or $d(w_i, S_j) < \epsilon/k$, add $\theta_i, w_i$ to $S_j$ to form $S_{j+1}$. Otherwise, we stop the process. We grow $S_0$ to until the process terminates and we have the grown set $S$.

If there is some $w_j \in S$, then it must be the case that there exists ${j_1},\cdots {j_q}$ such that $\|\theta_1 - \theta_{j_1} \| < \epsilon/k$ and
$\|\theta_{j_{i}} - \theta_{j_{i+1}}\| < \epsilon/k$, and
$\|\theta_{j_q}- w_j\| <\epsilon/k$ for some $w_j$. So, there exists $j$, such that $\|\theta_1 - w_j\| < \epsilon$. 

Otherwise, notice that for each $\theta_i \in S$, $\|w_j - \theta_i\|\geq \epsilon/k$ for all $j$, and $\|\theta_i - \theta_j\| \geq \epsilon/k$ for all $\theta_j\not \in S$. WLOG, let $S = \{\theta_1,\dots,\theta_l\}$. 
  
We consider changing all
$\theta_1, \ldots, \theta_{l}$ by the same $v$ and define 
%
\[H({\bf a}, v) = G({\bf a},\theta_1+v,...,\theta_l+v, \theta_{l+1}
\ldots, \theta_k).\]

The optimality conditions on ${\bf a}$ are 
\begin{align*}
   \abs{\pd{H}{a_i}} & = \lvert (2+2\beta)a_i  + 2\sum_{j\neq i} a_j \Phi_\epsilon(\theta_i,\theta_j) + 2\sum_{j=1}^k b_j \Phi_\epsilon(\theta_i,w_j) \rvert \leq \delta/k
\end{align*}
%
Next, since $\Phi_\epsilon(r)$ is $\lambda$-harmonic for $r \geq \epsilon/k$, we may calculate the Laplacian of $H$ as
%
\begin{align*}
\Delta_v H & = \sum_{i=1}^l \lambda \left(2\sum_{j=1}^k a_ib_j
  \Phi_\epsilon(\theta_i, w_j) + 2\sum_{j=l+1}^k a_ia_j
  \Phi_\epsilon(\theta_i, \theta_j)\right) \\
& = \sum_{i=1}^l \lambda \left(-2\beta a_i^2 -2a_i^2 - 2
  \sum_{j = 1, j\neq i}^l  a_ia_j \Phi_\epsilon(\theta_i,\theta_j)\right)+ \frac{\delta}{k} \sum_{i=1}^l \lambda a_i^2 \\
&= -2\lambda\expt\left[\left( \sum_{i=1}^l a_i \sigma(\theta_i,X)\right)^2\right] -2\beta\lambda \sum_{i=1}^l a_i^2+ \frac{\delta}{k}\lambda\sum_{i=1}^l  a_i^2 \\
\end{align*} 
%
The second line follows from our optimality conditions and the third line follows from completing the square. Since $\boldsymbol{(a,\theta)} \in \mathcal{M}_{G,\delta/k}$, we have $\Delta_v H \geq - \delta$. Because $\beta \geq \delta/k$, we see that we must satisfy $\sum_{i=1}^l a_i^2 \leq \delta/(\beta\lambda)$. Therefore, $a_i^2 \leq \delta/(\beta\lambda)$.

\end{proof}

\begin{lemma}\label{almostHarmRes}
  Assume the conditions of Lemma~\ref{almostHarmConv}. If
$G({\bf a, \boldsymbol{\theta}}) \leq G(\boldsymbol{0,0}) - \delta\sqrt{G(\boldsymbol{0,0})}$
  and $(\boldsymbol{a,\theta}) \in \mathcal{M}_{G,\beta\lambda\delta^2/2k^3}$,
  then there exists some $i, j$ such that $\|\theta_i - w_j\| <\epsilon$.
\end{lemma}
 
 \begin{proof}
 If there does not exists $i, j$ such that
   $\|\theta_i - w_j\| <\epsilon$, then by Lemma \ref{almostHarmConv}, this implies $a_i^2 < \delta^2/2k^2$ for all $i$. Now, for a integrable
   function $f(x)$, $\| f\|_X = \sqrt{\expt_X[f(X)^2]}$ is a
   norm. Therefore, if $f(x) = \sum_i b_i \sigma(w_i,x)$ be our true
   target function, we conclude that by triangle inequality
\begin{align*}
\sqrt{G(\boldsymbol{a,\theta})}  & \geq \norm{\sum_{i=1}^k a_i \sigma(\theta_i,x) - f(x)}_X \\
&\geq \|f(x)\|_X\ - \sum_{i=1}^k \|a_i\sigma(\theta_i,x) \|_X \\
& \geq
  \sqrt{G(\boldsymbol{0,0})} - \delta/2
\end{align*}
Squaring gives a contradiction, so we conclude that there must exist $i, j$ such that $\theta_i$ is in a $\epsilon$ neighborhood of $w_j$.
 \end{proof}
 
 \begin{lemma}[Initialization]\label{almostHarmInitialize}
Assume the conditions of Theorem~\ref{almostHarmSGD} and Lemma~\ref{almostHarmConv}. With high probability, we can initialize $\boldsymbol{(a^{(0)},\theta^{(0)})}$ such that $G({\bf a^{(0)}}, \boldsymbol{\theta^{(0)}}) \leq G(\boldsymbol{0,0}) - 2\delta\sqrt{G(\boldsymbol{0,0})}$ with $\delta = \frac{1}{1+\beta}d^{-O(d\sqrt{\lambda})}(d/\epsilon)^{ - O(d)}$.
 \end{lemma}
 
 \begin{proof}
  Consider choosing $\theta_1 = {\bf 0}$ and then
  optimizing $a_1$. Given $\theta_1$, the loss decrease is:
%
\begin{align*}
   G(a_1,{\bf 0}) - G({\bf 0},{\bf 0}) & = \min_{a_1} (1+\beta)a_1^2 +
  2\sum_{j=1}^k a_1 b_j\Phi_\epsilon({\bf 0},w_j) \\
 %
 & = -\frac{1}{1+\beta}\left(  \sum_{j=1}^k b_j
   \Phi_\epsilon({\bf 0},w_j)\right)^2 
\end{align*}

Because $w_j$ are random Gaussians with variance $O(d \log k)$, we have $\|w_j\| \leq O(d\log k)$ with high probability for all $j$. By Lemma~\ref{almostHarmReal}, our potential satisfies ${\Phi}_\epsilon({\bf 0}, w_j) \geq d^{-O(d\sqrt{\lambda})}(d/\epsilon)^{ - O(d)}$. And since $b_j$ are uniformly chosen in $[-1,1]$, we conclude that with high probability over the choices of $b_j$, $-\frac{1}{2}\left(  \sum_{j=1}^k b_j\Phi(\theta_1,w_j)\right)^2 \geq d^{-O(d\sqrt{\lambda})}(d/\epsilon)^{ - O(d)}$ by appealing to Chebyshev's inequality for uniform variables (separate lemma?).

Therefore, we conclude that with high probability, $G(a_1, {\bf 0}) \leq G(\boldsymbol{0,0}) - \frac{1}{1+\beta}d^{-O(d\sqrt{\lambda})}(d/\epsilon)^{ - O(d)}$. Since $G(\boldsymbol{0,0}) \leq O(k)$, we are done. 


%%Cramer-Rao type Argument
\if{1}
Let $f(x) =  \sum_{j=1}^k b_j \Phi(x,w_j)$, then it suffices (***Justify later) to show that $\Var(f(X)) \geq 1/\poly(d) $, where $X$ is a multivariate standard normal. By appealing to Cramer-Rao type bounds given in \cite{cacoullos1982upper}, we see that
%
\begin{align*}
 \Var(f(X)) &\geq \frac{1}{d}\expt \left[\sum_{i=1}^d \frac{\partial}{\partial x_i} f(X)\right]^2 \\
 &= \frac{1}{d}\left( \sum_{j=1}^k b_j \sum_{i=1}^d \expt \frac{\partial}{\partial x_i}\Phi(X,w_j) \right)^2 \\
\end{align*}

By Stein's identity, $\expt[\frac{\partial}{\partial x_i}\Phi(X,w_j)] = \expt[X_i\Phi(X,w_j)]$. And since $\Phi$ is translationally symmetric, if we let $w_{ji}$ be the $i$-th coordinate of $w_j$, then $\expt[ X_i\Phi(X,w_j)] = \expt[(X_i - w_{ji})\Phi(X, {\bf 0})] = -w_{ji}\expt[\Phi(X,{\bf 0})] = -Cw_{ji}$, where $C$ is a positive constant and $C\geq 1/\poly(d)$ (justify later).

Therefore, $\Var(f(X)) \geq \frac{C^2}{d} \left(\sum_{j=1}^k b_j \sum_{i=1}^d w_{ji}\right)^2$. However, note that $w_{ji}$ are independent Gaussians of variance 1, so we conclude that $\sum_{j=1}^k b_j \sum_{i=1}^d w_{ji}$ is a Gaussian of variance $d\|{\bf b}\|^2 \geq d$. So, with high probability over $w_1,...,w_j$, we conclude that $\Var(f(X)) \geq C^2 \geq 1/\poly(d)$.
\fi
\end{proof}
%



\begin{proof}[Proof of Theorem \ref{almostHarmSGD}]
Let our potential be the one as constructed in Lemma~\ref{almostHarmReal} that is $\lambda$-harmonic for all $r \geq \epsilon/2$. First, by Lemma~\ref{almostHarmInitialize},  we can initialize $\boldsymbol{(a^{(0)},\theta^{(0)})}$ such that $G(\boldsymbol{a^{(0)},\theta^{(0)}}) \leq  G({\bf 0,0}) - 2\delta \sqrt{G({\bf 0,0})}$ for $\delta = d^{-O(d\sqrt{\lambda})}(d/\epsilon)^{-O(d)}$. Since $G({\bf 0, 0})$ is bounded by $O(k)$ and $G \geq 0$, if we set $\alpha = \poly(1/\beta,1/\lambda)d^{\sqrt{\lambda}}(d/\epsilon)^{O(d)}$, then running Algorithm~\ref{SecondGD} will terminate return some $(\boldsymbol{a,\theta})$ in at most $\poly(1/\beta,1/\lambda)d^{\sqrt{\lambda}}(d/\epsilon)^{O(d)}$ iterations since otherwise by Lemma \ref{GradDecrease} and \ref{HessianDecrease}, our objective function cannot decrease by at least $\alpha^3$ (change) too many times.

Assume that there does not exist $\theta_i, w_j$ such that $\|\theta_i - w_j\| < \epsilon$. Then let $\boldsymbol{a',\theta'}$ be point reached from a step of gradient descent starting from $\boldsymbol{a,\theta}$. Since our algorithm has terminated, $G(\boldsymbol{a,\theta}) - G(\boldsymbol{a',\theta'}) \leq \alpha^2$ (change). We claim that $\boldsymbol{a,\theta} \in \mathcal{M}_{G,\beta\lambda\delta^2/2k^2}$. First, Lemma~\ref{almostHarmReal} ensures all first three derivatives of $\Phi$ are bounded by $(d/\epsilon)^{2d}$, except at $w_1,...,w_k$. Now, if $\boldsymbol{a,\theta} \not \in \mathcal{M}_{G,\beta\lambda\delta^2/2k^2}$, then Lemma~\ref{GradDecrease} and ~\ref{HessianDecrease}, we know that $\|\boldsymbol{\theta' - \theta} \| < (d/\epsilon)^{-O(d)}$ and $G(\boldsymbol{a,\theta}) - G(\boldsymbol{a_T,\theta_T}) \geq \alpha^2$, a contradiction. Lastly, since our algorithm maintains that our objective function is decreasing, so $G(\boldsymbol{a,\theta}) \leq G({\bf 0,0}) - 2\delta \sqrt{G({\bf 0,0})}$. Finally, we conclude by Theorem \ref{almostHarmRes}.
\end{proof}

\subsection{Node-by-Node Analysis}
We cannot easily analyze the convergence of GD to the global minima when all $\theta_i$ are simultaneously moving since the pairwise interaction terms between the $\theta_i$ present complications. To derive a tighter control on $(\boldsymbol{a,\theta})$, we run a greedy
node-wise SGD algorithm to learn the hidden weights, i.e. we run a
full SGD algorithm with respect to $(a_i,\theta_i)$ sequentially. The
main idea is that after running SGD with respect to $\theta_1$,
$\theta_1$ should be close to some $w_j$ for some $j$. Then, we can
carefully induct and show that $\theta_2$ must be some other $w_k$ for
$k\neq j$ and so on.

%
\begin{algorithm}[tb]
 \caption{Node-wise Gradient Descent Algorithm with Output Weights Optimization}
   \label{NodeGDOpt}
\begin{algorithmic}
  \STATE {\bfseries Input:}
  $(\boldsymbol{a,\theta}) = (a_1,...,a_k,\theta_1,...,\theta_k), a_i
  \in\R, \theta_i\in\mathcal{M}$;
  $T\in \N$; $\widehat{L}$; $\alpha\in \R$; $\delta \in \R$;
  $\gamma \in R$; \vspace{0.1in} \STATE{\bf Initialize} $\boldsymbol{(a,\theta) = (0,0)}$
  \FOR{$i=1$ {\bfseries to} $k$} 
  \STATE $(a_i, \theta_i) = NoisyGD \left(\widehat{L}_{a_i, \theta_i},(a_i,\theta_i),T, \alpha,\delta \right)$
   \STATE    $(a_1,...,a_i) =  GD \left(\widehat{L}_{a_1,..,a_i},
     (a_1,..,a_i), T , \alpha \right)$\;
   \ENDFOR
   \STATE {\bf return} $a = (a_1,...,a_k), \theta = (\theta_1,..., \theta_k)$
   \end{algorithmic}
\end{algorithm}

To apply our node-wise guarantees, we would first need to first tighten the convergence properties of SGD in the case where we have only 1 variable charge, say $\theta_1$. Let $L_1(a_1,\theta_1)$ be the objective $L$ restricted to $a_1,\theta_1$ being variable, and $a_2,...,a_k = 0$ are fixed. Our better control on the movements of $\theta_1$ allows us to remove our regularization. While our previous guarantees before allow us to reach a $\epsilon$-neighborhood of $w_j$ when running SGD on $L_1$, we actually would like to reach a $(d/\epsilon)^{-O(d)}$-neighborhood of $w_j$. This is accomplished by reasoning about the first derivatives of our potential, mainly by showing that in an $\epsilon$-neighborhood of $w_j$, descending along the gradient always brings $\theta_1$ closer to $w_j$.
%
\begin{lemma}\label{nodewiseSGD}
Let $\mathcal{M} = \R^{d}$. For all $\epsilon \in (0,1),$ there exists an activation $\sigma$ such that if $w_1,...,w_k \in \R^d$ with $w_i$ randomly chosen from $w_i \sim  \mathcal{N}({\bf 0}, O(d\log k){\bf I_{d\times d}})$ and $b_1,...,b_k$ be randomly chosen at uniform from $[-1,1]$, then with high probability, we can choose an initial point $(a^{(0)}, \theta^{(0)})$ such that after running SGD (Algorithm \ref{SGD}) on the restricted regularized objective $L_1(a,\theta)$ for at most $\poly(1/\beta,1/\lambda)d^{\sqrt{\lambda}}(d/\epsilon)^{O(d)}$ iterations, there exists some $w_j$ such that $\|\theta - w_j\| < O(d/\epsilon)^{-O(d)}$ and $|a + b_j| < O(d/\epsilon)^{-O(d)}$.
\end{lemma}


\begin{proof}
First, by Lemma~\ref{almostHarmInitialize},  we can initialize ${(a^{(0)},\theta^{(0)})}$ such that $L_1({a^{(0)},\theta^{(0)}}) \leq  L_1({\bf 0,0}) - 2\delta \sqrt{L_1({\bf 0,0})}$ for $\delta = d^{-O(d\sqrt{\lambda})}(d/\epsilon)^{-O(d)}$. Next, Lemma~\ref{almostHarmReal} ensures all first three derivatives of $\Phi$ are bounded by $O((d/\epsilon)^{2d})$, except at $w_1,...,w_k$. Now, consider running noisy GD to generate a sequence $x_1 = (a^{(1)},\theta^{(1)}),...,x_T = (a^{(T)},\theta^{(T)})$ with step-sizes of $(d/\epsilon)^{-O(d)}$ so that each gradient step moves at most a distance of $(d/\epsilon)^{-O(d)}$. 

If, for all $i$, the iterates $x_i \not \in \mathcal{M}_{G,\beta\lambda\delta^2/2k^2}$ and $\theta_i$ is not within a $(d/\epsilon)^{-O(d)}$ neighborhood of $w_1,...,w_k$, then we derive a contradiction by looking at the decrease in the expected objective function. Since our derivatives are bounded outside of a neighborhood of $w_1,...,w_k$, by Lemma~\ref{GradDecrease} and ~\ref{HessianDecrease}, we know that our expected objective function is decreasing by $(d/\epsilon)^{-O(d)}$ in at most $(d/\epsilon)^{O(d)}$ iterations, if we set $\alpha = \poly(1/\beta,1/\lambda)d^{\sqrt{\lambda}}(d/\epsilon)^{O(d)}$. Since $G({\bf 0, 0})$ is bounded by $O(k)$ and $G \geq 0$, then we derive a contradiction since the expected decrease is bounded by $O(k)$. 

Now, there must exist $x_i \in \mathcal{M}_{G,\beta\lambda\delta^2/2k^2}$ or $x_i$ is within a $(d/\epsilon)^{-O(d)}$ neighborhood of $w_1,...,w_k$ for some $i$. Assume the former holds and the latter does not. First, by Azuma's inequality, in $(d/\epsilon)^{O(d)}$ iterations, we can guarantee that $L_1(\boldsymbol{a,\theta}) \leq L_1({a^{(0)},\theta^{(0)}}) + (d/\epsilon)^{O(d)} \leq L_1({\bf 0,0}) - \delta \sqrt{L_1({\bf 0,0})}$ with high probability for $\delta = d^{-O(d\sqrt{\lambda})}(d/\epsilon)^{-O(d)}$. Finally, we conclude by Theorem \ref{almostHarmRes} that there exists $w_j$ such that $\|\theta - w_j\| < \epsilon$.

Now, our gradient with respect to $\theta$ is
%
\begin{align*}
\nabla_\theta L_1 &= 2ab_j \Phi_\epsilon'(\|\theta - w_j\|) \frac{\theta - w_j}{\|\theta - w_j\|}+ 2\sum_{i\neq j} ab_i\Phi_\epsilon'(\|\theta - w_i\|) \frac{\theta - w_i}{\|\theta - w_i\|}
\end{align*}
%

By our construction, since $\|\theta - w_j\| \leq \epsilon$, we may lower bound $|\Phi_\epsilon'(\|\theta - w_j\|)| \geq e^{-\sqrt{\lambda}\epsilon}\epsilon^{1-d}(2d+\sqrt{\lambda})^{-2d}\epsilon^{2d}/3 \geq O((d/\epsilon)^{-2d})$. On the other hand, for all $i \neq j$, we note that with high probability $\|w_i - w_j\| \geq \Omega(d \log k)$. Therefore, we may upper bound $|\Phi_{\epsilon}(\|\theta - w_i\|)| \leq 1/\poly(k)(d/\epsilon)^{-O(d)}$. Finally, we note that all $|b_i | \geq 1/\poly(k)$ with high probability. 

Together, we conclude that $\nabla_\theta L_1 = 2ab_j \Phi_\epsilon'(\|\theta - w_j\|) \frac{\theta - w_j}{\|\theta - w_j\|} + 2a\eta$, where $\|\eta\| \leq (d/\epsilon)^{-O(d)} b_j \Phi_\epsilon'(\|\theta - w_j\|) \frac{\theta - w_j}{\|\theta - w_j\|}$. 

Similarly, since $(a,\theta) \in \mathcal{M}_{L_1, (d/\epsilon)^{-O(d)}}$, we know that
%
\begin{align*}
   \abs{\pd{L_1}{a}} & = \lvert 2a  + 2b_j \Phi_\epsilon(\|\theta - w_j\|) + 2\sum_{i \neq j} b_i \Phi_\epsilon(\|\theta - w_i\|) \rvert \leq (d/\epsilon)^{-O(d)}
\end{align*}

By a similar argument as on the derivative, we see that $a = -2b_j \Phi_\epsilon(\|\theta - w_j\|) + O(d/\epsilon)^{-O(d)}$ and $|b_j \Phi_\epsilon(\|\theta - w_j\|) | \geq (d/\epsilon)^{-2d}$. By combining our observations, we realize that since noisy GD is moving in the direction of $-\nabla_\theta G_1$, it is moving $\theta$ closer to $w_j$ since 
%
\begin{align*}
\nabla_\theta L_1 &=  2ab_j \Phi_\epsilon'(\|\theta - w_j\|) \frac{\theta - w_j}{\|\theta - w_j\|} + (d/\epsilon)^{-O(d)} \\
&= -b_j^2\Phi_\epsilon(\|\theta-w_j\|)\Phi_\epsilon'(\|\theta - w_j\|) \frac{\theta - w_j}{\|\theta - w_j\|} + (d/\epsilon)^{-O(d)} 
\end{align*}

However, we see that with high probability, $|b_j^2\Phi_\epsilon(\|\theta-w_j\|)\Phi_\epsilon'(\|\theta - w_j\|)| \geq 1/\poly(k)(d/\epsilon)^{-4d}$, which contradicts $\boldsymbol{(a,\theta)}$ in $\mathcal{M}_{G,(d/\epsilon)^{-O(d)}}$ since $\|\nabla_\theta G_1 \| \geq 1/\poly(k)(d/\epsilon)^{-4d}$. 

Therefore, it must be that while running noisy GD, there exists some iterate $x_l$ such that $x_l$ in within a $(d/\epsilon)^{-O(d)}$ neighborhood of some $w_j$. By our observation above, we see that $-\nabla_\theta G_1(x_l) = 2b_j^2\Phi_\epsilon(\|x_l-w_j\|)\Phi_\epsilon'(\|x_l - w_j\|) \frac{x_l - w_j}{\|x_l - w_j\|}+ (d/\epsilon)^{-O(d)}$. Note that $\Phi_\epsilon$ and $\Phi_\epsilon'$ differ in sign, so $x_{l}$ moves in the direction of $w_j$ with negligible remainder terms. Since we are taking step-sizes of $(d/\epsilon)^{-O(d)}$, we conclude that $x_T$ remains also in a $(d/\epsilon)^{-O(d)}$ neighborhood of $w_j$. 

Finally, we see that the charges also converge since $a = -2b_j \Phi_\epsilon(\|\theta - w_j\|) + O(d/\epsilon)^{-O(d)}$ and $\|\theta - w_j\| = (d/\epsilon)^{-O(d)}$. By noting that $\Phi_\epsilon(0) = 1$ and $\Phi_\epsilon$ is $(d/\epsilon)^{2d}$-Lipschitz, we conclude. 
\end{proof}

\begin{theorem}\label{nodewiseSGD}
Let $\mathcal{M} = \R^{d}$ and let $L$ be as in \ref{errLoss} with some activation $\sigma$. For all $\epsilon \in (0,1),$ there exists an activation $\sigma$ such that if $w_1,...,w_k \in \R^d$ with $w_i$ randomly chosen from $w_i \sim  \mathcal{N}({\bf 0}, O(d\log k){\bf I_{d\times d}})$ and $b_1,...,b_k$ be randomly chosen at uniform from $[-1,1]$, then with high probability, after running nodewise SGD (Algorithm \ref{NodeGDOpt}) on the objective $L$ for at most $\poly(1/\beta,1/\lambda)d^{\sqrt{\lambda}}(d/\epsilon)^{O(d)}$ iterations, $\boldsymbol{(a,\theta)}$ is in a $(d/\epsilon)^{-O(d)}$ neighborhood of the global minima.
\end{theorem}

\begin{proof}
Let $(a_i, \theta_i)$ be the $i$-th node that is initialized and applied noisy gradient descent onto. We want to show that the nodes $(a_i, \theta_i)$ will converge, in a node-wise fashion, to some permutation of $\{(b_1,w_1),...,(b_k,w_k)\}$. By Lemma~\ref{nodewiseSGD}, we know that with high probability $(a_1,\theta_1)$ will converge to some $(d/\epsilon)^{-O(d)}$ neighborhood of $(b_{\pi(1)}, w_{\pi(1)})$ for some function $\pi: [k] \to [k]$. Now, since [condition holds], by Lemma~\ref{almostHarmInitialize}, we can initialize ${(a_2^{(0)},\theta_2^{(0)})}$ such that $L_2({a_2^{(0)},\theta_2^{(0)}}) \leq  L_2({\bf 0,0}) - 2\delta \sqrt{L_2({\bf 0,0})}$ for $\delta = d^{-O(d\sqrt{\lambda})}(d/\epsilon)^{-O(d)}$. Then, by Lemma~\ref{nodewiseSGD}, we know that if $x_i = (a_2^{(i)}, \theta_2^{(i)}$ are the noisy GD iterates, then there exists a minimal $l$ such that $x_l$ is in a $(d/\epsilon)^{-O(d)}$ neighborhood of $w_j$ for some $j$. [Needs to be argued further] By Azuma's inequality, $L_2({a_2^{(l)},\theta_2^{(l)}}) \leq  L_2({\bf 0,0}) - \delta \sqrt{L_2({\bf 0,0})}$, with high probability. 

Now, we claim that $x_l$ is not in a $(d/\epsilon)^{-O(d)}$ neighborhood of $(b_{\pi(1)}, w_{\pi(1)})$. Assume otherwise. First, we see that our decrease in objective function implies that $L_{a_1,\theta_1,a_2,\theta_2}(a_1,\theta_1, a_2^{(l)},\theta_2^{(l)}) \leq L_{a_1,\theta_1,a_2,\theta_2}(a_1,\theta_1, {\bf 0,0}) - \delta \sqrt{L_2({\bf 0,0})}$. However, notice that our assumption implies that $a_2^{(l)},\theta_2^{(l)}$ is in a $(d/\epsilon)^{-O(d)}$ neighborhood of $a_1,\theta_1$. By the smoothness of $L$, we see that $L_{a_1,\theta_1,a_2,\theta_2}(a_1,\theta_1, a_2^{(l)},\theta_2^{(l)}) \geq L_{a_1,\theta_1,a_2,\theta_2}(a_1,\theta_1, a_2^{(l)},\theta_1) - (d/\epsilon)^{-O(d)} = L_{a_1,\theta_1,a_2,\theta_2}(a_1 + a_2^{(l)},\theta_1, {\bf 0, 0}) - (d/\epsilon)^{-O(d)} $. Together, we know that  $L_{a_1,\theta_1,a_2,\theta_2}(a_1+a_2^{(l)},\theta_1,{\bf 0, 0}) \leq L_{a_1,\theta_1,a_2,\theta_2}(a_1,\theta_1, {\bf 0,0}) - (d/\epsilon)^{-O(d)}$

However, by Theorem \ref{quadConverge}, our gradient descent guarantees that $L_{a_1,\theta_1,a_2,\theta_2}(a_1,\theta_1, {\bf 0,0}) - L_{a_1,\theta_1,a_2,\theta_2}(a_1 + a_2^{(l)},\theta_1, {\bf 0, 0}) \leq O(1/T)$, where $T$ is the number of iterations of gradient descent. Since $T = (d/\epsilon)^{-O(d)}$ is large enough, we derive a contradiction. Therefore, our claim is done and by induction, $\pi$ is a permutation. Now, our theorem follows. 
\end{proof}


%%% Local Variables:
%%% mode: latex
%%% TeX-master: "icmlpaper2017.tex"
%%% End: