\section{Deep Learning, Potentials, and Electron-Proton Dynamics}

\subsection{Preliminaries}
We focus on learning depth two networks with a linear activation on
the output node. If the network takes inputs $x \in \R^d$ (say from
some distribution $\mathcal{D}$), then the network output, denoted
$f(x)$ is a sum over $k = \poly(d)$ hidden units with weight vectors
$w_{i} \in \R^d,$ activation $\sigma(x,w):\R^d \times \R^d\to \R,$ and
output weights $b_i \in \R.$ Thus, we can write
$f(x) = \sum_{i=1}^k b_i\sigma(x,w_i)$. We denote this concept class
$\mathcal{C}_{\sigma,k}.$ Our hypothesis concept class is also
$\mathcal{C}_{\sigma,k}.$

% of our learning
% procedure is the output of a two-layer neural network with linear
% output activation and has the form $f(x) = \sum_{i=1}^k
% b_i\sigma(x,w_i)$. 


%  where $\sigma(x,w):\R^d \times \R^d\to \R$
% is the activation function that takes in a hidden weight vector
% $w \in \R^d$ and the input vector $x \in \R^d$.  

% given by the form
% $f(x) = \sum_{i=1}^k b_i\sigma(x,w_i)$. 

{\color{red} Under some assumptions, we
will show that gradient descent can learn at least one of the $w_i's$
of the target network for certain activation functions. The algorithm
will try to learn a guess
$\widetilde{f}(x_j) = \sum_{i=1}^k a_i \sigma(x_j,\theta_i)$ for $f$
and then running gradient descent over the parameters $a_i, \theta_i$
will move them to $b_i, w_i$. We will prove that at convergence, at
least one $\theta_i$ is equal to (or close to) some $w_j$. Note that
we may end up with a many to one mapping of the learned hidden weights
to the true hidden weights, instead of a bijection.


% Next, we provide a partial theoretical justification for this phenomena with simplifying assumptions.
}
 
% The target concept class $\mathcal{C}_{\sigma,k}$ of our learning
% procedure is the output of a two-layer neural network with linear
% output activation and has the form $f(x) = \sum_{i=1}^k
% b_i\sigma(x,w_i)$. 

Let $\boldsymbol{a} = (a_1,...,a_k)$ and $\boldsymbol{\theta} = (\theta_1,...,\theta_k)$; similarly for $\boldsymbol{b}, \boldsymbol{w}$. For some distribution $\mathcal{D}$, we can given $n$ pairs of training data $(x_i, f(x_i))$, where $x_i$ are drawn i.i.d. from $\mathcal{D}$. Our hypothesis class is the same as the concept class and our loss function is the squared loss. 
%
\begin{equation}\label{errEmp}
\widehat{L}(\boldsymbol{a,\theta})  = \frac{1}{n}\sum_{j=1}^n \left(\sum_{i=1}^k b_i\sigma(x_j,w_i) - \sum_{i=1}^k a_i \sigma(x_j,\theta_i)\right)^2
\end{equation}
%
We would rather work with the true loss directly:
$L(\boldsymbol{a,\theta}) = \expt[\widehat{L}]$. To further simplify $L$, let us
re-parametrize $a$ with $-a$ and expand.
\begin{align}
& L(\boldsymbol{a,\theta})  = \expt_{X\sim D}\left[ \left(
  \sum_{i=1}^k a_i \sigma(X,\theta_i) + \sum_{i=1}^k
  b_i\sigma(X,w_i)\right)^2\right] \nonumber \\
%
& = \sum_{i=1}^k \sum_{j=1}^k a_i a_j \Phi(\theta_i,\theta_j)+ 2 a_ib_j \Phi(\theta_i,w_j)+ b_i b_j \Phi(w_i,w_j)
 \label{errLoss}
\end{align}
%
where $\Phi(\theta, w) = \expt_{X\sim D}[ \sigma(X,\theta) \sigma(X,w)]$ is
the {\bf potential function} {\it corresponding to the activation function} $\sigma$. Although our loss is jointly non-convex but is quadratic in $\boldsymbol{a}$, so we have convergence guarantees to $\boldsymbol{a^*(\theta)}$, the optimal set of output weights of a given $\boldsymbol{\theta}$. 

We will optimize over the familiar $\mathcal{M}= \R^d$ but often over $\mathcal{M} = S^{d-1}$. Let $\Pi_\mathcal{M}$ be the projection operator on $\mathcal{M}$. On $\R^d$, the gradient and Hessian $\nabla_{\R^d} f, \nabla_{\R^d}^2 f$ is defined as standard and the Laplacian is simply $\Delta_{\R^d} f = \Tr(\nabla_{\R^d}^2 f)$. The simplest way to define these terms on $S^{d-1}$ is $\nabla_{S^{d-1}} f(x) = \nabla_{\R^d} f(x/\|x\|)$, where $\| \cdot \|$ denotes the $l_2$ norm and $x \in S^{d-1}$. The Hessian and Laplacian are analogously defined and the subscripts are usually dropped where clear from context. If $f$ is multivariate with variable $x_i$, then let $f_{x_i}$ be a
restriction of $f$ onto the variable $x_i$ with all other variables
fixed. Let $\nabla_{x_i}f, \Delta_{x_i}f$ to be the gradient and
Laplacian, respectively, of $f_{x_i}$ with respect to
$x_i$. Lastly, we say $x$ is a critical point of $f$ if $\nabla f$
does not exist or $\nabla f = 0$.

A gradient-based method is an algorithm that calls Algorithm \ref{GD}
in all the optimization procedures to minimize the empirical
loss. Given $\mathcal{D}$, the training data, and the activation
function $\sigma$, we attempt to show that there exists a gradient-based
algorithm that optimizes $\widehat{L}$ on $\mathcal{M}$ and finds
$\widetilde{f} \in \mathcal{C}_{\sigma, k}$ such that with high probability, $\theta_i$ is in an $\epsilon$-neighborhood of $w_j$ for some (or all) $i, j$ in time $\poly(d,1/\epsilon)$ or some
other meaningful bound.

\begin{algorithm}[hb]
 \caption{$x = GD(L,x_0, T,\alpha$)}
   \label{GD}
\begin{algorithmic}
   \STATE {\bfseries Input:} $L: \mathcal{M} \to \R$; $x_0 \in \mathcal{M}$; $T\in \N$; $\alpha\in \R$
   \STATE Initialize $x = x_0$
   \FOR{$i=1$ {\bfseries to} $T$}
   \STATE $x = x - \alpha\nabla L(x)$
   \STATE $x = \Pi_\mathcal{M} x$
   \ENDFOR
\end{algorithmic}
\end{algorithm}

\subsection{Activation-Potential Correspondence}
We restrict our attention to potential (and activation) functions with some natural symmetry, so they are either {\it translationally or rotationally invariant}. Specifically, we may write $\Phi= h(\theta-w)$ for some function $h(x)$  in the first case, with $\theta, w \in \R^d$. In the second case, $\Phi = h(\theta^Tw)$ and we enforce $\theta, w \in S^{d-1}$. Such potentials are called {\bf symmetric}. Our results focus on rotationally invariant potentials, as they correspond to classical neural networks.

{\bf Remark}: Symmetric potentials satisfy $\Phi(\theta,\theta)$ {\it is
  a positive constant and we will also normalize
  $\Phi(\theta,\theta) = 1$ for the rest of the paper.} 
  
We make a distributional assumption that our input distribution
$\mathcal{D} = \mathcal{N}(0, {\bf I_{d\times d}})$ is fixed as the
standard Gaussian in $\R^d$. This assumption is not critical and a
simpler distribution might lead to better statistical bounds. However, if we allow arbitrary distributions, then hardness results in PAC-learning
halfspaces would apply \cite{klivans2006cryptographic}.

We call a potential function {\bf realizable} if it corresponds to some activation $\sigma$.  We briefly state some results about characterizations of realizable potentials for translationally and rotationally invariant potentials. Full proofs and calculations of activation-potential correspondences, such as those claimed in Table \ref{table1}, can be found in the supplementary material. 
%
\begin{restatable}{theorem}{tranReal}
\label{thm:tranReal}
Let $\mathcal{M} = \R^d$ and $\Phi$ is $\FT$-integrable. Then, $\Phi$ is realizable if $\mathfrak{F}(\Phi)(\omega) \geq 0$ and the corresponding activation is 
\[\sigma(x) = (2\pi)^{1/4}e^{x^2/4}\mathfrak{F}^{-1}(\sqrt{\mathfrak{F}(\Phi)})(x)\]

where $\mathfrak{F}$ is the generalized Fourier transform in $\R^d$
\end{restatable}
%
\begin{restatable}{theorem}{rotReal}
\label{thm:rotReal}
Let $\mathcal{M} = S^{d-1}$ and $\Phi(\theta,w) = f(\theta^Tw)$. Then, $\Phi$ is realizable if $f$ has non-negative Taylor coefficients, $c_i \geq 0$ , and the corresponding activation
\[\sigma(x) = \sum_{i=1}^\infty \sqrt{c_i} h_i(x)\]
converges almost everywhere, where $h_i(x)$ is the i-th Hermite polynomial.
\end{restatable}
\Anote{give examples of phi's and corresponding sigmas in in a paragraph.}
%
\subsection{Electron-Proton Dynamics}

%
% Our main observation is that the gradient descent dynamics of learning such to layer networks is equivalent to the dynamics of a set of proton-electron charges under a certain electrical attraction force function. 
{\color{red} Assume for now that the coefficients $b_i$ and $a_i$ are $1$. 
Thus we are only perform gradient descent over $\theta_i$ to minimize the expected square loss of $f-\widetilde{f}$.
%
The charges reside in $R^d$. The protons are fixed at locations $w_1,..,w_k$. The electrons are at positions $\theta_1,..,\theta_k$ and can move: the total force on each charge is the sum of the pairwise forces, determined by the gradient of the potential function $\Phi(\theta, w) = \expt_{X\sim \mathcal{D}}[ \sigma(X,\theta) \sigma(X,w)]$.  }

By interpreting the pairwise potentials as electrostatic attraction
potentials, we notice that our dynamics is similar to electron-proton
type dynamics under some potential $\Phi$, where $w_i$ are fixed point
charges in $\R^d$ and $\theta_i$ are moving point charges in $\R^d$
that are trying to find $w_i$. We note that standard electron-proton dynamics interprets the force between particles as an acceleration vector; in our case, it is more accurately interpreted as a velocity vector. 
%
\begin{definition}\label{EPDef}
  Given a potential $\Phi : \mathcal{M} \times \mathcal{M} \to \R$ and
  particle locations $\theta_1,...,\theta_k \in \mathcal{M}$ along
  with their respective charges $a_1,...,a_k \in \R$. We define {\bf
    Electron-Proton Dynamics} under $\Phi$ with some subset $S \subseteq [k]$ of fixed particles to be the solution $(\theta_1(t),...,\theta_k(t))$ to the following system of
  differential equations: For each pair $(\theta_i,\theta_j)$, there
  is a force from $\theta_j$ exerted on $\theta_i$ that is given by
  ${\bf F}_{i}(\theta_j) = a_ia_j\nabla_{\theta_i}
  \Phi(\theta_i,\theta_j)$ and
\[-\frac{d\theta_i}{dt} =  \sum_{ j \neq i} {\bf F}_{i}(\theta_j)\]
for all $i \not \in S$, with $\theta_i(0) = \theta_i$. For $i \in S$, $\theta_i(t) =  \theta_i$.
\end{definition}
%
\begin{restatable}{theorem}{epdyn}
\label{EPDyn}
Let $\Phi$ be a symmetric potential and $L$ be as in \eqref{errLoss}. Running continuous gradient descent on $\frac{1}{2} L$ with respect to $\theta$, initialized at
$(\theta_1,...,\theta_k)$ produces the same dynamics as
Electron-Proton Dynamics under $2\Phi$ with fixed particles at
$w_1,...,w_k$ with respective charges $b_1,..,b_k$ and moving
particles at $\theta_1,...,\theta_k$ with respective charges
$a_1,...,a_k$.
\end{restatable} 
%
%

%%% Local Variables:
%%% mode: latex
%%% TeX-master: "icmlpaper2017.tex"
%%% End: