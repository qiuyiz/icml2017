
\section{Realizable Potentials}
\label{realizable}

\subsection{Activation-Potential Calculations}
First define the {\it dual} of a function $f: \R \to \R$ is defined to be 
%
\[ \widehat{f}(\rho) = E_{X,Y \sim N(\rho)}[f(X)f(Y)],\]
%
where $N(\rho)$ is the bivariate normal distribution with $X, Y$ unit variance and $\rho$ covariance. This is as in \cite{DanielyFS16}.
%
\begin{lemma}\label{rotLem}
Let $\mathcal{M} = S^{d-1}$ and $\sigma$ be our activation function, then $\widehat{\sigma}$ is the corresponding potential function.
\end{lemma}

\begin{proof}
If $u, v$ have norm 1 and if $X$ is a standard Gaussian in $\R^d$, then note that $X_1 = u^TX$ and $X_2 = v^TX$ are both standard Gaussian variables in $\R^1$ and the covariance is $E[X_1X_2] = u^Tv$. 

Therefore, the dual function of the activation gives us the potential function.
\begin{align*}
E_{X}[\sigma(u^TX)\sigma(v^TX)] & =
E_{X,Y \sim N(u^Tv)}[\sigma(X)\sigma(Y)] \\
& = \widehat{\sigma}(u^Tv).
\end{align*}
\end{proof}

By Lemma \ref{rotLem}, the calculations of the activation-potential
for the sign, ReLU, Hermite, exponential functions are given in
\cite{DanielyFS16}. For the Gaussian and Bessel activation functions,
we can calculate directly. In both case, we notice that we may write
the integral as a product of integrals in each dimension. Therefore,
it suffices to check the following 1-dimensional identities.
\begin{align*}
  & \int_{-\infty}^\infty
    \sqrt{2}e^{x^2/4}e^{-(x-\theta)^2}\sqrt{2}e^{x^2/4}e^{-(x-w)^2} \frac{1}{\sqrt{2\pi}} e^{-x^2/2}\, dx \\
  & \qquad = \sqrt{\frac{2}{\pi}}\int_{-\infty}^\infty
    e^{-(x-\theta)^2}e^{-(x-w)^2} \, dx = e^{-(\theta -w)^2/2}
\end{align*}
\begin{align*}
& \int_{-\infty}^\infty (\frac{2}{\pi})^{3/2}e^{x^2/2}K_0(|x-\theta|)K_0(|x-w|)  \frac{1}{\sqrt{2\pi}} e^{-x^2/2}\, dx \\
& \qquad 
= \int_{-\infty}^\infty \frac{2}{\pi^2}K_0(|x-\theta|)K_0(|x-w|) \, dx
  = e^{-|\theta -w|}
\end{align*}


The last equality follows by Fourier uniqueness and taking the Fourier transform of both sides, which are both equality $\sqrt{2/\pi}(\omega^2+1)^{-1}$. 


\subsection{Characterization Theorems}

\tranReal*

\begin{proof}
Let $h(x) = \FT^{-1}(\sqrt{\FT(f)})(x)$ and this is well-defined since the fourier transform was non-negative everywhere. Now, let $\sigma(x,w) = (2\pi)^{1/4}e^{x^2/4}h(x-w)$ and by assumption, it is bounded almost everywhere. Realizability follows by checking:
%
\begin{align*}
    E_{X \sim N}[\sigma(X,w)\sigma(X,\theta)]  &= \int_{\R^n} h(x-w)h(x-\theta) \, dx \\
    &= \int_{\R^n} h(x)h(x-(\theta-w)) \, dx \\
    &= \FT^{-1}(\FT(h\ast h)(\theta -w)) \\
    &= \FT^{-1}(\FT(h)^2(\theta - w)) \\
    &= \FT^{-1}(\FT(f)(\theta - w)) \\
    &= f(\theta - w) 
\end{align*}
\end{proof}




\rotReal*

\begin{proof}
By \ref{rotLem} and due to the orthogonality of hermite polynomials, if $f = \sum_i a_i h_i$, where $h_i(x)$ is the i-th Hermite polynomial, then
%
\[\widehat{f}(\rho) = \sum_{i} a_i^2 \rho^i\]

Therefore, any function with non-negative taylor coefficients is a valid potential function, with the corresponding activation function determined by the sum of hermite polynomials, and the sum is bounded almost everywhere by assumption.
\end{proof}

\subsection{Further Characterizations}

\begin{definition}
Let $\Phi$ be $\FT$-integrable if $\FT({\Phi}(\omega))$ is integrable, where $\FT$ is the standard Fourier transform.
\end{definition}

\begin{lemma}
Let $\mathcal{M} = \R^{d}$. If $\Phi_1, \Phi_2$ are $\FT$-integrable realizable potentials, then $\Phi_1 + \Phi_2$ is $\FT$-integrable realizable.
\end{lemma}

\begin{proof}
Since $\Phi_1, \Phi_2$ is $\FT$-integrable, then since the Fourier transform is linear, $\Phi_1+\Phi_2$ is also $\FT$-integrable. Since $\Phi_1, \Phi_2$ are realizable, they are positive definite and by Bochner's theorem, their Fourier transforms are non-negative. Therefore, we know $\FT(\Phi_1 + \Phi_2)\geq 0$ and $\sqrt{\FT(\Phi_1 + \Phi_2)}$ is defined and is square-integrable. By properties of the Fourier transform, $\FT^{-1}(\sqrt{\FT(\Phi_1 + \Phi_2)})$ exists and is also square-integrable, therefore bounded almost anywhere. Lastly, by Theorem \ref{thm:tranReal}, we conclude that $\Phi_1 + \Phi_2$ is realizable.
\end{proof}

\begin{corollary}
Let $w(x) \geq 0$ be a positive weighting function such that $\int_a^b w(x) \, dx$ is bounded. If $\Phi_x$ is a parametrized family of $\FT$-integrable realizable potentials, then, $\int_a^b w(x) \Phi_x$ is $\FT$-integrable realizable.
\end{corollary}

\begin{proof}
Let $\Phi = \int_a^b w(x) \Phi_x$. From linearity of the Fourier transform and $\int_a^b w(x)\, dx$ is bounded, we know that $\Phi$ is $\FT$-integrable. Then, by the continuity of Fourier transform, we can argue similarly as the previous lemma.
\end{proof}



\begin{lemma}\label{baseConstruct}
Let $\mathcal{M} = \R^d$ for $d \equiv 3 \mod 4$. Then, for any $\epsilon, t > 0$, there exists a $\FT$-integrable realizable $\Phi$ such that for $t \geq r > \epsilon$, $\Phi^{(d-1)}(r) = t -r$ and $\Phi^{k}(r) = 0$ for $r > t$ for all $0 \leq k \leq d$.
\end{lemma}

\begin{proof}
First, consider a radial activation function $\sigma(x, \theta) = \bf{1}_{\|\theta - x\| \leq t/2}$, which is the indicator in the disk of radius $t/2$. This activation, when weighted correctly, gives rise to a realizable radial potential that measures the volume of the intersection of two spheres of radius $t$ centered at $\theta$ and $w$.
%
\[\Phi_t(\theta, w) = E_X[\sigma(X,\theta)\sigma(X,w)]\]
\[= \begin{cases}
C\int_{\|\theta - w\|/2}^{t/2} ((t/2)^2 - x^2)^{(d-1)/2} \, dx & \|\theta - w\| \leq t\\
0 & o.w.\\
\end{cases}\]
%
Therefore, as a function of $r=\|\theta - w \|$, we see that when $r \leq t$, $\Phi_t(r) = C\int_{r/2}^{t/2} ((t/2)^2-x^2)^{(d-1)/2} \, dx$ and $\Phi_t'(r) = -C'((t/2)^2-(r/2)^2)^{(d-1)/2}$. Since $d \equiv 3 \mod 4$, we notice that $\Phi_t'$ has a positive coefficient in the leading $r^{d-1}$ term and since it is a function of $r^2$, it has a zero $r^{d-2}$ term. Therefore, we can scale $\Phi_t$ such that 

\[\Phi_t^{(d-1)}(r) = \begin{cases}
r & r \leq t\\
0 & o.w. \\
\end{cases} \]

Now, for any $\epsilon > 0$, let us construct our desired $\Phi$ by taking a positive sum of $\Phi_t$. $\Phi_t$ is clearly realizable and is $\FT$-integrable since it is continuous and has compact support. Consider

\[\Phi(r) = \int_{\epsilon}^{t} \frac{1}{x^2}\Phi_x(r) \, dx\]

First, note that the total weight $\int_\epsilon^t \frac{1}{x^2}$ is bounded. Then, when $r \geq t$, since $\Phi_x(r) = 0$ for $x \leq t$, we conclude that $\Phi^{(k)}(r) = 0$ for any $k$. Otherwise, for $\epsilon < r < t$, we can interchange limits by non-negativity to get

\[\Phi^{(d-1)}(r) = \int_{\epsilon}^r \frac{1}{x^2}\Phi_x^{(d-1)}(r) \, dx + \int_{r}^t \frac{1}{x^2} \Phi_x^{(d-1)}(r) \, dx\]

\[ = 0 + \int_r^t \frac{r}{x^2} \, dx = 1 -r/t \]

Scaling by $t$ gives our desired claim.
\end{proof}

\begin{lemma}\label{transConstruct}
Let $\mathcal{M} = \R^d$ for $d \equiv 3 \mod 4$ and let $\Phi(r)$ be a radial potential. Also, $\Phi^{(d)}(r) \leq 0$ and $\Phi^{(d+1)}(r)\geq 0$ for all $r > 0$ and $\lim_{r \to \infty} \Phi^{(k)}(r) = 0$ for all $0 \leq k \leq d$. 

Then, for any $\epsilon > 0$, there exists a $\FT$-integrable realizable potential $\overline{\Phi}$ $\Phi$ such that $\overline{\Phi}^{(k)}(r) = \Phi^{(k)}(r)$ for all $0 \leq k \leq d-1$ and $r \geq \epsilon$. 
\end{lemma}

\begin{proof}
By Lemma \ref{baseConstruct}, we can find $\Phi_t$ such that for $r \geq \epsilon$, we have $\Phi_t^{(d-1)}(r) = t - r $. Therefore, consider 

\[\overline{\Phi}(r) = \int_{\epsilon}^\infty \Phi^{(d+1)}(x) \Phi_x(r) \, dx\] 

Note that this is a positive sum with $\int_{\epsilon}^\infty \Phi^{(d+1)}(x) \, dx = -\Phi^{(d)}(\epsilon) < \infty$. By the non-negativity of $-\Phi^{(d)}_x$, we can interchange limits to get

\[\overline{\Phi}^{(d-1)}(r) = \int_{\epsilon}^\infty  \Phi^{(d+1)}(x) (\Phi_x^{(d-1)}(r)) \, dx\] 

\[ = \int_r^\infty \Phi^{(d+1)}(x) \int_r^x 1 \, dy \,dx \]

\[= \int_r^\infty \int_y^\infty \Phi^{(d+1)}(x) \, dx \, dy = \int_r^\infty  -\Phi^{(d)}(y) \, dy\]

\[ = \Phi^{(d-1)}(r)\]

Now, since $\overline{\Phi}^{(d-1)}(r) = \Phi^{(d-1)}(r)$ for $r\geq \epsilon$ and $\lim_{r\to\infty} \overline{\Phi}^{k}(r) = \lim_{r\to\infty} {\Phi}^{k}(r) = 0$ for $0 \leq k \leq d-1$, repeated integration gives us our claim.
\end{proof}

\begin{corollary}
Let $\mathcal{M} = \R^d$ for $d = 3$. Then, for any $\epsilon > 0$, we can construct a radial potential $\Phi_\epsilon$ that is $\lambda$-harmonic for $r \geq \epsilon$.
\end{corollary}

\begin{proof}
Consider $\Phi(r) = e^{-r}/r$. Then, by our formula for the 3-dimensional Laplacian,

\[\Delta \Phi =  \frac{1}{r^2} \frac{\partial}{\partial r} (r^2 \frac{\partial \Phi}{\partial r}) = \Phi\]

We can check that $\Phi^{(k)}(r)$ is non-negative for $k$ even and non-positive for $k$ odd. So, by Lemma \ref{transConstruct}, our claim follows.
\end{proof}