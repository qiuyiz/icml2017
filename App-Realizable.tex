
\section{Realizable Potentials}
\label{realizable}

\subsection{Activation-Potential Calculations}
First define the {\it dual} of a function $f: \R \to \R$ is defined to be 
%
\[ \widehat{f}(\rho) = E_{X,Y \sim N(\rho)}[f(X)f(Y)],\]
%
where $N(\rho)$ is the bivariate normal distribution with $X, Y$ unit variance and $\rho$ covariance. This is as in \cite{DanielyFS16}.
%
\begin{lemma}\label{rotLem}
Let $\mathcal{M} = S^{d-1}$ and $\sigma$ be our activation function, then $\widehat{\sigma}$ is the corresponding potential function.
\end{lemma}

\begin{proof}
If $u, v$ have norm 1 and if $X$ is a standard Gaussian in $\R^d$, then note that $X_1 = u^TX$ and $X_2 = v^TX$ are both standard Gaussian variables in $\R^1$ and the covariance is $E[X_1X_2] = u^Tv$. 

Therefore, the dual function of the activation gives us the potential function.
\begin{align*}
E_{X}[\sigma(u^TX)\sigma(v^TX)] & =
E_{X,Y \sim N(u^Tv)}[\sigma(X)\sigma(Y)] \\
& = \widehat{\sigma}(u^Tv).
\end{align*}
\end{proof}

By Lemma \ref{rotLem}, the calculations of the activation-potential
for the sign, ReLU, Hermite, exponential functions are given in
\cite{DanielyFS16}. For the Gaussian and Bessel activation functions,
we can calculate directly. In both case, we notice that we may write
the integral as a product of integrals in each dimension. Therefore,
it suffices to check the following 1-dimensional identities.
\begin{align*}
  & \int_{-\infty}^\infty
    \sqrt{2}e^{x^2/4}e^{-(x-\theta)^2}\sqrt{2}e^{x^2/4}e^{-(x-w)^2} \frac{1}{\sqrt{2\pi}} e^{-x^2/2}\, dx \\
  & \qquad = \sqrt{\frac{2}{\pi}}\int_{-\infty}^\infty
    e^{-(x-\theta)^2}e^{-(x-w)^2} \, dx = e^{-(\theta -w)^2/2}
\end{align*}
\begin{align*}
& \int_{-\infty}^\infty (\frac{2}{\pi})^{3/2}e^{x^2/2}K_0(|x-\theta|)K_0(|x-w|)  \frac{1}{\sqrt{2\pi}} e^{-x^2/2}\, dx \\
& \qquad 
= \int_{-\infty}^\infty \frac{2}{\pi^2}K_0(|x-\theta|)K_0(|x-w|) \, dx
  = e^{-|\theta -w|}
\end{align*}


The last equality follows by Fourier uniqueness and taking the Fourier transform of both sides, which are both equality $\sqrt{2/\pi}(\omega^2+1)^{-1}$. 


\subsection{Characterization Theorems}

\tranReal*

\begin{proof}
Let $h(x) = \mathfrak{F}^{-1}(\sqrt{\mathfrak{F}(f)})(x)$ and this is well-defined since the fourier transform was non-negative everywhere. Now, let $\sigma(x,w) = (2\pi)^{1/4}e^{x^2/4}h(x-w)$ and by assumption, it is bounded almost everywhere. Realizability follows by checking:
%
\begin{align*}
    E_{X \sim N}[\sigma(X,w)\sigma(X,\theta)]  &= \int_{\R^n} h(x-w)h(x-\theta) \, dx \\
    &= \int_{\R^n} h(x)h(x-(\theta-w)) \, dx \\
    &= \mathfrak{F}^{-1}(\mathfrak{F}(h\ast h)(\theta -w)) \\
    &= \mathfrak{F}^{-1}(\mathfrak{F}(h)^2(\theta - w)) \\
    &= \mathfrak{F}^{-1}(\mathfrak{F}(f)(\theta - w)) \\
    &= f(\theta - w) 
\end{align*}
\end{proof}




\rotReal*

\begin{proof}
By \ref{rotLem} and due to the orthogonality of hermite polynomials, if $f = \sum_i a_i h_i$, where $h_i(x)$ is the i-th Hermite polynomial, then
%
\[\widehat{f}(\rho) = \sum_{i} a_i^2 \rho^i\]

Therefore, any function with non-negative taylor coefficients is a valid potential function, with the corresponding activation function determined by the sum of hermite polynomials, and the sum is bounded almost everywhere by assumption.
\end{proof}
