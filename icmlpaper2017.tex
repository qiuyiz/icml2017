%%%%%%%%%%%%%%%%%%%%%%%%%%%%%%%%%%%%%%%%%%%%%%%%%%%%%%%%%%%%%%%%%%
%%%%%%%% ICML 2017 EXAMPLE LATEX SUBMISSION FILE %%%%%%%%%%%%%%%%%
%%%%%%%%%%%%%%%%%%%%%%%%%%%%%%%%%%%%%%%%%%%%%%%%%%%%%%%%%%%%%%%%%%

% Use the following line _only_ if you're still using LaTeX 2.09.
%\documentstyle[icml2017,epsf,natbib]{article}
% If you rely on Latex2e packages, like most moden people use this:
\documentclass{article}

% use Times
\usepackage{times}
% For figures
\usepackage{graphicx} % more modern
%\usepackage{epsfig} % less modern
\usepackage{subfigure} 

% For citations
\usepackage{natbib}

% For algorithms
\usepackage{algorithm}
\usepackage{algorithmic}


% As of 2011, we use the hyperref package to produce hyperlinks in the
% resulting PDF.  If this breaks your system, please commend out the
% following usepackage line and replace \usepackage{icml2017} with
% \usepackage[nohyperref]{icml2017} above.
\usepackage{hyperref}


% Packages hyperref and algorithmic misbehave sometimes.  We can fix
% this with the following command.
\newcommand{\theHalgorithm}{\arabic{algorithm}}

% Employ the following version of the ``usepackage'' statement for
% submitting the draft version of the paper for review.  This will set
% the note in the first column to ``Under review.  Do not distribute.''
\usepackage{icml2017} 

% Employ this version of the ``usepackage'' statement after the paper has
% been accepted, when creating the final version.  This will set the
% note in the first column to ``Proceedings of the...''
%\usepackage[accepted]{icml2017}
\usepackage{fullpage}
\usepackage{amsthm}
\usepackage{amssymb}
\newtheorem{theorem}{Theorem}[section]
\newtheorem{lemma}[theorem]{Lemma}
\newtheorem{fact}[theorem]{Fact}
\newtheorem{proposition}[theorem]{Proposition}
\newtheorem{observation}[theorem]{Observation}
\newtheorem{corollary}[theorem]{Corollary}
\newtheorem{remark}[theorem]{Remark}
\newtheorem{conjecture}[theorem]{Conjecture}
\newtheorem{definition}[theorem]{Definition}
\newtheorem{claim}[theorem]{Claim}
\newtheorem{assumption}{Assumption}

\usepackage{amsmath}
\usepackage{thmtools, thm-restate}
\usepackage{multirow}
\usepackage{tabularx}
\usepackage{colortbl}
\usepackage{color}
\usepackage[small]{caption}
%\usepackage[ruled,vlined]{algorithm2e}


%%%%%%%%%%%%% Macros %%%%%%%%%%%%% 
\newcommand{\llabel}[1]{\label{#1}}
\newcommand{\heading}[1]{{\bf #1}}

\newcommand{\zo}{\{0,1\}}
\newcommand{\mzo}{\{-1,+1\}}
\newcommand{\F}{{\mathbb{F}}}
\newcommand{\N}{{\mathbb{N}}}
\newcommand{\Z}{{\mathbb{Z}}}
\newcommand{\R}{{\mathbb{R}}}
\newcommand{\C}{{\mathbb{C}}}
\newcommand{\eps}{\epsilon}
\newcommand{\beq}{\begin{eqnarray}}
\newcommand{\eeq}{\end{eqnarray}}
\newcommand{\tO}{\tilde{O}}
\newcommand{\bt}{\tilde{b}}
\newcommand{\vb}{{\bar b}}
\newcommand{\sign}{\text{sign}}
\newcommand{\T}{T}
\newcommand{\ip}[1]{\langle #1 \rangle}
\DeclareMathOperator{\Tr}{Tr}

\newcommand{\Vol}{\mathop\mathrm{Vol}\nolimits}
\newcommand{\Const}{\mathop\mathrm{Const}\nolimits}
\DeclareMathOperator*{\expt}{\mathbb{E}}
\newcommand{\E}[2]{{\mathbb{E}_{#1}\left[#2\right]}}
\newcommand{\EE}[2]{{\expt_{#1}{#2}}}
\newcommand{\EX}{{\mathbb E}}
\newcommand{\Sur}{\mathop\mathrm{Sur}\nolimits}
\newcommand{\polylog}{\mathop\mathrm{polylog}\nolimits}
\newcommand{\xor}{\oplus}
\newcommand{\conj}[1]{{\overline {#1}}} %% conjugate
\newcommand{\pd}[2]{\frac{\partial#1}{\partial#2}}
\newif\ifshort
\shorttrue

\def\showauthornotes{1}

%%%%%%%%%%%%%% Use for definitions
\newcommand{\defeq}{\stackrel{\textup{def}}{=}}

%%%%%%%%%%%%%% Probability stuff
\DeclareMathOperator*{\pr}{\bf Pr}
\DeclareMathOperator*{\av}{\bf E}
\DeclareMathOperator*{\var}{\bf Var}

%%%%%%%%%%%%%% Matrix stuff
\newcommand{\tr}[1]{\mathop{\mbox{$\mathrm{Tr}$}}\left({#1}\right)}
\newcommand{\diag}[1]{{\bf Diag}\left({#1}\right)}

%% Notation for integers, natural numbers, reals, fractions, sets, cardinalities
%%and so on
\newcommand{\nfrac}[2]{\nicefrac{#1}{#2}}
\def\abs#1{\left| #1 \right|}
\newcommand{\norm}[1]{\ensuremath{\left\lVert #1 \right\rVert}}

\newcommand{\floor}[1]{\left\lfloor\, {#1}\,\right\rfloor}
\newcommand{\ceil}[1]{\left\lceil\, {#1}\,\right\rceil}

\newcommand{\pair}[1]{\left\langle{#1}\right\rangle} %for inner product

\newcommand\B{\{0,1\}}      % boolean alphabet  use in math mode
\newcommand\bz{\mathbb Z}
\newcommand\nat{\mathbb N}
\newcommand\rea{\mathbb R}
\newcommand\com{\mathbb{C}}
\newcommand\plusminus{\{\pm 1\}}
\newcommand\Bs{\{0,1\}^*}   % B star use in math mode

\newcommand{\V}[1]{\mathbf{#1}} % Used to denote bold commands
                                % e.g. vectors, matrices
\DeclareRobustCommand{\fracp}[2]{{#1 \overwithdelims()#2}}
\DeclareRobustCommand{\fracb}[2]{{#1 \overwithdelims[]#2}}
\newcommand{\marginlabel}[1]%
{\mbox{}\marginpar{\it{\raggedleft\hspace{0pt}#1}}}
\newcommand\card[1]{\left| #1 \right|} %cardinality of set S; usage \card{S}
\newcommand\set[1]{\left\{#1\right\}} %usage \set{1,2,3,,}
%\renewcommand\complement{\ensuremath{\mathsf{c}}}
\newcommand{\poly}{\mathrm{poly}}
 %\newcommand{\polylog}{{\mathrm{\mbox{polylog}}}}
%\newcommand\poly{\mathrm{poly}}  %usage \poly(n)
\newcommand{\comp}[1]{\overline{#1}}
\newcommand{\smallpair}[1]{\langle{#1}\rangle}
\newcommand{\ol}[1]{\ensuremath{\overline{#1}}\xspace}
\DeclareMathOperator*{\argmin}{arg\,min}
\DeclareMathOperator*{\argmax}{arg\,max}

%%%%%%%%%%%%%% Mathcal shortcuts
\newcommand\calF{\mathcal{F}}
\newcommand\calS{\mathcal{S}}
\newcommand\calG{\mathcal{G}}
\newcommand\calH{\mathcal{H}}
\newcommand\calC{\mathcal{C}}
\newcommand\calD{\mathcal{D}}
\newcommand\calI{\mathcal{I}}
\newcommand\calV{\mathcal{V}}
\newcommand\calK{\mathcal{K}}
\newcommand\calX{\mathcal{X}}
\newcommand\calU{\mathcal{U}}
\newcommand\calE{\mathcal{E}}

%%%%%%%%%%%%%% Author notes %%%%%%%%%%%%%

\definecolor{Mygray}{gray}{0.8}

 \ifcsname ifcommentflag\endcsname\else
  \expandafter\let\csname ifcommentflag\expandafter\endcsname
                  \csname iffalse\endcsname
\fi

\ifnum\showauthornotes=1
\newcommand{\todo}[1]{\colorbox{Mygray}{\color{red}#1}}
\else
\newcommand{\todo}[1]{#1}
\fi

\ifnum\showauthornotes=1
\newcommand{\Authornote}[2]{{\small\color{red}{[#1: #2]}}}
\else
\newcommand{\Authornote}[2]{}
\fi

%%%%%%%%%%%%%% Logical operators
\newcommand\true{\mbox{\sc True}}
\newcommand\false{\mbox{\sc False}}
\def\scand{\mbox{\sc and}}
\def\scor{\mbox{\sc or}}
\def\scnot{\mbox{\sc not}}
\def\scyes{\mbox{\sc yes}}
\def\scno{\mbox{\sc no}}

%% Parantheses
\newcommand{\paren}[1]{\left({#1}\right)}
\newcommand{\sqparen}[1]{\left[{#1}\right]}
\newcommand{\curlyparen}[1]{\left\{{#1}\right\}}
\newcommand{\smallparen}[1]{({#1})}
\newcommand{\smallsqparen}[1]{[{#1}]}
\newcommand{\smallcurlyparen}[1]{\{{#1}\}}

%% short-hands for relational simbols

\newcommand{\from}{:}
% \newcommand\xor{\oplus}
\newcommand\bigxor{\bigoplus}
\newcommand{\logred}{\leq_{\log}}
\def\iff{\Leftrightarrow}
\def\implies{\Rightarrow}


\newcommand{\Anote}{\Authornote{A}}
\newcommand{\Qnote}{\Authornote{Q}}
\newcommand{\Rnote}{\Authornote{R}}
\newcommand{\Snote}{\Authornote{S}}

\allowdisplaybreaks[3]
% The \icmltitle you define below is probably too long as a header.
% Therefore, a short form for the running title is supplied here:
\icmltitlerunning{Convergence of Electron-Proton Dynamics}

\begin{document} 

\twocolumn[
\icmltitle{Convergence of Electron-Proton Dynamics in Deep Learning}

% It is OKAY to include author information, even for blind
% submissions: the style file will automatically remove it for you
% unless you've provided the [accepted] option to the icml2017
% package.

% list of affiliations. the first argument should be a (short)
% identifier you will use later to specify author affiliations
% Academic affiliations should list Department, University, City, Region, Country
% Industry affiliations should list Company, City, Region, Country

% you can specify symbols, otherwise they are numbered in order
% ideally, you should not use this facility. affiliations will be numbered
% in order of appearance and this is the preferred way.
\icmlsetsymbol{equal}{*}

\begin{icmlauthorlist}
\icmlauthor{Rina Panigrahy}{goo}
\icmlauthor{Ali Rahimi}{goo}
\icmlauthor{Sushant Sachdeva}{goo}
\icmlauthor{Qiuyi Zhang}{berk,goo}
\end{icmlauthorlist}

\icmlaffiliation{berk}{University of California Berkeley, Berkeley, California, USA}
\icmlaffiliation{goo}{Google Research, Mountain View, California, USA}

\icmlcorrespondingauthor{Qiuyi Zhang}{qiuyizhang@gmail.com}
\icmlcorrespondingauthor{Rina Panigrahy}{rinap@google.com}

% You may provide any keywords that you 
% find helpful for describing your paper; these are used to populate 
% the "keywords" metadata in the PDF but will not be shown in the document
\icmlkeywords{deep learning, theoretical machine learning, ICML}

\vskip 0.3in
]

% this must go after the closing bracket ] following \twocolumn[ ...

% This command actually creates the footnote in the first column
% listing the affiliations and the copyright notice.
% The command takes one argument, which is text to display at the start of the footnote.
% The \icmlEqualContribution command is standard text for equal contribution.
% Remove it (just {}) if you do not need this facility.

%\printAffiliationsAndNotice{}  % leave blank if no need to mention equal contribution
%\printAffiliationsAndNotice{\icmlEqualContribution} % otherwise use the standard text.
%\footnotetext{hi}

\begin{abstract} 
  We study the efficacy of learning neural networks with neural
  networks by the (stochastic) gradient descent method. While gradient
  descent enjoys empirical success in a variety of applications, there
  is a lack of theoretical guarantees that explains the practical
  utility of deep learning. We focus on two-layer neural networks with
  a linear activation on the output node. We show that under some mild
  assumptions and certain classes of activation functions, if the target function can also be represented by such a network then gradient descent does learn at least one hidden unit of the network -- the incoming edge weight vector for at least one hidden node coincides with that of the true network for the target function %we show that every upon convergence 
  %every hidden unit in our trained network coincides with some %hidden unit in the original network. 
  Further this convergence happens in
  $\poly(d,1/\epsilon)$, time and sample complexity.
\end{abstract} 

\section{Introduction}


Deep learning has resulted in major strides in machine learning applications including speech recognition, image classification, and ad-matching. The simple idea of using multiple layers of nodes with a non-linear activation function at each node allows one to express any function.  To learn a certain target function we just use (stochastic) gradient descent to minimize the loss; this approach has resulted in significantly lower error rates for several real world functions, such as those in the above applications. Naturally the question remains: how close are we to the optimal values of the network weight parameters? Are we stuck in some bad local minima? While there are several recent works \cite{ChoromanskaHMAL14, DauphinPGCGB14, Kawaguchi16a} that have tried to study the presence of local minima the picture is far from clear.

There has been some work on studying how well can neural networks learn some synthetic function classes (e.g. polynomials~\cite{valiant2014learning}, decision trees). 
In this work we study how well can neural networks learn neural networks with gradient descent?
% In this work we ask how well can deep learning learn certain types of synthetic functions. 
% Simple examples of synthetic functions are. 
% Here we will study this question for functions that can be represented by a deep network of a certain depth and width. 
%
 % Can we use neural networks to learn neural networks with gradient descent methods? 
 % gradient descent on neural networks learn a randomly initialized target network with the same shape.
%
 Specifically, if the target function is a neural network with randomly initialized weights, and we attempt to learn it using a network with the same architecture, then will gradient descent converge to the target function?

% If a target function is deep network with a certain shape and has randomly initialized edge weights, then will gradient descent converge to the target function?
% {\color{red}That is, if the function to be learnt is a neural network, and we try to learn it with a network of the same shape and randomly initialized edge weights, then will the gradient descent converge to the right function? }




Experimental simulations show that for short depths (say two), and for different widths, with random network weights, stochastic gradient descent of a hypothesis network with the same architecture converges to an $\ell_2$ error that is a small percentage of a random network, indicating that SGD can learn these shallow networks with random weights (see section~\ref{experiments}). 

% Our experimental simulations show that for different widths and heights functions represented by neural networks with random edge weights can be learnt by stochastic gradient descent.
 % Our experimental simulations show that for short depths (say 2) and with random edge weights the error rate drops to a small percentage. 
%To understand this better, 



\paragraph{Results and Contributions.} 
We theoretically investigate the question of convergence for networks of depth two.
 % with certain simiplifying assumptions.  
Our main conceptual contribution is that for depth $2$ networks where the top node is a sum node, the question of whether gradient descent converges to the desired target function is equivalent to the following question in electrodynamics: Given $n$ fixed charges in $\rea^d,$ and $n$ mobile chages,
% initialized at random positions 
with all the mobile charges moving under the influence of the 
attractive force between opposite-sign charges, and repulsive force between  same-sign charges,
% electrical force of attraction from the protons and repulsion from the remaining electrons. 
at equilibrium, are all the mobile charges matched up with all the fixed charges, upto a permutation? 

In the above, $n$ corresponds to the number of hidden units, $d$ is the number of inputs to the  network, the positions of each fixed charge corresponds to the input weight vector of
% a  edge weights incident on each of the
a hidden unit in the target network, and the initial positions of the mobile charges are the initial values of the edge weights for the hidden units at the beginning of the gradient descent in the learning network. The motion of the charges essentially tracks the change in the network during gradient descent. The force between a pair of charges is not given by the standard electrical force of $1/r^2$ (where $r$ is the distance between unit charges), but by a function determined by the activation and the input distribution. Thus the question of convergence in these simplified depth two networks can be resolved by studying the equivalent electrodynamics question with the corresponding force function.
%
% \Anote{If you're going to have an informal theorem statement, they should be understandable without reading the main text. That's because a reader like me will ignore your intro text and just look at the theorem statements to gauge your paper. If the theorems look interesting, then they'll read the paper. This theorem refers to f, which is undefined. it also doesn't explain at all in what way the electrons and protons correspond to f. You should either make this explicit in the informal theorem's statement, or you should just blend this informal theorem into the rest of the section.}
\begin{theorem}[informal statement of Theorem~\ref{EPDyn}]
Applying gradient descent for learning the output of a depth two network
with activation $\sigma$ in the hidden layer and a linear output
node, under squared loss, 
% two layer
% neural network 
% The gradient descent process for learning our neural network $f$ 
is equivalent to the motion of k charges in the presence of k fixed
charges where the force between any pair of charges is given by a
potential function that depends on $\sigma$ and the input distribution.
\end{theorem}
%

 % and that the training data comes from a standard multivariate Gaussian distribution.
Our main technical contribution is to prove the existence of an activation function such that the corresponding gradient descent dynamics under standard Gaussian inputs result in learning at least one of the hidden nodes in the target network. We then show that this allows us to learn the complete target network one node at a time. We leave open the problem of convergence for forces corresponding to more realistic activation functions. We assume the sample complexity is close to its infinite limit. 
%
% {\color{red} }
%We also derive the force function for several possible activation functions. }
% Further for a certain synthetic activation function, we prove that the electrodynamic force function results in convergence thus 
% implying the desired convergence for simplified depth 2 networks with those activation functions 

%
% \Anote{same problem as the other informal theorem. i don't know what a $\theta$ is. it has only been defined in the text. furthermore, it's far too informal. instead of "carefully" it should give this notion a name. again, better to just blend this content in the rest of the text without calling it an informal theorem.}
\begin{theorem}[informal statement of Theorem~\ref{almostHarmSGD}]
There is an activation function such that running gradient
  descent for minimizing the squared loss along with $\ell_2$
  regularization for standard Gaussian inputs, at convergence, 
% For some
%   carefully chosen activation function, with regularization, at
%   convergence of running the SGD algorithm, 
  we learn at least one of
  the hidden weights of the target neural network.
\end{theorem}


We  prove that the above result can be iterated to learn the entire network node-by-node using gradient descent (Theorem~\ref{nodeWise}).  Our algorithm learns a network with the same architecture and number of hidden nodes as the target network. This is in contrast to the improper learning setting of many proposed algorithms.

In the supplementary material, we show some weak results for more practical activations. For the sign activation, we show that for the loss with respect to a single node, the only local minima are at the hidden target nodes with high probability if the target network has a randomly picked top layer. For the polynomial activation, we derive a similar result under the assumption that the hidden nodes are orthonormal and show a polynomial convergence of an SGD algorithm to learn these weights.


\paragraph{Intuition and Techniques.}
%
Note that for the standard electric potential function given by $\Phi = 1/r$ where $r$ is the distance between the charges, it is known from Earnshaw's theorem that an electrodynamic system with some fixed charges and some moving electrons is at equilibrium only when the moving charges coincide with the fixed charges. Given our translation above between electrodynamic systems and depth 2 networks (Section~\ref{sec:epdyn}), this would imply learnability of depth 2 networks with gradient descent under $\ell_2$ loss, if the activation function corresponds to the electrostatic potential. However, there exists no activation function $\sigma$ corresponding to this $\Phi$. 
%

% \paragraph{Organization.} In section 2, we introduce our framework and assumptions, and derive the equivalence between gradient descent and electron-proton dynamics under a suitable potential. 
%  % These convergence results are proven to simply illustrate our ideas. 
% In Section~\ref{}, we construct a realizable almost $\lambda$-harmonic potential and prove finite convergence guarantees. In section 6, experimental results confirm that depth-2 neural networks can be learned by gradient descent with common activation functions but seem to discredit that claim for higher depth networks. In section 5, we consider more realistic activation functions, such as the sign and polynomial function.

The proof of Earnshaw's theorem is based on the fact that the electrostatic potential is harmonic, \emph{i.e}, its Laplacian (trace of its Hessian) is identically zero. This ensures that at every critical point, there is direction of potential reduction (unless the hessian is identically zero). We generalize these ideas to potential functions that are eigenfunctions of the Laplacians, $\lambda$-harmonic potentials (Section~\ref{sec:earnshaw}). However, there potentials are unbounded. Subsequently, we construct a non-explicit activation function such that the corresponding potential is bounded and is almost $\lambda$-harmonic, \emph{i.e.}, it is $\lambda$-harmonic outside a small sphere (Section~\ref{sec:almost-harmonic}). For this activation function, we show at a stable critical point, we must learn at least one of the hidden nodes. Gradient descent (possibly with some noise, as in~\cite{GeHJY15}) is believed to converge to stable critical points. However, for simplicity, we use second order information to escape saddle points.

%\Snote{We use recent results on gradient descent~\cite{Rongetal} to show that gradient descent (with perturbations) converges close a stable critical point.}

{\color{red}The use of second-order methods is not limiting since noisy gradient descent algorithms can descent along negative curvature directions. Therefore, stochastic gradient descent should also converge to $M_{G,\epsilon}$ \cite{GeHJY15}, although we lack some regularity conditions. Furthermore, a more controlled perturbed gradient descent \cite{jin2017escape} can be applied in our setting to reach $M_{G,\epsilon}$ but requires more in-depth analysis.}


% Our main tool for analysis is to derive second-order information about our dynamics by using the Laplacian of the Hessian (or a submatrix of the Hessian) of our loss function. 
% Together with some generalization error bounds and discrete optimization results, we can finally translate these convergence results into finite time convergence rates. 


% We derive some sufficient conditions that characterize potentials
%  that arise from activation functions $\sigma$. 

% The different $\sigma$ that we study, their corresponding potentials, and their convergence results are summarized in Table ~\ref{table1}.
% First we show how to construct an activation function, such that the corresponding potential satisfies a nice smooth property which we call almost $\lambda$-harmonic, 
% for which we show that at convergence of the gradient descent, at least some $\theta_i$ coincides with some $w_j$.
%

%
%
% \begin{table}[tb]
% \caption{Activation, Potentials, and Convergence Results Summary}
% \label{table1}
% \noindent
% \vskip 0.1in
% \begin{center}
% \begin{small}
% \begin{sc}
% \begin{tabular}{
%   |p{\dimexpr.3\linewidth-2\tabcolsep-1.3333\arrayrulewidth}% column 1
%   |p{\dimexpr.37\linewidth-2\tabcolsep-1.3333\arrayrulewidth}% column 2
%   |p{\dimexpr.33\linewidth-2\tabcolsep-1.3333\arrayrulewidth}|% column 3
%   }
%    \hline 
%         Name of Activation&  Potential  ($\Phi(\theta,w)$)    & Convergence? \\ \hline 
%         Sign & $1 - \frac{2}{\pi}\cos^{-1}(\theta^Tw)$       & Yes d = 2, \ref{signCon} \ref{SignConv}\\ 
%         Polynomial  & $(\theta^Tw)^m$       & Yes, $\poly(d,\frac{1}{\epsilon}) \ref{PolyConv}$  \\        
%         Almost   $\lambda$-harmonic  & Poly($\theta^Tw$), \ref{AlmostHarmonic}  & Yes, $\poly(d,\frac{1}{\epsilon})$ \\
%          % Bessel    &  $e^{-\|\theta-w\|_1}$        & Yes for $d=1$  \\   
%         \hline
% \end{tabular}
% \end{sc}
% \end{small}
% \end{center}
% \vskip -0.1in
% \end{table} 
% %


We acknowledge that there is still a large gap between our developed theory and practice. However, our work can offer theoretical explanations and guidelines for the design of better activation functions or gradient-based training algorithms. For example, better accuracy and training speed were reported when using the newly discovered exponential linear unit (ELU) activation function in \cite{ClevertUH15, ShahKSS16}. We hope for more theory-backed answers to these and many other questions in deep learning.



%%% Local Variables:
%%% mode: latex
%%% TeX-master: "icmlpaper2017.tex"
%%% End:

\section{Deep Learning, Potentials, and Electron-Proton Dynamics}
\label{sec:epdyn}
\paragraph{Preliminaries.}
We will work in the space  $\mathcal{M}= \R^d.$ 
We denote the gradient and Hessian as $\nabla_{\R^d} f$ and  $\nabla_{\R^d}^2 f$ respectively.
The Laplacian is defined as
$\Delta_{\R^d} f = \Tr(\nabla_{\R^d}^2 f)$. 
%
If $f$ is multivariate with variable $x_i$, then let $f_{x_i}$ be a
restriction of $f$ onto the variable $x_i$ with all other variables
fixed. Let $\nabla_{x_i}f, \Delta_{x_i}f$ to be the gradient and
Laplacian, respectively, of $f_{x_i}$ with respect to $x_i$. Lastly,
we say $x$ is a critical point of $f$ if $\nabla f$ does not exist or
$\nabla f = 0$. 

We focus on learning depth two networks with a linear activation on
the output node. If the network takes inputs $x \in \R^d$ (say from
some distribution $\mathcal{D}$), then the network output, denoted
$f(x)$ is a sum over $k = \poly(d)$ hidden units with weight vectors
$w_{i} \in \R^d,$ activation $\sigma(x,w):\R^d \times \R^d\to \R,$ and
output weights $b_i \in \R.$ Thus, we can write
$f(x) = \sum_{i=1}^k b_i\sigma(x,w_i)$. We denote this concept class
$\mathcal{C}_{\sigma,k}.$ Our hypothesis concept class is also
$\mathcal{C}_{\sigma,k}.$ 

% of our learning
% procedure is the output of a two-layer neural network with linear
% output activation and has the form $f(x) = \sum_{i=1}^k
% b_i\sigma(x,w_i)$. 


%  where $\sigma(x,w):\R^d \times \R^d\to \R$
% is the activation function that takes in a hidden weight vector
% $w \in \R^d$ and the input vector $x \in \R^d$.  

% given by the form
% $f(x) = \sum_{i=1}^k b_i\sigma(x,w_i)$. 

 
% The target concept class $\mathcal{C}_{\sigma,k}$ of our learning
% procedure is the output of a two-layer neural network with linear
% output activation and has the form $f(x) = \sum_{i=1}^k
% b_i\sigma(x,w_i)$. 

Let $\boldsymbol{a} = (a_1,...,a_k)$ and $\boldsymbol{\theta} =
(\theta_1,...,\theta_k)$; similarly for $\boldsymbol{b},
\boldsymbol{w}$ and our guess is $\hat{f}(x) = \sum_{i=1}^k a_i
\sigma(x, \theta_i)$. 
We define $\Phi,$ the {\bf potential function} {\it corresponding to the activation
  } $\sigma,$ as
\[\Phi(\theta, w) = \expt_{X\sim D}[ \sigma(X,\theta) \sigma(X,w)].\]
%
We work directly with the true squared loss error
$L(a,\theta)= \expt_{x \sim \mathcal{D}}[(f - \hat{f})^2]$. To
simplify $L$, we re-parametrize $a$ by $-a$ and expand.
%
%\begin{equation}\label{errEmp}
%\widehat{L}(\boldsymbol{a,\theta})  = \frac{1}{n}\sum_{j=1}^n \left(\sum_{i=1}^k %b_i\sigma(x_j,w_i) - \sum_{i=1}^k a_i \sigma(x_j,\theta_i)\right)^2
%\end{equation}
%
\begin{align}
 L(\boldsymbol{a,\theta})  & = \expt_{X\sim D}\left[ \left(
  \sum_{i=1}^k a_i \sigma(X,\theta_i) + \sum_{i=1}^k
  b_i\sigma(X,w_i)\right)^2\right] \nonumber \\
%
& = \sum_{i=1}^k \sum_{j=1}^k a_i a_j \Phi(\theta_i,\theta_j)+ 2 a_ib_j \Phi(\theta_i,w_j)+ b_i b_j \Phi(w_i,w_j),
 \label{errLoss}
\end{align}
%
Given $\mathcal{D},$ the activation function $\sigma$, and the loss
$L$, we attempt to show that we can use some variant of gradient
descent to learn, with high probability, an $\epsilon$-approximation
of $w_j$ for some (or all) $j$. 
Note that our loss is jointly convex, though it is quadratic in
$\boldsymbol{a}$. 








% \subsection{Activation-Potential Correspondence}
In this paper, we restrict our attention to translationally invariant activations
and potentials.
%
% We restrict our attention to potential (and activation) functions with some natural symmetry, so they are either {\it translationally or rotationally invariant}.
Specifically, we may write $\Phi= h(\theta-w)$ for some function $h(x).$ Furthermore, a translationally invariant function $\Phi(r)$ is {\it radial} if it is a function of $r = \|x - y \|$.
%   in the first case, with $\theta, w \in \R^d$. In the second case, $\Phi = h(\theta^Tw)$ and we enforce $\theta, w \in S^{d-1}$. Such potentials are called {\bf symmetric}. Our results focus on rotationally invariant potentials, as they correspond to classical neural networks.

{\bf Remark}: Translationally symmetric potentials satisfy
$\Phi(\theta,\theta)$ {\it is a positive constant}. We normalize
$\Phi(\theta,\theta) = 1$ for the rest of the paper.
  
We assume that our input distribution
$\mathcal{D} = \mathcal{N}(0, {\bf I_{d\times d}})$ is fixed as the
standard Gaussian in $\R^d$. This assumption is not critical and a
simpler distribution might lead to better bounds. However, for
arbitrary distributions, there are hardness results for PAC-learning
halfspaces \cite{klivans2006cryptographic}.

We call a potential function {\bf realizable} if it corresponds to
some activation $\sigma$.  The following theorem characterizes
realizable translationally invariant potentials under standard
Gaussian inputs. Proofs and a similar characterization for rotationally invariant
potentials can be found in \supmaterial{Appendix~\ref{sec:realizable}}.
%  We briefly state some results about
% characterizations of realizable potentials for translationally and
% rotationally invariant potentials. Full proofs and calculations of
% activation-potential correspondences
% % , such as those claimed in Table \ref{table1}, 
% can be found in the supplementary material.
%
\begin{restatable}{theorem}{tranreal}
\label{thm:tranReal}
Let $\mathcal{M} = \R^d$ and $\Phi$ is square-integrable and
$\FT(\Phi)$ is integrable. Then, $\Phi$ is realizable under standard
Gaussian inputs if $\mathfrak{F}(\Phi)(\omega) \geq 0$ and the
corresponding activation is
$\sigma(x) =
(2\pi)^{d/4}e^{x^Tx/4}\mathfrak{F}^{-1}(\sqrt{\mathfrak{F}(\Phi)})(x),
$ where $\mathfrak{F}$ is the generalized Fourier transform in $\R^d.$
\end{restatable}
%

% \Anote{give examples of phi's and corresponding sigmas in in a paragraph.}
%
\subsection{Electron-Proton Dynamics}

%
% Our main observation is that the gradient descent dynamics of learning such to layer networks is equivalent to the dynamics of a set of proton-electron charges under a certain electrical attraction force function. 
% {\color{red} Assume for now that the coefficients $b_i$ and $a_i$ are $1$. 
% Thus we are only perform gradient descent over $\theta_i$ to minimize the expected square loss of $f-\widetilde{f}$.
%
% The charges reside in $R^d$. The protons are fixed at locations
% $w_1,..,w_k$. The electrons are at positions $\theta_1,..,\theta_k$
% and can move: 

By interpreting the pairwise potentials as electrostatic attraction
potentials, we notice that our dynamics is similar to electron-proton
type dynamics under potential $\Phi$, where $w_i$ are fixed point
charges in $\R^d$ and $\theta_i$ are moving point charges in $\R^d$
that are trying to find $w_i$. The total force on each charge is the
sum of the pairwise forces, determined by the gradient of $\Phi.$ We
note that standard dynamics interprets the force between particles as
an acceleration vector. In gradient descent, it is interpreted as a
velocity vector. 
%
\begin{definition}\label{EPDef}
  Given a potential $\Phi$ and particle locations
  $\theta_1,...,\theta_k \in \rea^d$ along with their respective
  charges $a_1,...,a_k \in \R$. We define {\bf Electron-Proton
    Dynamics} under $\Phi$ with some subset $S \subseteq [k]$ of fixed
  particles to be the solution $(\theta_1(t),...,\theta_k(t))$ to the
  following system of differential equations: For each pair
  $(\theta_i,\theta_j)$, there is a force from $\theta_j$ exerted on
  $\theta_i$ that is given by
  ${\bf F}_{i}(\theta_j) = a_ia_j\nabla_{\theta_i}
  \Phi(\theta_i,\theta_j)$ and
\[-\frac{d\theta_i}{dt} =  \sum_{ j \neq i} {\bf F}_{i}(\theta_j)\]
for all $i \not \in S$, with $\theta_i(0) = \theta_i$. For $i \in S$, $\theta_i(t) =  \theta_i$.
\end{definition}
%
For the following theorem, we assume that $\boldsymbol{\theta}$ is fixed.
%  ${\color{red} Although our loss is jointly
% non-convex, it is quadratic in Therefore we can
% restrict our attention to convergence in $\boldsymbol{\theta}$ to
% $\boldsymbol{w}$, since we have convergence guarantees to
% $\boldsymbol{a^*(\theta)}$, the optimal set of output weights of a
% given. }
\begin{restatable}{theorem}{epdyn}
\label{EPDyn}
Let $\Phi$ be a symmetric potential and $L$ be as in \eqref{errLoss}. Running continuous gradient descent on $\frac{1}{2} L$ with respect to $\theta$, initialized at
$(\theta_1,...,\theta_k)$ produces the same dynamics as
Electron-Proton Dynamics under $2\Phi$ with fixed particles at
$w_1,...,w_k$ with respective charges $b_1,..,b_k$ and moving
particles at $\theta_1,...,\theta_k$ with respective charges
$a_1,...,a_k$.
\end{restatable} 


%

%%% Local Variables:
%%% mode: latex
%%% TeX-master: "icmlpaper2017.tex"
%%% End:
\section{Main Results} 
%
When running gradient descent on non-convex loss functions, we often
can and do get stuck at a local minima. In this section, we use
second-order information to deduce that the local
minima for these certain classes of potentials occur only when certain convergence statements hold. In these section, these potentials are often {\it non-smooth, unbounded and un-realizable}. However, in the later sections, we apply insights developed here to derive similar convergence results for approximations of these potentials.
%
\subsection{Earnshaw's Theorem}
%
Earnshaw's theorem in electrodynamics shows that there is no stable
local minima for electron-proton dynamics. This hinges on the property
that the electric potential
$\Phi(\theta,w) = \|\theta-w\|^{2-d}, d \neq 2$ is harmonic, with
$d = 3$ in natural setting. If $d = 2$, we instead have
$\Phi(\theta, w) = - \ln(\|\theta - w\|)$. First, we notice that this
is a symmetric loss, and our usual loss in \eqref{errLoss} has
constant terms that can be dropped to further simplify.
%
\begin{equation}\label{errSimp}
\overline{L}(a,\theta) =  2\sum_{i=1}^k\sum_{i < j} a_ia_j\Phi(\theta_i,\theta_j) + 2\sum_{i=1}^k\sum_{j=1}^ka_ib_j \Phi(\theta_i,w_j)
\end{equation} 
%
\begin{definition}
$\Phi(\theta,w)$ is a {\bf harmonic} potential on $\Omega$ if $\Delta_\theta \Phi(\theta,w) = 0$ for all $\theta \in \Omega$, except possibly at $\theta = w$.
\end{definition}

\begin{definition}
  Let $\Omega \subseteq \R^d$ and consider a function
  $f:\Omega \to \R$. A critical point $x^* \in \Omega$ is a {\bf local
    minimum} if there exists $\epsilon > 0$ such that
  $f(x^*+v) \geq f(x^*)$ for all $\|v\|\leq \epsilon$. It is a {\bf
    strict local minimum} if the inequality is strict for all
  $\norm{v} \le \epsilon.$
\end{definition} 
%
\begin{fact}
  % Let $\Omega \subseteq \R^d$ and consider a function
  % $f:\Omega \to \R$.  
  Let $x^*$ be a critical point of a function $f : \Omega \to \R$ such
  that $f$ is twice differentiable at $x^*.$ Then, if $x^*$ is a local
  minimum then $\lambda_{min}(\nabla^2 f(x^*)) \geq 0.$ Moreover, if
  $\lambda_{min}(\nabla^2 f(x^*)) > 0,$ then $x^*$ is a strict local minimum.
\end{fact}
%
\begin{restatable}{theorem}{earnshaw}
[Earnshaw's Theorem. See~\cite{arnold1985mathematical}]
\label{Earnshaw} 
Let $\mathcal{M} = \R^d$ and let $\Phi$ be harmonic and $L$
be as in $\eqref{errSimp}$. Then, $L$ admits no
differentiable strict local minima.
\end{restatable}
%


\if{1}
\begin{corollary}
Let $\mathcal{M} = \R^d$, $\Phi(\theta,w) = \|\theta-w\|^{2-d}$, and $L$ be as in \eqref{errSimp}. If $\theta_i$ are all distinct and $(\boldsymbol{a,\theta})$ is a strict local minima, then $\theta$ has reached the global minima, i.e. $\theta_i = w_{\pi(i)}$ for some permutation $\pi$. 
\end{corollary}
\fi

Intuitively, the trace of the Hessian being 0 implies every
differentiable critical point has a direction of negative curvative
(unless the Hessian is the zero matrix altogether, in which case some
complex analysis still finds a direction of negative change
\cite{arnold1985mathematical}). 

Such a convergence result is desired but we run into two main problems. First, the singularity of the natural harmonic potentials at $0$ disqualifies them from being a realizable potential. Furthermore, harmonic potentials are not robust to approximation and statistical error as the convergence guarantees hinge on $\Delta_\theta\Phi$ being exactly 0. 

Second, the lack of strict local minima {\it does not} imply
convergence to the global minima under gradient descent. Notice that
harmonic potentials can admit local minima, as the Hessian matrix can
be the zero matrix, and so gradient descent could converge to these
local minima. However, if our loss function admits no local minima
(other than the global minima), then we can guarantee that gradient
descent with small stepsize converges to the global minima.

\subsection{$\lambda$-Harmonic Potentials}

The first alternative to harmonic potentials is the natural
consideration of strictly subharmonic potentials, which have a
positive Laplacian value almost everywhere. Subharmonic potentials are
also difficult to realize; however their convergence properties are
more robust than harmonic potentials and are discussed in supplementary material. 

%
In order to relate our loss function with its Laplacian, we consider potentials that are non-negative eigenfunctions of the Laplacian operator. Since the zero eigenvalue case simply gives rise to harmonic potentials, we restrict our attention to positive eigenfunctions.
%
\begin{definition}
A potential $\Phi$ is {\bf$\lambda$-harmonic} on $\Omega$ if there exists $\lambda > 0$ such that for every $\theta \in \Omega$, $\Delta_\theta \Phi(\theta, w) = \lambda \Phi(\theta,w) $, except possibly at $\theta = w$.
\end{definition}

We note that there are realizable versions of these potentials; for
example $\Phi(a,b) = e^{-\|a-b\|_1}$ in $\R^1$ as in Table~\ref{table1}. In Section 4, we construct a realizable approximation of these $\lambda$-harmonic potentials on $S^{d-1}$ that retains good convergence behavior.
%
\begin{restatable}{theorem}{eigStrict}
\label{EigStrict}
Let $\Phi$ be $\lambda$-harmonic and $L$ be as in \eqref{errLoss}. Then,
$L$ admits no local minima $\boldsymbol{(a,\theta)}$, except when
$L(\boldsymbol{a,\theta}) = L(0,\boldsymbol{\theta})$ or $\theta_i = w_j$ for some $i,j$. 
\end{restatable}
\begin{proof}
  Let $(\boldsymbol{a,\theta})$ be a critical point of $L.$ On the
  contrary, we assume that $\theta_i \neq w_j$ for all $i,j.$ WLOG, we
  can partition $[k]$ into $S_1,...,S_r$ such that for all $u \in S_i,
  v \in S_j$, we have $\theta_{u} = \theta_v$ iff $i=j$. 

Let $S_1 = \{ \theta_1, \ldots, \theta_{l}\}.$ 
% WLOG, let $\theta_1 =
% \theta_2 =... = \theta_l$ be non-distinct (with $l$ maximal) and
% $\theta_1 \neq w_j$ for any $j$. 
We consider changing all
$\theta_1, \ldots, \theta_{l}$ by the same $v$ and define 
%
\[H({\bf a}, v) = L({\bf a},\theta_1+v,...,\theta_l+v, \theta_{l+1}
\ldots, \theta_k).\]

The optimality conditions on ${\bf a}$ are 
\begin{align*}
0 = \pd{L}{a_i} = 2a_i  + 2\sum_{j\neq i} a_j \Phi(\theta_i,\theta_j)
  + 2\sum_{j=1}^k b_j \Phi(\theta_i,w_j).
\end{align*}
%
Next, by $\lambda$-harmonic definition, we may differentiate as $\theta_i \neq w_j$ and compute the Laplacian as 
\begin{align*}
\Delta_v H & = \lambda\sum_{i=1}^l a_i \left(2\sum_{j=1}^k b_j
  \Phi(\theta_i, w_j) + 2\sum_{j=l+1}^k a_j
  \Phi(\theta_i, \theta_j)\right) \\
& = \lambda\sum_{i=1}^l a_i \left(-2a_i\Phi(\theta_i, \theta_i) - 2
  \sum_{j = 1, j\neq i}^l  a_j \Phi(\theta_i,\theta_j)\right) \\
%
% & = -2\lambda\left(\sum_{i=1}^l a_i^2 \Phi(\theta_i, \theta_i)
%   +\sum_{i \neq j}^l a_i a_j \Phi(\theta_i, \theta_j)\right) \\
%
& = -2\lambda\left(\sum_{i=1}^l a_i^2
  +\sum_{i \neq j}^l a_i a_j \right) = -2 \lambda\left(\sum_{i=1}^l a_i\right)^2
\end{align*} 
%
If $\sum_{i=1}^l a_i \neq 0$, then we conclude that the Laplacian is
strictly negative, so we are not at a local minimum. 

Similarly, we can conclude that for each $S_i,$ 
% Therefore, if $\theta_i \neq w_j$ for all $i,j$, then we may partition $[k]$ into $S_1,...,S_r$ such that each $S_i$, $\theta_{u} = \theta_v$
% for $u, v \in S_i$ and
 $\sum_{u \in S_i} a_u = 0$. In this case, since $\sum_{i=1}^k a_i \sigma(\theta_i,x) = 0$, $L(\boldsymbol{a,\theta}) = L(0,\boldsymbol{\theta})$. 
\end{proof} 
%
\begin{observation}
We can initialize $\boldsymbol{a,\theta}$ such that $L(\boldsymbol{a,\theta}) < L(0,\boldsymbol{\theta})$ and continuous gradient descent will enforce the inequality as the $L$ is non-increasing. Initialization details are expounded in Observation~\ref{initialize}.
\end{observation}
%

%%% Local Variables:
%%% mode: latex
%%% TeX-master: "icmlpaper2017.tex"
%%% End:

\section{Runtime Bounds with Stochastic Gradients}
\Anote{this section should start with a definition, then have a main theorem that provides, as you presage, ``runtime bounds with stochastic gradient''. as it stands, i can't find such a theorem. i see lots of little claims though. please provide a main theorem at the top, and rename everything else to ``lemma''.}

The gradient descent convergence results in the previous sections lay
the foundation for reasoning about the convergence of stochastic
gradient descent (SGD). To derive finite runtime bounds, we need to
address three technical details: 1) realizable potentials are only
approximately $\lambda$-harmonic, 2) the variance of the stochastic
gradient should be small and 3) SGD should escape local minima in a
bounded number of iterations. The second and third details are analyzed
with standard techniques in optimization and statistics; the former
requires generalization bounds for neural networks and the latter
requires a lower bound on the negative curvature. Together, we will
show that running SGD after $\poly(d,1/\epsilon)$ iterations on an
almost $\lambda$-harmonic potential will allow us to learn at least
one of the hidden weights $w_j$ up to $\epsilon$ accuracy.

%
We first state a lemma concerning the construction of an almost
$\lambda$-harmonic function on $S^{d-1}.$ The construction is given in
Section~\ref{App:EigenFunc} and is based on truncating the taylor
series expansion of a solution to the $\lambda$-harmonic differential
equation. This results in a bounded degree polynomial which
(approximately) satisfies the guarantees of $\lambda$-harmonic
potentials when $\theta$ is far from $w_j$. 
%
\begin{definition}
$\Phi(\theta, w)$ is an $m$-almost $\lambda$-harmonic on $S^{d-1}$ if for
all $\theta, w\in S^{d-1}$, $|\Delta_{\theta}\Phi(\theta, w) - \lambda \Phi (\theta,w)| \leq \poly(m)|\theta^Tw|^m$ 
\end{definition}
%
%
\begin{restatable}{lemma}{almostHarmonic}\label{AlmostHarmonic}
For any $m$ and $\lambda = O(\poly(m))$,  we can construct a realizable odd $m$-almost $\lambda$-harmonic potential, $\Phi_m$, that corresponds to a odd polynomial $\sigma$ of degree $m$. Furthermore, $|\Phi_m(\theta^Tw)|\leq |\theta^Tw|$ for all $\theta, w \in S^{d-1}$ and all first, second, and third partial derivatives of $\Phi_m$ are bounded by $\poly(m)$.
\end{restatable} 
%
To make sure that $\|a\|$ remains controlled throughout the optimization process, we add a quadratic regularization term to $L$ and instead run SGD on $G = L + \|a\|^2$. Then, we show that if $(\boldsymbol{a,\theta})$ is in $\mathcal{M}_{G, \epsilon}$ for $\epsilon$ small, then $\theta_i$ is close to $w_j$ for some $j$. Finally, we show how to initialize $(\boldsymbol{a^{(0)},\theta^{(0)}})$ and run SGD to converge to $\mathcal{M}_{G,\epsilon}$, proving our main theorem.
%
\begin{theorem}\label{eigSGD}
  Let $\mathcal{M} = S^{d-1}$ and $b_1,...,b_k$ be bounded by $\poly(d)$ and $w_1,..,w_k$ are randomly chosen from $\mathcal{M}$.  

  With high probability, for all $\epsilon \in (0,1),$ there exists $m, \delta, \alpha$ (that depends on $d$) such that for the $m$-almost $1$-harmonic potential $\Phi_m$ the
  following holds: we can chose an initial point $(\boldsymbol{a^{(0)}, \theta^{(0)}})$ so that if after running SGD (Algorithm \ref{SGD}) on the regularized objective
  $G(\boldsymbol{a,\theta})$ such that either 1) $G(\boldsymbol{a,\theta}) \leq \epsilon$ or 2) there exists an $i, j$ such that $|w_j^T\theta_i| > 1- \epsilon$ in $poly(d,1/\epsilon)$ iteration complexity and $d^{\poly(d, 1/\epsilon)}$ sample complexity.
\end{theorem}
%
%
%
Our goal is to analyze running SGD on $G$ (as in \eqref{errLoss}) with a $m$-almost $\lambda$-harmonic potential. By reusing techniques in \cite{GeHJY15}, we show that SGD (with noise) can avoid all points that are have a negative curvature of at least $\epsilon$ in $\poly(1/\epsilon)$ iterations, resulting in the following Lemma. Intuitively, this means that SGD will converge to points with small gradient and small negative curvature, converging to a point in $\mathcal{M}_{G, \epsilon}$, where 
%
\[\mathcal{M}_{G, \epsilon} = \left\{x\in \mathcal{M} \Big| \|\nabla G(x)\|
  \leq \epsilon \text{ and } \lambda_{min}(\nabla^2 G(x)) \geq
  -\epsilon\right\}\]
%
We note that the more straightforward algorithm of computing the negative descent direction of the Hessian is also a tractable algorithm to converge to $\mathcal{M}_{G,\epsilon}$. The full description and proof of our SGD algorithm is given in the supplementary material. 

%
\begin{restatable}{lemma}{strongConverge2}\label{strongConverge2}  
We can construct a $m$-almost $\lambda$-harmonic potential $\Phi_m$ for which
we can choose stepsize $\eta = 1/\poly(d,1/\epsilon,\log(1/\zeta))$, such that with probability at least $1-\zeta$, running Algorithm \ref{SGD} on $\widehat{G}$ initialized at $x_0$ with stepsize $\eta$  returns a point $x_\eta$ that is in $\mathcal{M}_{G, \epsilon}$ after at most $T = \poly(d,1/\epsilon, \log(1/\zeta))$ iterations. 
 Furthermore, $G(a,x_T) \leq G(a, x_0)$ with high probability. 
\end{restatable}
%
\begin{restatable}{lemma}{eigConv}
\label{eigConv}
Let $\mathcal{M} = S^{d-1}$ and let $b_1,...,b_k$ be reals bounded by
$\poly(d)$. For all $\epsilon \in (0,1),$ we can pick
$m = O(\log(d)\log(1/\delta)/\epsilon)$ such that the following holds:

Let $L$ be as in \eqref{errLoss} with an $m$-almost 1-harmonic
potential $\Phi_m$ and let $G = L + \|a\|^2$.  If
$(\boldsymbol{a,\theta}) \in \mathcal{M}_{G, \delta^2 / (4d)}$ and
$G(\boldsymbol{a,\theta}) \leq G(\boldsymbol{0, 0})$, then for all
$i$, either $|a_i| < \delta,$ or there exists $j$ such that
$|w_j^T\theta_i| > 1- k^2\epsilon.$ Furthermore,
unless all $|a_i| <\delta$, there exists $i, j$ such that
$|w_j^T\theta_i| > 1-\epsilon$.
\end{restatable}
%
\begin{proof}
  The proof is similar to Theorem \ref{EigStrict}. Consider all
  $\theta_i$ such that $|a_i| > \delta$ and
  $|w_j^T\theta_i| \leq 1-\epsilon$ for all $j$. WLOG, let these be
  $\theta_1,...,\theta_l$, for some $l \leq k$. Assume for now that
  furthermore, $|\theta_i^T\theta_j| \leq 1-\epsilon$ for
  $i \leq l, j\geq l+1$.  Then, consider a correlated rotation on the
  sphere, which is a correlated translation in spherical
  coordinates. So, consider the loss function 
%
\[ H({\bf a, v}) = G({\bf a,\theta_1+v,...,\theta_l+v},\theta_{l+1},..\theta_k)\]

Let $(\boldsymbol{a,\theta}) \in \mathcal{M}_{G, \delta^2/(4d)}$. Then,
the optimality conditions on ${\bf a}$ are as usual:
%
\begin{align*}
   \abs{\pd{H}{a_i}} & = \lvert 2\sum_{j=1}^k b_j \Phi_m(\theta_i^T w_j) +
    2\sum_{j: j\neq i} a_j \Phi_m(\theta_i^T\theta_j) \\
& \qquad \qquad \qquad + (2\Phi_m(1) +
    2)a_i \rvert \\
& \leq \delta 
\end{align*}
%
Since $\Phi_m$ is $m$-almost 1-harmonic, we get,
%
\begin{align*}
  &  \left| \Delta_{{\bf v}} H -  2\sum_{i=1}^l\sum_{j=1}^k a_i b_j
    \Phi_m(\theta_i^Tw_j) \right. \\
& \qquad \qquad \qquad \left.- 2\sum_{i=1}^l\sum_{j= l+1}^k
  a_ia_j(-\Phi_m(\theta_i^T\theta_j)) \right| \\
  & \leq   2\poly(m)\left(\sum_{i=1}^l\sum_{j=1}^k a_i b_j  |\theta_i^Tw_j|^m+\sum_{i=1}^l\sum_{j= l+1}^k a_ia_j|\theta_i^T\theta_j|^m\right)
\end{align*}
%
Together, we get
%
\begin{align*}
& |\Delta_{{\bf v}}H + \sum_{i=1}^l (2\Phi_m(\theta_i^T\theta_i)+2)a_i^2 + 2\sum_{i \neq j}^l a_ia_j\Phi_m(\theta_i^T\theta_j)| \\
& \leq \sum_{i=1}^l\delta |a_i| +\sum_{i=1}^l 2|a_i|\poly(m)|\sum_{j=1}^k b_j
  (\theta_i^Tw_j)^m +  \sum_{j=l+1}^k  a_j(\theta_i^T\theta_j)^m|
\\
\end{align*}

Again, since these inner products are assumed to be less than
$1-\epsilon$ and and $\|a\| \leq \poly(d)$ by
$G(\boldsymbol{a,\theta}) \leq G(\boldsymbol{0,0})$ and Lemma
\ref{bounded}, we can choose
$m = O(\frac{1}{\epsilon}\log(d)\log(1/\delta))$ large enough such
that
%
\begin{align*}
 2\poly(m)\sum_{i=1}^l|\sum_{j=1}^k b_j (\theta_i^Tw_j)^m +
  \sum_{j=l+1}^k  a_j(\theta_i^T\theta_j)^m| \\
 \leq  2\poly(d,k,m)e^{-\epsilon m} \leq \delta/2
\end{align*}

Finally, consider the following expression
%
\begin{align*}
D & = \expt\left[\left( \sum_{i=1}^l a_i \sigma(\theta_i,X)\right)^2\right] \\
%
& =\sum_{i=1}^l a_i^2\Phi_m(\theta_i^T\theta_i) + \sum_{i \neq j}^l
  a_ia_j\Phi_m(\theta_i^T\theta_j)
\end{align*}
%
Since $D \geq 0$ and combined with the fact that $|a_i| > \delta$,
%
\begin{align*}
\Delta_{{\bf v}}H & \leq  \sum_{i=1}^l (-2a_i^2 +\delta |a_i|+(\delta/2)|a_i|) \\
& \leq   \sum_{i=1}^l  |a_i|(-2\delta + 3\delta/2) < -l\delta^2/4
\end{align*}

Since the Laplacian is a sum of $d$ eigenvalues, this implies that we can find $u \in S^{d-1}$ such that $-l\delta^2/(4d) > u^T\nabla^2_v H u$. Finally, notice that if $w = (\underbrace{u,...u}_{l {\textrm{ times}}},0,...,0)$, then $w^T\nabla^2_{\boldsymbol{\theta}} G w = u^T\nabla^2_v H u < -l\delta^2/4d$. Since $w$ has $\|w\|^2 = l$, $\lambda_{min}(\nabla^2 G) < -\delta^2/(4d)$, contradicting that it is in $\mathcal{M}_{G, \delta^2/(4d)}$. 

Finally, it must be the case that there exists $\theta_{j_1},...\theta_{j_l}$ such that
$|\theta_i^T\theta_{j_1}| > 1-\epsilon$ and
$|\theta_{j_{i}}^T\theta_{j_{i+1}}| > 1-\epsilon$, and
$|\theta_{j_l}^Tw_j| > 1-\epsilon$ for some $w_j$. Therefore,
$|\theta_i^Tw_j| > 1- k^2\epsilon$ and unless all $|a_i| <\delta$,  there exists $i, j$ such that $|w_j^T\theta_i| > 1-\epsilon$.
\end{proof}

\begin{lemma}\label{eigRes}
  Assume the conditions of Lemma~\ref{eigConv}. If
$G({\bf a, \boldsymbol{\theta}}) \leq G(\boldsymbol{0,0}) - 2\delta\sqrt{G(\boldsymbol{0,0})}$
  and $(\boldsymbol{a,\theta}) \in \mathcal{M}_{G,\delta^2/(4\poly(d))}$,
  then there exists some $i, j$ such that $|\theta_i^Tw_j| > 1- \epsilon$.
\end{lemma}
 
 \begin{proof}
   By Lemma \ref{eigConv}, if there does not exists $i, j$ such that
   $|w_j^T\theta_i| > 1-\epsilon$, then for all $\theta_i$, then
   $|a_i| < \delta/\poly(d)$ for all $i$. Now, for a integrable
   function $f(x)$, $\| f\|_X = \sqrt{\expt_X[f(X)^2]}$ is a
   norm. Therefore, if $f(x) = \sum_i b_i \sigma(w_i,x)$ be our true
   target function, we conclude that by triangle inequality
\begin{align*}
\sqrt{G(\boldsymbol{a,\theta})}  & \geq \norm{\sum_{i=1}^k a_i \sigma(\theta_i,x) - f(x)}_X \\
&\geq \|f(x)\|_X\ - \sum_{i=1}^k \|a_i\sigma(\theta_i,x) \|_X \\
& \geq
  \sqrt{G(\boldsymbol{0,0})} - \delta
\end{align*}
Squaring gives a contradiction, so we conclude that there must exist $i, j$ such that $\theta_i$ is in a $\epsilon$ neighborhood of $w_j$.
 \end{proof}
 
 \begin{lemma}\label{initialize}
Assume the conditions of Lemma~\ref{eigConv} and let $\theta_1$ be chosen uniformly from $\mathcal{M}$. If $G(\boldsymbol{0,0}) \geq \epsilon$, then we can initialize $\boldsymbol{a^{(0)},\theta^{(0)}}$ such that $G({\bf a^{(0)}, \boldsymbol{\theta^{(0)}}}) \leq G(\boldsymbol{0,0}) - 2\delta\sqrt{G(\boldsymbol{0,0})}$ with $\delta = 1/\poly(d,1/\epsilon)$ with probability at least $e^{-O(d \log d)}$.
 \end{lemma}
 
 \begin{proof}
  Consider choosing $\theta_1$ uniformly from $\mathcal{M}$ and then
  optimizing $a_1$. Given $\theta_1$, the loss decrease is:
%
\begin{align*}
   G(a_1,\theta_1) - G(0,0) & = \min_{a_1} 2a_1^2 +
  2\sum_{j=1}^k a_1 b_j\Phi(\theta_1,w_j) \\
 %
 & = -\frac{1}{2}\left(  \sum_{j=1}^k b_j
   \Phi(\theta,w_j)\right)^2 
\end{align*}

Next, we can rewrite our loss as
%
\[G(\boldsymbol{0,0}) =  \sum_{i=1}^k\sum_{j=1}^k b_i b_j \Phi_m(w_i, w_j) \]

Therefore, we can find $i$, such that
$|\sum_{j=1}^k b_j \Phi_m(w_i, w_j)| \geq
G(\boldsymbol{0,0})/(kb_i)$. Now, notice that all first partials of $\Phi_m$ are $\poly(m)$-bounded, so $\Phi_m$ is $\poly(d, m)$- Lipschitz. Since $b_i$ is $\poly(d)$-bounded and $\Phi_m$ is $\poly(d,m)$-Lipschitz, we see
that in a $\Omega(1/poly(d,m))$-neighborhood of $w_i$,
%
\[|\sum_{j=1}^k b_j \Phi_m(\theta, w_j)| \geq G(\boldsymbol{0,0})/poly(d) \]

The probability $\theta$ randomly chosen would lie in this
neighborhood $\mathcal{N}$ is $e^{-O(d \log d)}$ (as the surface area of $S^{d-1}$ grow exponentially in $d$). So by choosing $\theta_1$ randomly and setting $a_1$ optimally and all the other $a_i$'s to $0$ with probability at least $e^{-O(d \log d)}$, we can initialize $(\boldsymbol{a^{(0)},\theta^{(0)}})$ such that 
%
\[G({\bf a^{(0)}, \boldsymbol{\theta^{(0)}}}) \leq G(\boldsymbol{0,0}) - \frac{1}{\poly(d)}G(\boldsymbol{0,0})^2\]

Finally, since $G(\boldsymbol{0,0}) \geq \epsilon$, our claim holds with $\delta = 1/\poly(d,1/\epsilon)$
\end{proof}
%

\if{1}

The following Lemma gives an alternate guarantee on $G({\bf a, \boldsymbol{\theta}})$ provided the $w$ are chosen randomly. 

 

\begin{lemma}\label{bigVariance}
Let $\Phi_m$ be as in Lemma \ref{AlmostHarmonic} and $b_1,...,b_k$ are $\poly(d)$-bounded. Let $\theta$ be uniformly randomly chosen on $S^{d-1}$. Then,
 
 $\expt\left[\left(  \sum_{j=1}^k b_j \Phi_m(\theta,w_j)\right)^2\right]
 \geq e^{-O(d)}G(\boldsymbol{0,0})$
\end{lemma}
\Snote{Richard, I don't think the details here are sufficient to
  understand the proof. Please elaborate, and fill in the missing details.}
\begin{proof}

We can rewrite our loss as
%
\[G(\boldsymbol{0,0}) =  \sum_{i=1}^k\sum_{j=1}^k b_i b_j \Phi_m(w_i, w_j) \]

Therefore, we can find $i$, such that
$|\sum_{j=1}^k b_j \Phi_m(w_i, w_j)| \geq
G(\boldsymbol{0,0})/(kb_i)$. Now, notice that all first partials of $\Phi_m$ are $\poly(m)$-bounded, so $\Phi_m$ is $\poly(d, m)$- Lipschitz. Since $b_i$ is $\poly(d)$-bounded and $\Phi_m$ is $\poly(d,m)$-Lipschitz, we see
that in a $\Omega(1/poly(d,m))$-neighborhood of $w_i$, call it $\mathcal{N}$,
%
\[|\sum_{j=1}^k b_j \Phi_m(\theta, w_j)| \geq G(\boldsymbol{0,0})/poly(d) \]

Note that $P(\theta \in \mathcal{N}) = e^{-O(d)}$ since the surface area of the sphere is exponential in $d$. Thus, we have

\[\expt[(  \sum_{j=1}^k b_j \Phi_m(\theta,w_j))^2]
 \geq \prob(\theta \in \mathcal{N})\expt[(\sum_{j=1}^k b_j \Phi(\theta, w_j))^2 | \theta \in \mathcal{N}]\]
\[\geq e^{-O(d)} L(\boldsymbol{0,0})\]
 \Snote{How does the claim
  about the square follow?} 
\end{proof}




\begin{lemma}\label{largeVariance}
  Let $\Phi_m$ be as in Lemma \ref{AlmostHarmonic} \Snote{Always say
    Lemma/Theorem when referring to one. So this should be
    Lemma~\ref{AlmostHarmonic}} and let $\theta, w_1,...,w_k$ be
  independent and uniformly randomly chosen on $S^{d-1}$. Then, over
  $\theta, w_1,...,w_k$,
 
 $\expt\left[\left(  \sum_{j=1}^k b_j \Phi(\theta,w_j)\right)^2\right]
 \geq \norm{b}_2^2 \cdot \Omega(1/d^m).$
\end{lemma}
%
\begin{proof}
Note that $\Phi_m$ is a degree $m$ polynomial with non-negative coefficients and it only has odd-degree terms. Since it is odd, $E[\Phi_m(\theta^Tw_i)] = 0$ and the cross terms in the expansion of the square vanish:
%
\[ \expt\left[\left(  \sum_{j=1}^k b_j
     \Phi_m(\theta^Tw_j)\right)^2\right] = \sum_{j=1}^k b_j^2 \expt[\Phi_m(\theta^Tw_j)^2]\]

So, it suffices to show $\expt[\Phi_m(\theta,w_j)^2] \geq 1/m$.  Now, since $\Phi_m$ has non-negative coefficients,
\begin{align*}
& \expt[\Phi_m(\theta^Tw_j)^2] = \expt\left[\left(\sum_{i=1, i \,
                               odd}^m c_i
                               (\theta^Tw_j)^i\right)^2\right] \\
& \qquad = \sum_{i,j \,odd} c_ic_j E[(\theta^Tw_j)^{i+j}] \geq \sum_{i,j
  \,odd}c_i c_j \Omega(1/(i+j))  \\
& \qquad =\Omega(1/d^m) 
\end{align*}

Where we used the fact that
$\sum_{i,j} c_ic_j = (\sum_i c_i)^2 = \Phi_m(1)^2 = 1$ and
$E[(\theta^Tw_j)^{n}] \geq \Omega(1/d^{n})$, which holds by noticing that with high probability $\theta^T w_j = \Omega(1/\sqrt{d})$. \Snote{The last claim is not correct. $(\theta^Tw_j)^n$ is
  $O(1/\sqrt{n}^n$ with high probability.}
\end{proof}

\fi 
 
\begin{proof}[Proof of Theorem \ref{eigSGD}]
Lemma~\ref{strongConverge2} ensures that we are in $\mathcal{M}_{G,\delta^2/(4\poly(d))}$ in $\poly(d,1/\epsilon)$ iterations.
If $G(\boldsymbol{0,0}) \leq \epsilon$, then by Lemma~\ref{strongConverge2}, with high probability, $G(\boldsymbol{a,\theta}) \leq G(\boldsymbol{0,0}) \leq \epsilon$. Otherwise, by Lemma \ref{initialize},  we can initialize $\boldsymbol{a^{(0)},\theta^{(0)}}$ such that $G(\boldsymbol{a^{(0)},\theta^{(0)}}) \leq  G(0,0) - 2\delta \sqrt{G(0,0)}$ for $\delta = 1/\poly(d,1/\epsilon)$ after $e^{O(d\log d)}$ samples by applying standard probabilistic bounds. Finally, we conclude by Theorem \ref{eigRes}.
\end{proof}
%

\if{1}

\Rnote{From here begins the reset algorithm...perhaps better with node-wise algorithm...}

In the end, we can further combine Theorem \ref{eigSGD} with Algorithm \ref{GDReset} to derive a convergence result that all $\theta_i$ will be close to some $w_j$.
 
 \begin{algorithm}[tb]
 \caption{SGD Algorithm with Resets}
   \label{GDReset}
\begin{algorithmic}
  \STATE {\bfseries Input:}
  $(\boldsymbol{a,\theta}) = (a_1,...,a_k,\theta_1,...,\theta_k), a_i
  \in\R, \theta_i\in\mathcal{M}$;
  $T\in \N$; $\widehat{L}$; $\alpha\in \R$; $\delta \in \R$;
  $\gamma \in R$; $\nu \in \R$; $\epsilon \in \R$ \vspace{0.1in} 
  
   \REPEAT
  \STATE $(\boldsymbol{a},\boldsymbol{\theta}) = SGD \left(\widehat{L} + \|a\|^2, (\boldsymbol{a},\boldsymbol{\theta}),T, \alpha,\delta \right)$
  \STATE Find $S \subset [k]$ maximal with $|a_j| < \nu$ for $j \in S$
  \IF{$|S| > 0$} 
     \FOR{$j \in S$} 
    \REPEAT \STATE Sample $\theta_j$
  uniformly from $\mathcal{M}$
  \UNTIL{$\left (\frac{\partial
        \widehat{L}(\boldsymbol{a,\theta_j})}{\partial a_j} \right)^2
    \geq \gamma$}
    \ENDFOR
  \ELSE
  \STATE  {\bf return} $(\boldsymbol{a}, \boldsymbol{\theta}) $
  \ENDIF
  \UNTIL{$\widehat{L}(\boldsymbol{a,\theta}) \leq \epsilon/2$}
   \STATE {\bf return} $(\boldsymbol{a}, \boldsymbol{\theta}) $
   \end{algorithmic}
\end{algorithm}

\begin{corollary}
Assume the conditions of Thm~\ref{eigSGD}. For any $0 < \epsilon < 1$, with high probability, running Algorithm \ref{GDReset} on $L$ with suitable parameters $\delta, \gamma, \alpha$ converges in $T = \poly(d, 1/\epsilon)$ iterations to $\boldsymbol{a,\theta}$ such that either $L(\boldsymbol{a,\theta}) \leq \epsilon$ or each $\theta_i$ is within $k^2\epsilon$-neighborhood of some $w_j$
\end{corollary}

\begin{proof}
If the algorithm returns $\boldsymbol{a,\theta}$ such that $\widehat{L}(\boldsymbol{a,\theta}) \leq \epsilon/2$, then since $\widehat{L}$ is a stochastic oracle for $L$ with error at most $\epsilon/2$, we conclude $L(\boldsymbol{a,\theta}) \leq \epsilon$. 

Otherwise, by Theorem \ref{eigSGD}, 
\end{proof}

\fi









%%% Local Variables:
%%% mode: latex
%%% TeX-master: "icmlpaper2017.tex"
%%% End:
\section{Convergence of Almost Strictly Subharmonic Potentials}\label{App:Subharm}


%
%
\begin{definition}
$\Phi(\theta,w)$ is a {\bf strictly subharmonic} potential on $\Omega$ if it is differentiable and $\Delta_\theta \Phi(\theta,w) > 0$ for all $\theta \in \Omega$, except possibly at $\theta = w$.
\end{definition}

An example of such a potential is $\Phi(\theta, w) = \|\theta -w \|^{2-d-\epsilon}$ for any $\epsilon > 0$. Although this potential is unbounded at $\theta = w$ for most $d$, we remark that it is bounded when $d = 1$. Furthermore, the sign of the output weights $a_i, b_i$ matter in determining the sign of the Laplacian of our loss function. Therefore, we need to make suitable assumptions in this framework.

%

Under Assumption \ref{outputFixed}, we are working with an even
simpler loss function:
\begin{equation}\label{errFixed}
L(\theta) =  2\sum_{i=1}^k\sum_{i < j} \Phi(\theta_i,\theta_j) - 2\sum_{i=1}^k\sum_{j=1}^k\Phi(\theta_i,w_j)
\end{equation}
%
\Anote{i don't think you should show this theorem. you ahve more significant results, and this distracts from them. plus it doesn't refer to any realistic learning setup. it's an intermediate warmup illustration for yourself, not for the reader.}
\begin{theorem}\label{subStrict}
Let $\Phi$ be a symmetric strictly subharmonic potential on $\mathcal{M}$ with $\Phi(\theta,\theta) = \infty$. Let Assumption \ref{outputFixed} hold and let $L$ be as in \eqref{errFixed}. Then, $L$ admits no local minima, except when $\theta_i = w_j$ for some $i, j$.
\end{theorem}
%
\begin{proof}
First, let $\Phi$ be translationally invariant and $\mathcal{M} =
\R^d$. Let $\boldsymbol{\theta}$ be a critical point. Assume, for sake
of contradiction, that for all $i, j$, $\theta_i \neq w_j$. If
$\theta_i$ are not distinct, separating them shows that
we are not at a local minima since $\Phi(\theta_i,\theta_j) = \infty$
and finite elsewhere. 

The main technical detail is to remove interaction terms between
pairwise $\theta_i$ by considering a correlated movement, where each
$\theta_i$ are moved along the same direction $v$. In this case,
notice that our objective, as a function of $v$, is simply
\begin{align*}
& H(v) = L(\theta_1+ v, \theta_2 + v, ...,\theta_k + v) \\
& =  2\sum_{i=1}^k\sum_{i < j} \Phi(\theta_i+v,\theta_j+v) -
  2\sum_{i=1}^k\sum_{j=1}^k \Phi(\theta_i+v,w_j)
\end{align*}

Note that the first term is constant as a function of $v$, by
translational invariance. Therefore,
\[\nabla_{v}^2 H = -2\sum_{i=1}^k \sum_{j=1}^k \nabla^2\Phi(\theta_i, w_j)\]
By the subharmonic condition,
$\Tr(\nabla_{v}^2H) = -2\sum_{i=1}^k\sum_{j=1}^k
\Delta_{\theta_i}\Phi(\theta_i,w_j) < 0$.
Therefore, we conclude that $\theta$ is not a local minima of $H$ and
$L$.  We conclude that $\theta_i = w_j$ for some $i, j$.

The above technique generalizes to $\Phi$ being rotationally invariant
case by working in spherical coordinates and correlated translations
are simply rotations. Note that we can change to spherical coordinates
(without the radius parameter) and let
$\widetilde{\theta_1},...,\widetilde{\theta_{k}}$ be the standard
spherical representation of $\theta_1,...,\theta_k$.

We will consider a correlated translation in the spherical coordinate space, which are simply rotations on the sphere. Let $v$ be a vector in $\R^{d-1}$ and our objective is simply
\[ H(v) = L( \widetilde{\theta_1}+v,...,\widetilde{\theta_{k}} +v)\]

Then, we apply the same proof since $\Phi( \widetilde{\theta_i}+v, \widetilde{\theta_j}+v)$ is constant as a function of $v$ by rotationally invariance.
\end{proof}
%
\begin{corollary}
Assume the conditions of Theorem \ref{subStrict} and $\Phi(\theta,\theta) < \infty$. Then, $L$ admits no local minima, except at the global minima.
\end{corollary} 

\begin{proof}
From the same proof from theorem \ref{subStrict}, we conclude that there must exists $i, j$ such that $\theta_i = w_j$. Then, since $\Phi(\theta,\theta) < \infty$, notice that $\theta_i, w_j$ cancels each other out and by drop $\theta_i, w_j$ from the loss function, we have a new loss function $L$ with $k-1$ variables. Then, using induction, we see that $\theta_i = w_{\pi(i)}$ at the local minima for some permutation $\pi$.
\end{proof}


For concreteness, we will focus on a specific potential function with this property: the Gaussian kernel $\Phi(\theta, w) = \exp(-\|\theta - w\|^2/2)$. In $\R^d$, the Laplacian is $\Delta \Phi = ( \|\theta - w\|^2 -d ) \exp(-\|\theta - w\|^2/2)$, which becomes positive when $\|\theta - w \|^2 \geq d$. Thus, $\Phi$ is strictly subharmonic outside a ball of radius $\sqrt{d}$. This informally implies that $\theta_1$ converges to a $\sqrt{d}$-ball around some $w_j$. 





For concreteness, we will focus on a specific potential function with
this property: the Gaussian kernel $\Phi(\theta, w) = \exp(-c\|\theta
- w\|^2/2)$, which corresponds to a Gaussian activation. In $\R^d$, the Laplacian is $\Delta \Phi = ( c\|\theta - w\|^2 -d ) \exp(-c\|\theta - w\|^2/2)$, which becomes positive when
$\|\theta - w \|^2 \geq d/c$. Thus, $\Phi$ is strictly subharmonic
outside a ball of radius $\sqrt{d/c}$. Note that Gaussian potential
restricted to $S^{d-1}$ gives rise to the exponential activation
function, so we can show convergence similarly.  
%
\begin{theorem}\label{gaussStrict}
\label{GaussStrict}
Let $\mathcal{M} = \R^d$ and $\Phi(\theta,w) = e^{-c\|\theta-w\|^2/2}$ and Assumption \ref{outputFixed} holds. Let $L$ be as in \eqref{errFixed} and $\|\boldsymbol{w}\|\leq poly(d)$. 

If $c = O(d/\epsilon)$ and $(\boldsymbol{a,\theta}) \in \mathcal{M}_{e^{-\poly(d,1/\epsilon)}}$, then there exists $i, j$ such that $\| \theta_i - w_j \|^2 \leq \epsilon$.
\end{theorem}

\begin{proof}
Consider again a correlated movement, where each $\theta_i$ are moved along the same direction $v$. As before, this drops the pairwise $\theta_i$ terms. If for all $i, j$ $\| \theta_i - w_j \|^2 \leq \epsilon$, then we see that $\Delta_{\theta_i} \Phi = ( c\|\theta - w\|^2 -d ) \exp(-c\|\theta - w\|^2/2) > e^{-poly(d,1/\epsilon)}$. 
%
\[\Tr(\nabla^2 L) = -2\sum_{i=1}^k \sum_{j=1}^k \Delta_{\theta_i}\Phi(\theta_i, w_j) < -e^{-poly(d,1/\epsilon)}\]

Therefore, $\nabla^2 L$ must admit a strictly negative eigenvalue that
is less than $e^{-c_3 d}$, which implies our claim (we drop the
$\poly(d,k)$ terms).

******Add more rigor/sample complexity later if we want to keep this

To use Thm \ref{strongConverge}, we check the regularity conditions: note that we can choose $L, B, \rho$ to be $\poly(d)$ since the first, second and third partials of $\Phi$ are all bounded by $\poly(d)$. Now, by choosing $\epsilon = e^{-O(d)}$ for some $c$, we know that with high probability, running SGD in $T = e^{O(d)}$ iterations will output $\theta_1$ such that 1) $\|\nabla L (\theta_1)\| < e^{-O(d)}$ and 2) $\lambda_{min}(\nabla^2L(\theta_1)) > -e^{-O(d)}$.
\end{proof}


\section{Common Activations}
First, we consider the sign activation function. Under restrictions on the size of the input dimension or the number of hidden units, we can prove convergence results under the sign activation function, as it gives rise to a harmonic potential.

\Anote{ythis theorem is about $\theta\in R$, which is not of any practical interest. you should remove it.}
\begin{assumption}
\label{outputFixed}
All output weights $b_i = 1$ and therefore the output weights  $a_i = - b_i = -1$ are fixed throughout the learning algorithm. 
\end{assumption}

\begin{restatable}{lemma}{signcon}\label{signCon}
Let $\mathcal{M} = S^1$ and let Assumption \ref{outputFixed} hold. Let $L$ be as in \eqref{errSimp} and $\sigma$ is the sign activation function. Then $L$ admits no strict local minima, except at the global minima.
\end{restatable}


We cannot simply analyze the convergence of GD
on all $\theta_i$ simultaneously since as before, the pairwise
interaction terms between the $\theta_i$ present complications. Therefore, we now only consider the convergence guarantee of gradient descent on the first node, $\theta_1$, to some $w_j$, while the other nodes are inactive (i.e. $a_2,...,a_k = 0$). In essence, we are working with the following simplified loss function.
%
\begin{equation}\label{errLossUnit}
L(a_1,\theta_1) =  a_1^2 \Phi(\theta_1,\theta_1)  + 2\sum_{j=1}^k a_1b_j \Phi(\theta_1,w_j)
\end{equation}


\begin{restatable}{lemma}{signconv}
\label{SignConv}
Let $\mathcal{M} = S^1$ and $L$ be as in \eqref{errLossUnit} and $\sigma$ is the sign activation function. Then, almost surely over random choices of $b_1,...,b_k$, all local minima of $L$ are at $\pm w_j$. 
\end{restatable}
%
For the polynomial activation and potential functions, we also can show convergence under orthogonality assumptions on $w_j$. Note that the realizability of polynomial potentials is guaranteed in Section~\ref{sec:realizable}.

\begin{restatable}{theorem}{polystrict}
\label{PolyStrict}
Let $\mathcal{M} = S^{d-1}$. Let $w_1,...,w_k$ be orthonormal vectors in $\R^d$ and $\Phi$ is of the form $\Phi(\theta,w) = (\theta^Tw)^l$ for some fixed integer $l \geq 3$. Let $L$ be as in \eqref{errLossUnit}. Then, all critical points of $L$ are not local minima, except when $\theta_1 = w_j$ for some $j$.   
\end{restatable}





\section{Experiments}
\label{experiments}
For our experiments, our training data is given by $(x_i, f(x_i))$, where $x_i$ are randomly chosen from a standard Gaussian in $\R^d$ and $f$ is a randomly generated neural network with weights chosen from a standard Gaussian. We run gradient descent (Algorithm \ref{GD}) on the empirical loss, with stepsize around $\alpha = 10^{-5}$, for $T = 10^6$ iterations. The nonlinearity used at each node is sigmoid, and further, unlike the assumptions in the theoretical analysis, the top node is a sign function outputing binary variable. Thus a random guess for the network will result in a error rate of $0.5$ ($50\%$) Our experiments show that for depth-2 neural networks, even with non-linear outputs, the training error starting from a value of about $50\%$ diminishes quickly to under $2\%$. This seems to hold even when the width, the number of hidden nodes, is substantially increased (even up to 125 nodes), but depth is held constant; although as the number of nodes increases, the rate of decrease is slower. This substantiates our claim that depth-2 neural networks are learnable.

However, it seems that for depth greater than 2, the test error becomes significant  when width is high. When the width is a small constant, the increase in depth also impedes the learnability of the neural network and the training error does not get close enough to 0. It seems that for neural networks with greater depth, positive convergence results in practice are elusive.
\begin{table}[tb]
\caption{Test Error of Learning Neural Networks of Various Depth and Width}
\vskip 0.1in
\begin{center}
\begin{small}
\begin{sc}
\begin{tabular}{
  p{\dimexpr.2\linewidth-2\tabcolsep-1.3333\arrayrulewidth}|% column 0
   |p{\dimexpr.2\linewidth-2\tabcolsep-1.3333\arrayrulewidth}% column 1
  |p{\dimexpr.2\linewidth-2\tabcolsep-1.3333\arrayrulewidth}% column 2
  |p{\dimexpr.2\linewidth-2\tabcolsep-1.3333\arrayrulewidth}% column 3
  |p{\dimexpr.2\linewidth-2\tabcolsep-1.3333\arrayrulewidth}% column 4
   |p{\dimexpr.2\linewidth-2\tabcolsep-1.3333\arrayrulewidth}% column 5
  }
   \hline 
           & Width 5   &  Width 10   & Width 20 & Width 40     \\ \hline 
    Depth 2 & 0.0015   & 0.0017      &   0.0018 & 0.0019 \\ \hline
    Depth 3 & 0.0033   & 0.0264        &   0.1503 & 0.2362 \\ \hline
    Depth 5 & 0.0036   & 0.0579        &   0.2400 & 0.4397 \\ \hline
    Depth 9 & 0.0085   & 0.1662        &   0.4171 & 0.6071 \\ \hline
    Depth 17 & 0.0845   & 0.3862        &   0.4934 & 0.5777 \\ \hline
\end{tabular}
\end{sc}
\end{small}
\end{center}
\vskip -0.1in
\end{table}


% \begin{figure}[tb]
% \begin{center}


% \begin{figure}[h]
% \vskip 0.1in
%   \centering
% \includegraphics[width = 2.5in]{plotChangeDepth.jpg}\includegraphics[width = 2.71in]{plotChangeWidth.jpg}
% \caption{Left:Test Error of Networks of Varying Depth. Right: Test Error of Networks of Varying Width.}
% %  \fbox{\rule[-.5cm]{0cm}{4cm} \rule[-.5cm]{4cm}{0cm}}
% \end{figure}
\Snote{Do we need the figure? Figures should be the same size. Make the fonts larger, they
  are hard to read. The figure caption should read training error
  rather than test error.}
% \end{center}
\vskip -0.1in
We note that we have been using training error as a measure of
success, but it's possible that the true underlying parameters are not
learned. If our loss function were strongly convex, small training
error would imply a small norm in the parameter space. 



%%% Local Variables:
%%% mode: latex
%%% TeX-master: "icmlpaper2017.tex"
%%% End:

\section{Conclusion}

In this work, we view deep learning of neural networks in the context of electron-proton dynamics and analyzed the convergence of the underlying weight parameters of the neural network using arguments inspired from physics and non-convex optimization. To do so, we first established mathematical relationship between activation functions and their corresponding potentials. Next, we interpreted gradient descent as electrodynamics under a certain potential. Finally, we discovered classes of activation functions that give rise to positive convergence results, some of which relate to very commonplace activations, such as the sign and polynomial. For these classes of depth-2 neural networks, our results imply that they are provably learnable by deep learning. Our experiments seem to imply that higher depth neural networks are not learnable. 

However, we believe that convergence results for depth-2 neural networks can be extended to even more activation functions, such as the sigmoid or the ReLU. Also, we believe these convergence results can be proven with minimal assumptions. We hope that our work is a step in the theoretical understanding of the performance of neural networks seen in practice.


\bibliographystyle{icml2017}
\bibliography{biblio}

\newpage
\appendix
\section{Electron-Proton Dynamics}

\epdyn*

\begin{proof}
The initial values are the same. Notice that continuous gradient descent on $L(\boldsymbol{a,\theta})$ with respect to $\theta$ produces dynamics given by $\frac{d\theta_i(t)}{dt} = -\nabla_{\theta_i}L(\boldsymbol{a,\theta})$. Therefore,
\[\frac{d\theta_i(t)}{dt} = -2\sum_{j \neq i} a_i a_j
\nabla_{\theta_i}\Phi(\theta_i,\theta_j) - 2\sum_{j=1}^k
a_ib_j\nabla_{\theta_i} \Phi(\theta_i,w_j)\] 
And gradient descent does not move $w_i$. By definition, the dynamics corresponds to Electron-Proton Dynamics as claimed.
\end{proof}

\section{Realizable Potentials}
\label{realizable}

\subsection{Activation-Potential Calculations}
First define the {\it dual} of a function $f: \R \to \R$ is defined to be 
%
\[ \widehat{f}(\rho) = E_{X,Y \sim N(\rho)}[f(X)f(Y)],\]
%
where $N(\rho)$ is the bivariate normal distribution with $X, Y$ unit variance and $\rho$ covariance. This is as in \cite{DanielyFS16}.
%
\begin{lemma}\label{rotLem}
Let $\mathcal{M} = S^{d-1}$ and $\sigma$ be our activation function, then $\widehat{\sigma}$ is the corresponding potential function.
\end{lemma}

\begin{proof}
If $u, v$ have norm 1 and if $X$ is a standard Gaussian in $\R^d$, then note that $X_1 = u^TX$ and $X_2 = v^TX$ are both standard Gaussian variables in $\R^1$ and the covariance is $E[X_1X_2] = u^Tv$. 

Therefore, the dual function of the activation gives us the potential function.
\begin{align*}
E_{X}[\sigma(u^TX)\sigma(v^TX)] & =
E_{X,Y \sim N(u^Tv)}[\sigma(X)\sigma(Y)] \\
& = \widehat{\sigma}(u^Tv).
\end{align*}
\end{proof}

By Lemma \ref{rotLem}, the calculations of the activation-potential
for the sign, ReLU, Hermite, exponential functions are given in
\cite{DanielyFS16}. For the Gaussian and Bessel activation functions,
we can calculate directly. In both case, we notice that we may write
the integral as a product of integrals in each dimension. Therefore,
it suffices to check the following 1-dimensional identities.
\begin{align*}
  & \int_{-\infty}^\infty
    \sqrt{2}e^{x^2/4}e^{-(x-\theta)^2}\sqrt{2}e^{x^2/4}e^{-(x-w)^2} \frac{1}{\sqrt{2\pi}} e^{-x^2/2}\, dx \\
  & \qquad = \sqrt{\frac{2}{\pi}}\int_{-\infty}^\infty
    e^{-(x-\theta)^2}e^{-(x-w)^2} \, dx = e^{-(\theta -w)^2/2}
\end{align*}
\begin{align*}
& \int_{-\infty}^\infty (\frac{2}{\pi})^{3/2}e^{x^2/2}K_0(|x-\theta|)K_0(|x-w|)  \frac{1}{\sqrt{2\pi}} e^{-x^2/2}\, dx \\
& \qquad 
= \int_{-\infty}^\infty \frac{2}{\pi^2}K_0(|x-\theta|)K_0(|x-w|) \, dx
  = e^{-|\theta -w|}
\end{align*}


The last equality follows by Fourier uniqueness and taking the Fourier transform of both sides, which are both equality $\sqrt{2/\pi}(\omega^2+1)^{-1}$. 


\subsection{Characterization Theorems}

\tranReal*

\begin{proof}
Let $h(x) = \FT^{-1}(\sqrt{\FT(f)})(x)$ and this is well-defined since the fourier transform was non-negative everywhere. Now, let $\sigma(x,w) = (2\pi)^{1/4}e^{x^2/4}h(x-w)$ and by assumption, it is bounded almost everywhere. Realizability follows by checking:
%
\begin{align*}
    E_{X \sim N}[\sigma(X,w)\sigma(X,\theta)]  &= \int_{\R^n} h(x-w)h(x-\theta) \, dx \\
    &= \int_{\R^n} h(x)h(x-(\theta-w)) \, dx \\
    &= \FT^{-1}(\FT(h\ast h)(\theta -w)) \\
    &= \FT^{-1}(\FT(h)^2(\theta - w)) \\
    &= \FT^{-1}(\FT(f)(\theta - w)) \\
    &= f(\theta - w) 
\end{align*}
\end{proof}




\rotReal*

\begin{proof}
By \ref{rotLem} and due to the orthogonality of hermite polynomials, if $f = \sum_i a_i h_i$, where $h_i(x)$ is the i-th Hermite polynomial, then
%
\[\widehat{f}(\rho) = \sum_{i} a_i^2 \rho^i\]

Therefore, any function with non-negative taylor coefficients is a valid potential function, with the corresponding activation function determined by the sum of hermite polynomials, and the sum is bounded almost everywhere by assumption.
\end{proof}

\subsection{Further Characterizations}

\begin{definition}
Let $\Phi$ be integrable or square integrable. Then, $\Phi$ is $\FT$-integrable if $\FT({\Phi}(\omega))$ is integrable, where $\FT$ is the standard Fourier transform.
\end{definition}

\begin{lemma}\label{sumReal}
Let $\mathcal{M} = \R^{d}$. If $\Phi_1, \Phi_2$ are $\FT$-integrable realizable potentials, then $\Phi_1 + \Phi_2$ is $\FT$-integrable realizable.
\end{lemma}

\begin{proof}
Since $\Phi_1, \Phi_2$ is $\FT$-integrable, then since the Fourier transform is linear, $\Phi_1+\Phi_2$ is also $\FT$-integrable. Since $\Phi_1, \Phi_2$ are realizable, they are positive definite and by Bochner's theorem, their Fourier transforms are non-negative. Therefore, we know $\FT(\Phi_1 + \Phi_2)\geq 0$ and $\sqrt{\FT(\Phi_1 + \Phi_2)}$ is defined and is square-integrable. By properties of the Fourier transform, $\FT^{-1}(\sqrt{\FT(\Phi_1 + \Phi_2)})$ exists and is also square-integrable, therefore bounded almost anywhere. Lastly, by Theorem \ref{thm:tranReal}, we conclude that $\Phi_1 + \Phi_2$ is realizable.
\end{proof}

\begin{lemma}\label{intReal}
Let $w(x) \geq 0$ be a positive weighting function such that $\int_a^b w(x) \, dx$ is bounded. If $\Phi_x$ is a parametrized family of $\FT$-integrable realizable potentials, then, $\int_a^b w(x) \Phi_x$ is $\FT$-integrable realizable.
\end{lemma}

\begin{proof}
Let $\Phi = \int_a^b w(x) \Phi_x$. From linearity of the Fourier transform and $\int_a^b w(x)\, dx$ is bounded, we know that $\Phi$ is $\FT$-integrable. Then, by the continuity of Fourier transform, we can argue similarly as the previous lemma.
\end{proof}



\begin{lemma}\label{baseConstruct}
Let $\mathcal{M} = \R^d$ for $d \equiv 3 \mod 4$. Then, for any $\epsilon, t > 0$, there exists a $\FT$-integrable realizable $\Phi$ such that for $t \geq r > \epsilon$, $\Phi^{(d-1)}(r) = t -r$ and $\Phi^{k}(r) = 0$ for $r > t$ for all $0 \leq k \leq d$.
\end{lemma}

\begin{proof}
First, consider a radial activation function $\sigma_t(x, \theta) = \bf{1}_{\|\theta - x\| \leq t/2}$, which is the indicator in the disk of radius $t/2$. This activation, when weighted correctly, gives rise to a realizable radial potential that is simply the convolution of $\sigma_t$ with itself, measuring the volume of the intersection of two spheres of radius $t$ centered at $\theta$ and $w$.
%
\[\Phi_t(\theta, w) = E_X[\sigma_t(X,\theta)\sigma_t(X,w)]\]
\[= \begin{cases}
C\int_{\|\theta - w\|/2}^{t/2} ((t/2)^2 - x^2)^{(d-1)/2} \, dx & \|\theta - w\| \leq t\\
0 & o.w.\\
\end{cases}\]
%
Therefore, as a function of $r=\|\theta - w \|$, we see that when $r \leq t$, $\Phi_t(r) = C\int_{r/2}^{t/2} ((t/2)^2-x^2)^{(d-1)/2} \, dx$ and $\Phi_t'(r) = -C'((t/2)^2-(r/2)^2)^{(d-1)/2}$. Since $d \equiv 3 \mod 4$, we notice that $\Phi_t'$ has a positive coefficient in the leading $r^{d-1}$ term and since it is a function of $r^2$, it has a zero $r^{d-2}$ term. Therefore, we can scale $\Phi_t$ such that 

\[\Phi_t^{(d-1)}(r) = \begin{cases}
r & r \leq t\\
0 & o.w. \\
\end{cases} \]

$\Phi_t$ is clearly realizable and we claim that it is $\FT$-integrable. First, $\Phi_t$ is integrable. Furthermore, $\sigma_t$ is integrable and square integrable, so its Fourier transform exists and by Parseval's, $\FT{\sigma_t}$ is square integrable. By the convolution properties of the Fourier transform, $\FT{\Phi_t} = \FT{\sigma_t}^2$, it follows that $\Phi_t$ is $\FT$-integrable.

Now, for any $\epsilon > 0$, let us construct our desired $\Phi$ by taking a positive sum of $\Phi_t$ and then appealing to Lemma \ref{intReal}. Consider

\[\Phi(r) = \int_{\epsilon}^{t} \frac{1}{x^2}\Phi_x(r) \, dx\]

First, note that the total weight $\int_\epsilon^t \frac{1}{x^2}$ is bounded. Then, when $r \geq t$, since $\Phi_x(r) = 0$ for $x \leq t$, we conclude that $\Phi^{(k)}(r) = 0$ for any $k$. Otherwise, for $\epsilon < r < t$, we can interchange limits by non-negativity to get

\[\Phi^{(d-1)}(r) = \int_{\epsilon}^r \frac{1}{x^2}\Phi_x^{(d-1)}(r) \, dx + \int_{r}^t \frac{1}{x^2} \Phi_x^{(d-1)}(r) \, dx\]

\[ = 0 + \int_r^t \frac{r}{x^2} \, dx = 1 -r/t \]

Scaling by $t$ gives our desired claim.
\end{proof}

\begin{lemma}\label{transConstruct}
Let $\mathcal{M} = \R^d$ for $d \equiv 3 \mod 4$ and let $\Phi(r)$ be a radial potential. Also, $\Phi^{(d)}(r) \leq 0$ and $\Phi^{(d+1)}(r)\geq 0$ for all $r > 0$ and $\lim_{r \to \infty} \Phi^{(k)}(r) = 0$ for all $0 \leq k \leq d$. 

Then, for any $\epsilon > 0$, there exists a $\FT$-integrable realizable potential $\overline{\Phi}$ $\Phi$ such that $\overline{\Phi}^{(k)}(r) = \Phi^{(k)}(r)$ for all $0 \leq k \leq d-1$ and $r \geq \epsilon$. 
\end{lemma}

\begin{proof}
By Lemma \ref{baseConstruct}, we can find $\Phi_t$ such that for $r \geq \epsilon$, we have $\Phi_t^{(d-1)}(r) = t - r $. Therefore, consider 

\[\overline{\Phi}(r) = \int_{\epsilon}^\infty \Phi^{(d+1)}(x) \Phi_x(r) \, dx\] 

Note that this is a positive sum with $\int_{\epsilon}^\infty \Phi^{(d+1)}(x) \, dx = -\Phi^{(d)}(\epsilon) < \infty$. By the non-negativity of $-\Phi^{(d)}_x$, we can interchange limits to get

\[\overline{\Phi}^{(d-1)}(r) = \int_{\epsilon}^\infty  \Phi^{(d+1)}(x) (\Phi_x^{(d-1)}(r)) \, dx\] 

\[ = \int_r^\infty \Phi^{(d+1)}(x) \int_r^x 1 \, dy \,dx \]

\[= \int_r^\infty \int_y^\infty \Phi^{(d+1)}(x) \, dx \, dy = \int_r^\infty  -\Phi^{(d)}(y) \, dy\]

\[ = \Phi^{(d-1)}(r)\]

Now, since $\overline{\Phi}^{(d-1)}(r) = \Phi^{(d-1)}(r)$ for $r\geq \epsilon$ and $\lim_{r\to\infty} \overline{\Phi}^{k}(r) = \lim_{r\to\infty} {\Phi}^{k}(r) = 0$ for $0 \leq k \leq d-1$, repeated integration gives us our claim.
\end{proof}

\begin{theorem}\label{almostHarmReal}
Let $\mathcal{M} = \R^d$ for $d \equiv 3 \mod 4$. Then, for any $\epsilon > 0$, we can construct a radial potential $\Phi_\epsilon(r)$ that is $1$-harmonic when $r \geq \epsilon$.
\end{theorem}

\begin{proof}
Consider a potential of the form $\Phi(r) = p(r)e^{-r}/r^{d-2}$. We claim that there exists a polynomial $p$ of degree $k = (d-3)/2$ with non-negative coefficients and $p(0) = 1$ such that $\Phi$ is $1$-harmonic.

When $d = 3$, it is easy to check that $\Phi(r) = e^{-r}/r$ is our desired potential. Otherwise, by our formula for the radial Laplacian in $d$ dimensions, we want to solve the following differential equation:

\[\Delta \Phi =  \frac{1}{r^{d-1}} \frac{\partial}{\partial r} (r^{d-1} \frac{\partial \Phi}{\partial r}) = \Phi\]

Solving this gives us the following second-order differential equation on $p$

\[rp'' - (d+2r-3)p' +(d-3)p = 0\]

Let us write $p(r) = \sum_{i=0}^k a_i x^i$. Then, substituting into our differential equation gives us the following equations by setting each coefficient of $x^i$ to zero:

$x^0:$ $a_0 = a_1$,

$x^i$:  $a_{i-1}(d-3 - 2(i-1)) = a_i (i(d-3 - i-1))$

$x^k:$ $-2k +d-3 = 0$

Note that for $i = 1$, this becomes $a_0 = a_1$, which is already satisfied. Thus, setting $a_0 = 1$ and running the recurrence gives us our desired polynomial. Note that the recurrence is valid and produces positive coefficients since $i < k  = (d-3)/2$. Therefore, our claim follows and $\Phi$ is $1$-harmonic.

Lastly, we assert that $\Phi^{(j)}(r)$ is non-negative for $j$ even and non-positive for $j$ odd. Our second claim is that if $\Phi$ is of the form $\Phi(r) = p(r) e^{-r}/r^{l}$ for some $p$ of degree $k < l$ and $p$ has non-negative coefficients, then $\Phi'(r) = - q(r) e^{-r}/r^{l+1}$ for some $q$ of degree $k+1$ with non-negative coefficients. 

It suffices to prove our second claim. Differentiating $\Phi$ gives:

\[\Phi' = \frac{e^{-r}}{r^{l+1}} (rp'(r) - (l + r)p(r))\]

It is clear that if $p$ has degree $k$, then $q(r) = (l+r)p(r) - rp'(r)$ has degree $k+1$, so it suffices to show that it has non-negative coefficients. Let $p_0,..., p_k$ be the non-negative coefficients of $p$. Then, by our formula, we see that 

$q_0 = l p_0$

$q_i = lp_i - ip_i + p_{i-1} = (l-i)p_i + p_{i-1}$ 

$q_{k+1} = p_k$

Since $i \leq k < l$, we conclude that $q$ has non-negative coefficients. Finally, our assertion follows with induction since $\Phi^{(0)}(r)$ is non-negative and has our desired form with $k = (d-3)/2 < d-2$. By Lemma \ref{transConstruct}, our theorem follows.
\end{proof}

\begin{theorem}
Let $\mathcal{M} = \R^d$ for $d \equiv 3 \mod 4$ and let $L$ be as in \eqref{errLoss}. For any $\epsilon > 0$, we can find a realizable potential $\Phi$ such that if $\boldsymbol{(a,\theta)}$ is a local minima of $L$, then for all $i$, either 1) there exists $S \subseteq [k]$ such that $i\in S$ and $\sum_{i \in S} a_i \sigma(\theta_i, x) = 0$ or 2) there exists $j$ such that $\|\theta_i - w_j\| < \epsilon$.
\end{theorem}

\begin{proof}
 The proof is similar to Theorem \ref{EigStrict}. Let $\Phi$ be the realizable potential in \ref{almostHarmReal} such that $\Phi(r)$ is 1-harmonic when $r \geq \epsilon/k$. And let $\boldsymbol{a,\theta}$ be a local minima of $L$. Assume there is a non-empty subset $S \subseteq [k]$ such that for each $i \in S$, $\|w_j - \theta_i\|\geq \epsilon/k$ for all $j$, and $\|\theta_i - \theta_j\| \leq \epsilon/k$ for all $j\in [k]/S$. WLOG, let $S = \{1,\dots,l\}$. 
  
We consider changing all
$\theta_1, \ldots, \theta_{l}$ by the same $v$ and define 
%
\[H({\bf a}, v) = L({\bf a},\theta_1+v,...,\theta_l+v, \theta_{l+1}
\ldots, \theta_k).\]

The optimality conditions on ${\bf a}$ are 
\begin{align*}
0 = \pd{H}{a_i} = 2a_i  + 2\sum_{j\neq i} a_j \Phi(\theta_i,\theta_j)
  + 2\sum_{j=1}^k b_j \Phi(\theta_i,w_j).
\end{align*}
%
Next, since $\Phi(r)$ is $1$-harmonic for $r \geq \epsilon/k$, we may differentiate as $\theta_i \neq w_j$ and compute the Laplacian as 
\begin{align*}
\Delta_v H & = \sum_{i=1}^l a_i \left(2\sum_{j=1}^k b_j
  \Phi(\theta_i, w_j) + 2\sum_{j=l+1}^k a_j
  \Phi(\theta_i, \theta_j)\right) \\
& = \sum_{i=1}^l a_i \left(-2a_i\Phi(\theta_i, \theta_i) - 2
  \sum_{j = 1, j\neq i}^l  a_j \Phi(\theta_i,\theta_j)\right) \\
& = -2\left(\sum_{i=1}^l a_i^2 \Phi(\theta_i, \theta_i)
+\sum_{i \neq j}^l a_i a_j \Phi(\theta_i, \theta_j)\right) \\
&= -2\expt\left[\left( \sum_{i=1}^l a_i \sigma(\theta_i,X)\right)^2\right]
\end{align*} 
%
If $f(x) = \sum_{i=1}^l a_i\sigma(\theta_i, x) = 0$, then we satisfy condition 1. Otherwise, we conclude that the Laplacian is
strictly negative, so we are not at a local minimum.  

Therefore, it must be the case that there exists ${j_1},\cdots {j_q}$ such that $\|\theta_i - \theta_{j_1} \| < \epsilon/k$ and
$\|\theta_{j_{i}} - \theta_{j_{i+1}}\| < \epsilon/k$, and
$\|\theta_{j_q}- w_j\| <\epsilon/k$ for some $w_j$. So, there exists $j$, such that $\|\theta_i - w_j\| < \epsilon$. 


\end{proof}


\section{Earnshaw's Theorem}

\earnshaw*
\begin{proof}
  If $(\boldsymbol{a,\theta})$ is a differentiable strict local
  minima, then for any $i,$ we must have
\[\nabla_{\theta_{i}} L = 0, \text{ and }  \Tr(\nabla^2_{\theta_i}L) > 0.\]
Since $\Phi$ is harmonic, we also have
\begin{align*}
&  \Tr(\nabla^2_{\theta_i}L(\theta_1,...,\theta_n)) = \Delta_{\theta_i} L \\
&  =  2\sum_{ j\neq i} a_ia_j \Delta_{\theta_i}\Phi(\theta_i,\theta_j)
  + 2\sum_{j=1}^ka_ib_j  \Delta_{\theta_i}\Phi(\theta_i,w_j) = 0,
\end{align*}
which is a contradiction. In the first line, there is a factor of 2 by symmetry.
\end{proof}


\section{Generalization Error and Iteration Bounds}
\Snote{Please comment out results and sections that are not relevant.}
\label{finite}
 
The design and analysis of gradient descent has so far assumed that we can calculate expectations perfectly. In reality, these expectations are instead replaced with empirical means. The approximate calculation of our potential and all its derivatives with samples are justified by the generalization error bounds implied by Rademacher complexities. 

Unfortunately, we cannot directly use the Gaussian distribution as it is unbounded. Therefore, we assume that our drawing distribution is the truncated Gaussian distribution in $\R^d$ such that our samples are always bounded in $l_2$ norm by $B$. We will show that applying this truncation will not affect the expectation very much. 

\begin{definition}
$Y$ follows a truncated Gaussian distribution at $B$ in $\R^d$ if $Y = X | ( \|X\| \leq B)$, where $X$ is a standard Gaussian in $\R^d$ and $ X \Big|S$ indicates the random variable $X$ conditioned on event $S$.
\end{definition}
%
\begin{lemma}
\label{choppedLem}
Let $f : \R^d \to \R$ be a function such that $|f(x)| \leq c\|x\|^p$. Let $X$ be a standard Gaussian in $\R^d$ and let $Y$ be a truncated Gaussian at $B$. Then, there exists $B = \poly(d,p,\log(c), \log(1/\epsilon))$ such that 
\[ \left|\expt[f(X)] - \expt[f(Y)] \right| \leq \epsilon\]
\end{lemma}

\begin{proof}
By standard concentration bounds and analysis, 
\begin{align*}
& |E[f(X)] - E[f(X)| \|X\|\leq B]| \\
& \qquad \leq (\frac{1}{P(\|X\|\leq B)}- 1) E[|f(X)|{\bf 1}_{\|X\|\leq B}]| \\
& \qquad \qquad + |E[f(X){\bf 1}_{\|X\|>B}]| \\
& \qquad \leq E[|f(X)|{\bf 1}_{\|X\|>B}] + 2cB^pe^{-B^2/8d}
\end{align*}


Taking $B = \poly(d,p,\log(c), \log(1/\epsilon))$ will make the second term $< \epsilon/2$, then the first term is also bounded by:
\begin{align*}
  \expt[|f(X)|{\bf 1}_{\|X\|>B}] & \leq (2\pi)^d \int_B^\infty cr^pe^{-r^2/2}r^{d-1}\,dr \\
                                  &\leq C(2\pi)^dB^{p+d}e^{-B^2/2} < \epsilon/2
\end{align*}
\end{proof}

Next, we can appeal to the following well-cited theorems and standard techniques. For a better understanding of the notation and theorems used, we refer the refer to \cite{bartlett2002rademacher}.
%
\begin{theorem}[\cite{bartlett2002rademacher}]\label{rademacher}
Consider a function class $\mathcal{F}$ of functions $f : \mathcal{X} \to [0,1]$. And let $ x_1,...,x_n \in \mathcal{X}$ be i.i.d. samples selected according to some distribution $\mathcal{D}$. Let the Rademacher complexity of $\mathcal{F}$ to be 
\[R_n (\mathcal{F}) = E_{x_i,\sigma_i}\left[\sup_{f \in \mathcal{F}} \left|\frac{2}{n}\sum_{i=1}^nf(x_i)\sigma_i\right|\right]\]

where $\sigma_i$ are Rademacher variables. Then, for any $n$ and $\delta \in (0,1)$, with probability at least $1-\delta$, 
\[|E_{X}[f(X)]  - \frac{1}{n}\sum_{i=1}^n f(x_i)| \leq R_n(\mathcal{F}) + \sqrt{\frac{8\ln(2/\delta)}{n}} \]
for all $f \in\mathcal{F}$ simultaneously.
\end{theorem}

For all of our polynomial time bounds, whether we are calculating the potential or its derivatives, we are interested in bounding the Rademacher complexity of the class of functions of the form $\sigma_1(x^T\theta) \sigma_2(x^Tw)$, where $\sigma_1,\sigma_2$ are scalar functions. Let $f \circ g$ denote the composition of functions $f(x)$ and $g(x)$. If $\mathcal{G}$ is a class of functions, let $f \circ \mathcal{G}$ denote the class of functions $f \circ g$ for all $g \in \mathcal{G}$. 
%
\begin{theorem}\label{complexity}
Let $\mathcal{X} = \{x \in \R^d, \|x\|_2\leq B\}$ and let $\sigma_1,\sigma_2 : \R \to [-C,C]$ be $L$-Lipschitz functions. Consider the following class of functions
%
\[\mathcal{F}_{\sigma_1,\sigma_2} = \{x\to\sigma_1(x^T\theta) \sigma_2(x^Tw) | x\in\mathcal{X}, \theta, w \in S^{d-1}\}.\]
Then, \[R_n(\mathcal{F}_{\sigma_1,\sigma_2}) \leq 48CLB\sqrt{\frac{2}{n}}\]
\end{theorem}
\begin{proof}
First, let's define a simpler function class: $\mathcal{G} = \{x\to x^T\theta  | x\in\mathcal{X}, \theta \in S^{d-1}\}$. By simple Rademacher bounds on the class of linear functions \cite{kakade2009complexity}, 
 %
 \[R_n(\mathcal{G}) \leq B\sqrt{\frac{2}{n}}\]
%
Since $\sigma_1,\sigma_2$ are L-lipschitz, we apply standard structural results in \cite{bartlett2002rademacher}
%
\[R_n(\sigma_i\circ \mathcal{G}) \leq 2LB\sqrt{\frac{2}{n}}\]
%
Let $s(x) = x^2$, then $s$ is $2C$-Lipschitz when $|x|\leq C$, so 
%
\[R_n(s\circ \sigma_i\circ\mathcal{G}) \leq 8CLB\sqrt{\frac{2}{n}},
R_n(s\circ(\sigma_1\circ \mathcal{G} +\sigma_2\circ \mathcal{G})) \leq
32 CLB\sqrt{\frac{2}{n}}\]
 
Since $2\sigma_1(x^T\theta)\sigma_2(x^Tw) = (\sigma_1(x^T\theta)+\sigma_2(x^Tw))^2 - \sigma_1(x^T\theta)^2 - \sigma_2(x^Tw)^2$, we conclude that 
\[R_n (\mathcal{F}_{\sigma_1,\sigma_2}) \leq 48CLB\sqrt{\frac{2}{n}}\]
\end{proof}

\begin{theorem}
\label{genErrBound}
Let $\mathcal{M} = S^{d-1}$ and $L$ be as in \ref{errLoss} with potential function $\Phi$ corresponding to activation $\sigma$. Let ${\bf b}$ be $\poly(d)$-bounded and $|\sigma(x)|, |\sigma'(x)|,|\sigma''(x)|, |\sigma'''(x)|$ are all bounded by $c|x|^m$  

Then, in $\poly(d^m, c, 1/\epsilon, \log(1/\zeta))$ samples, we can compute $\hat{L}$ such that with probability $1-\zeta$, we have simultaneously $\|\widehat{L}(\boldsymbol{ a, \theta}) -L(\boldsymbol{ a, \theta})\|, \|\nabla \widehat{L}(\boldsymbol{ a, \theta}) = L(\boldsymbol{ a, \theta})\|, \|\nabla^2\widehat{L}(\boldsymbol{ a, \theta}) -\nabla^2 L(\boldsymbol{ a, \theta})\| \leq \epsilon$ for all $\boldsymbol{a,\theta}$ where $\|{\bf a}\| \leq \poly(d)$. 
\end{theorem}

\begin{proof}
  We first bound the generalization error of each $\Phi$. Notice that
  our approximation to $\Phi$ is done by first drawing $n$
  i.i.d. samples $x_i \sim \mathcal{D}_B$, where $\mathcal{D}_B$ is
  the Gaussian in $\R^d$ truncated by the ball of radius
  $B = \poly(d,m,\log(c),\log(1/\epsilon))$. Then, we calculate the empirical
  average.
%
\[\widehat{\Phi}(\theta,w) = \frac{1}{n}\sum_{i=1}^n \sigma(x_i^T\theta)\sigma(x_i^Tw) \]

Since $|x_i^T\theta|\leq B$, we conclude that $\sigma, \sigma'$ is bounded by $\poly(B^m,c)$. Therefore, by Theorem \ref{rademacher}, \ref{complexity} and by simple union bounds over the $\poly(k) = \poly(d)$ sum of $\Phi$, we conclude that with probability $1-\zeta$, if we choose $n = \poly(d^p,c, 1/\epsilon, \log(1/\zeta))$, we have 
%
\[|E_{X\sim \mathcal{D}_B}[\sigma(X^Tw)\sigma(X^T\theta)] -
\widehat{\Phi}(\theta^Tw)| \leq \epsilon/\poly(d),\]
for all $\theta, w \in \mathcal{M}$. And combining with Lemma
\ref{choppedLem}, we conclude that
$|\widehat{\Phi}(\theta,w) - \Phi(\theta,w)|\leq \epsilon/\poly(d)$. Now, since $L(\boldsymbol{a,\theta})$ is a sum of $\poly(d)$ number of $\Phi$ weighted by scalars bounded by $\poly(d)$, we can apply a union bound and triangle inequality to conclude $\|\widehat{L} - L\| \leq \epsilon$ for all $\boldsymbol{a,\theta}$ where $\|{\bf a}\| \leq \poly(d)$.  We proceed with the same proof for the first and second derivatives and use a
union bound to derive our claim.
\end{proof}


\subsection{Finite Iteration Bounds} 
To derive a finite iteration bound, we will apply stochastic gradient descent to our objective function and use standard martingale techniques for analysis. We will need to slightly alter the main result in \cite{GeHJY15} because we lack strong convexity assumptions. Also, we will accordingly alter the stochastic gradient descent algorithm to terminate upon finding a critical point that is $\gamma$-strict, for small $\gamma$.
%






\begin{theorem}\label{strongConverge}
  Let $L :\Omega \to \R$ be a thrice differentiable function such that
  $|L(x)| \leq B_0, \|\nabla L(x)\| \leq B_1, \|\nabla^2 L(x)\| \leq B_2,\|\nabla^2L(x)
  -\nabla^2L(y)\| \leq B_3\|x - y\|$
  for all $x,y\in\Omega$. Also, assume that we access to stochastic
  function $\widehat{L}$ such that
  $\|\widehat{L}(x) -L(x)\|, \|\nabla \widehat{L}(x) - L(x)\|,
  \|\nabla^2\widehat{L}(x) -\nabla^2 L(x)\| \leq \epsilon/3$
  for some $\epsilon < 1$ for all $x\in \Omega$.

Then, we can choose stepsize $\eta = 1/\poly(d,B_0, B_1, B_2, B_3,\rho,1/\epsilon,\log(1/\zeta))$, such that with probability at least $1-\zeta$, running Algorithm \ref{SGD} on $\widehat{L}$ initialized at $x_0$ with stepsize $\eta$  returns a point $x_\eta$ that is in $\mathcal{M}_{L, \epsilon}$ after at most $T = \poly(d,B_0, B_1, B_2, B_3,\rho,1/\epsilon, \log(1/\zeta))$ iterations. Furthermore, $L(a,x_T) \leq L(a, x_0)$ with high probability.
\end{theorem}

The theorem proof builds on the following two lemmas in\cite{GeHJY15} , which allows us to make significant progress on the objective function. We adopt the notation in the paper where $\widetilde{O}, \widetilde{\Omega}$ drops factors polynomially dependent on parameters other than $\eta$. 

\begin{lemma}(Lemma 7 in \cite{GeHJY15})\label{GeLem7}		
Under the assumptions of Theorem \ref{strongConverge}, for any point with $\|\nabla f (w_t) \|\geq C\sqrt{\eta}$ and $C\sqrt{\eta} \leq \epsilon$, after one iteration $\expt[f(w_{t+1})] \leq f(w_t) - \widetilde{\Omega}(\eta^2)$.
\end{lemma} \Snote{I assume this $\eta$ is not the same as in the
  algorithm. Fix} \Rnote{It's the same as in Thm \ref{strongConverge}}

\begin{lemma}(Lemma 9 in \cite{GeHJY15})\label{GeLem9}
Under the assumptions of Theorem \ref{strongConverge}, for any point where $\|\nabla f(w_t)\|\leq C\sqrt{\eta}$ and $\lambda_{min}(\nabla^2 f (w_t)) \leq -\gamma$, there is a number of steps $T$ such that $E[f(w_{t+T})] \leq f(w_t) - \widetilde{\Omega}(\eta)$ and $T \leq \widetilde{O}(1/\eta)$. Furthermore, $\|w_t - w_{t+i}\| \leq O(\eta^{1/2})$ for all $1 \leq i \leq T$.
\end{lemma}

\begin{proof}
If Algorithm \ref{SGD} succeeds, then by triangle inequality and our error bounds on the stochastic function $\widehat{L}$ and its derivatives, we know that $x \in \mathcal{M}_\epsilon$. So, it suffices to argue that our algorithm succeeds with high probability.

By Lemma \ref{GeLem7} and \ref{GeLem9}, we see that if
$x_i \not \in\mathcal{M}_{\epsilon/3}$ for all the iterations, then in
expectation, our objective function decreases by $poly(\eta)$ and is a strict supermartingale. By using Azuma's inequality, this occurs with probability
$1-\zeta$ in $poly(\eta,\log(1/\zeta))$ iterations. Since our objective function is bounded by $\poly(d,B)$, we conclude that stochastic gradient descent must have encountered a
point $x_i$ such that $x_i \in \mathcal{M}_{\epsilon/3}$ after at most
$\poly(d,B,C,\rho,1/\epsilon,1/\gamma, \log(1/\zeta))$. These lemmas also imply that $L(x_i) \leq L(x_0)$ with high probability.

Finally, by our bounds on $\widehat{L}$ and $L$, we conclude that both
$\|\nabla\widehat{L}(x_i)\| \leq 2\epsilon/3$ and $\lambda_{min}(\nabla^2 \widehat{L}(x_i)) \geq -2\epsilon/3$. So, our algorithm succeeds with high probability.
\end{proof}


\begin{proof}
We will choose $\delta, \alpha = 1/\poly(d,1/\epsilon)$ and $m = O(\log d \log (1/\delta)/\epsilon)$. First, to use Theorem \ref{strongConverge}, we check the regularity conditions. By assumption, we can choose $B, L, \rho$ to be $\poly(d,1/\epsilon)$ since $\Phi_m$ and the second and third partials of $\Phi_m$ are all bounded by $\poly(d,1/\epsilon)$ in $\mathcal{M}$. Furthermore, our activation function $\sigma$ and its derivatives are bounded in magnitude by $c|x|^{m}$, where $m = O(\log d \log (1/\delta)/\epsilon)$ and by \cite{Hermite}\Rnote{cite something better than wikipedia, a}, we can let $c = O(m^m)$. By Theorem \ref{genErrBound}, with high probability, we can construct a stochastic oracle up to $\epsilon/3$ error with sample complexity $d^{poly(d,1/\epsilon)}$. Therefore, we conclude by Theorem \ref{strongConverge} that we are in $\mathcal{M}_{G,\delta^2/(4\poly(d))}$ in $\poly(d,1/\epsilon,1/\delta)$ iterations.
\end{proof}

ADDING DETERMINISTIC LEMMAS


\begin{lemma}\label{GradDecrease}		
Let the assumptions of Theorem \ref{strongConverge} hold in a $(\alpha B_1)$-neighborhood of $x$. If $\|\nabla f (x) \|\geq \eta$ and $x'$ is reached after one iteration of gradient descent (Algorithm \ref{GD}) with stepsize $\alpha \leq \frac{1}{B_2}$, then $\|x' - x\| \leq \alpha B_1$ and $f(x') \leq f(x) - \alpha\eta^2/2$.
\end{lemma} 

\begin{proof}
The gradient descent step is given by $x' = x - \alpha \nabla f(x)$. The bound on $\|x' - x\|$ is clear since $\|\nabla f(x) \| \leq B_1$.
\begin{align*}
f(x') &\leq f(x) - \alpha \nabla f(x)^T\nabla f(x)^T + \alpha^2\frac{B_2}{2} \|\nabla f(x)\|^2 \\
&\leq f(x) - (\alpha - \alpha^2 \frac{B_2}{2}) \eta^2 
\end{align*}
For $0 \leq \alpha \leq \frac{1}{B_2}$, we have $\alpha - \alpha^2B_2/2 \geq \alpha/2$, and our lemma follows.
\end{proof}

\begin{lemma}\label{HessianDecrease}
Let the assumptions of Theorem \ref{strongConverge} hold in a $(\alpha B_2)$-neighborhood of $x$. If $\lambda_{min}(\nabla^2 f (x)) \leq -\gamma$ and $x'$ is reached after one iteration of Hessian descent (Algorithm \ref{HD}) with stepsize $\alpha \leq \frac{1}{B_3}$, then $\|x' - x\| \leq \alpha B_2$ and $f(x') \leq f(x) - \alpha^2 \gamma^3/2$.
\end{lemma}

\begin{proof}
The gradient descent step is given by $x' = x + \beta v_{min}$, where $v_{min}$ is the unit eigenvector corresponding to $\lambda_{min}(\nabla^2f(x))$ and $\beta = -\alpha\lambda_{min}(\nabla^2 f(x))sgn(\nabla f(x)^Tv_{min})$. Our bound on $\|x' - x\|$ is clear since $\gamma \leq B_2$.
\begin{align*}
f(x') &\leq f(x) + \beta\nabla f(x)^Tv_{min} + \beta^2 v_{min}^T\nabla^2f(x)v_{min} + \frac{B_3}{6} |\beta|^3 \|v_{min}\|^3 \\
&\leq f(x) - |\beta|^2 \gamma + \frac{B_3}{6} |\beta|^3
\end{align*}
The last inequality holds since the sign of $\beta$ is chosen so that $\beta \nabla f(x)^Tv_{min} \leq 0$. Now, since $|\beta| = \alpha \gamma \leq \frac{\gamma}{B_3}$, $-|\beta|^2\gamma + \frac{B_3}{6} |\beta|^3 \leq - \alpha^2 \gamma^3/2$. 
\end{proof}

%
\begin{theorem}\cite{nesterov2013introductory}\label{quadConverge}
  Let $x_0 \in \Omega = \{ x \in \R^d | \|x \| \leq \poly(d)\}$ and let
  $L(x) = x^TAx + b^Tx$ be a quadratic loss, where $A$ is a positive
  semi-definite matrix with maximum eigenvalue bounded by $\beta$. Then, running
  Algorithm \ref{GD} on $L$ with stepsize $\alpha = 1/\beta$ converges
  to $x_T$ such that
  \[L(x_T) - \min_{x \in \Omega} L(x) \leq \frac{\beta \poly(d)}{T}\]
\end{theorem} 


%%% Local Variables:
%%% mode: latex
%%% TeX-master: "icmlpaper2017.tex"
%%% End:


\section{Convergence of Almost $\lambda$-Harmonic Potentials}\label{App:EigenFunc}

\AlmostHarmConv*

\begin{proof}
 The proof is similar to Theorem \ref{EigStrict}. Let $\Phi_\epsilon$ be the realizable potential in \ref{almostHarmReal} such that $\Phi_\epsilon(r)$ is $\lambda$-harmonic when $r \geq \epsilon$ with $\lambda = 1$. Note that $\Phi_\epsilon(0) = 1$ is normalized. And let $\boldsymbol{(a,\theta)} \in \mathcal{M}_{G,\delta}$. 
 
WLOG, consider $\theta_1$ and a initial set $S_0 = \{ \theta_1\}$ containing it. For a finite set of points $S$ and a point $x$, define $d(x,S) = \min_{y \in S} \| x - y\|$. Then, we consider the following set growing process. If there exists $\theta_i, w_i \not \in S_j$ such that $d(\theta_i, S_j) < \epsilon$ or $d(w_i, S_j) < \epsilon$, add $\theta_i, w_i$ to $S_j$ to form $S_{j+1}$. Otherwise, we stop the process. We grow $S_0$ to until the process terminates and we have the grown set $S$.

If there is some $w_j \in S$, then it must be the case that there exists ${j_1},\cdots {j_q}$ such that $\|\theta_1 - \theta_{j_1} \| < \epsilon$ and
$\|\theta_{j_{i}} - \theta_{j_{i+1}}\| < \epsilon$, and
$\|\theta_{j_q}- w_j\| <\epsilon$ for some $w_j$. So, there exists $j$, such that $\|\theta_1 - w_j\| < k\epsilon$. 

Otherwise, notice that for each $\theta_i \in S$, $\|w_j - \theta_i\|\geq \epsilon$ for all $j$, and $\|\theta_i - \theta_j\| \geq \epsilon$ for all $\theta_j\not \in S$. WLOG, let $S = \{\theta_1,\dots,\theta_l\}$. 
  
We consider changing all
$\theta_1, \ldots, \theta_{l}$ by the same $v$ and define 
%
\[H({\bf a}, v) = G({\bf a},\theta_1+v,...,\theta_l+v, \theta_{l+1}
\ldots, \theta_k).\]

The optimality conditions on ${\bf a}$ are 
\begin{align*}
   \abs{\pd{H}{a_i}} & = \lvert 4a_i  + 2\sum_{j\neq i} a_j \Phi_\epsilon(\theta_i,\theta_j) + 2\sum_{j=1}^k b_j \Phi_\epsilon(\theta_i,w_j) \rvert \leq \delta
\end{align*}
%
Next, since $\Phi_\epsilon(r)$ is $\lambda$-harmonic for $r \geq \epsilon$, we may calculate the Laplacian of $H$ as
%
\begin{align*}
\Delta_v H & = \sum_{i=1}^l \lambda \left(2\sum_{j=1}^k a_ib_j
  \Phi_\epsilon(\theta_i, w_j) + 2\sum_{j=l+1}^k a_ia_j
  \Phi_\epsilon(\theta_i, \theta_j)\right) \\
& \leq \sum_{i=1}^l \lambda \left(-4 a_i^2 - 2
  \sum_{j = 1, j\neq i}^l  a_ia_j \Phi_\epsilon(\theta_i,\theta_j)\right)+ \delta \sum_{i=1}^l \lambda |a_i| \\
&= -2\lambda\expt\left[\left( \sum_{i=1}^l a_i \sigma(\theta_i,X)\right)^2\right] -2\lambda \sum_{i=1}^l a_i^2+ \delta\lambda\sum_{i=1}^l  |a_i| \\
\end{align*} 
%
The second line follows from our optimality conditions and the third line follows from completing the square. Since $\boldsymbol{(a,\theta)} \in \mathcal{M}_{G,\delta}$, we have $\Delta_v H \geq - 2kd\delta$. Let $S = \sum_{i=1}^l a_i^2$. Then, by Cauchy-Schwarz, we have $-2 \lambda S + \delta\lambda\sqrt{k} \sqrt{S} \geq -2kd\delta$. When $S \geq \delta^2 k$, we see that $-\lambda S \geq -2 \lambda S + \delta\lambda \sqrt{k}\sqrt{S} \geq -2kd\delta$. Therefore, $S \leq 2kd\delta/\lambda$.
 
We conclude that $S \leq \max(\delta^2k, 2kd\delta/\lambda) \leq 2kd\delta/\lambda$ since $\delta\leq 1 \leq 2d/\lambda$ and $\lambda = 1$. Therefore, $a_i^2 \leq 2kd\delta$.

\end{proof}

\AlmostHarmRes*

 \begin{proof}
 If there does not exists $i, j$ such that
   $\|\theta_i - w_j\| <k\epsilon$, then by Lemma \ref{almostHarmConv}, this implies $a_i^2 < \delta^2/k^2$ for all $i$. Now, for a integrable
   function $f(x)$, $\| f\|_X = \sqrt{\expt_X[f(X)^2]}$ is a
   norm. Therefore, if $f(x) = \sum_i b_i \sigma(w_i,x)$ be our true
   target function, we conclude that by triangle inequality
\begin{align*}
\sqrt{G(\boldsymbol{a,\theta})}  & \geq \norm{\sum_{i=1}^k a_i \sigma(\theta_i,x) - f(x)}_X \\
&\geq \|f(x)\|_X\ - \sum_{i=1}^k \|a_i\sigma(\theta_i,x) \|_X \\
& \geq
  \sqrt{G(\boldsymbol{0,0})} - \delta
\end{align*}
This gives a contradiction, so we conclude that there must exist $i, j$ such that $\theta_i$ is in a $k\epsilon$ neighborhood of $w_j$.
 \end{proof}
 
\AlmostHarmInitialize*

 \begin{proof}
  Consider choosing $\theta_1 = {\bf 0}$ and then
  optimizing $a_1$. Given $\theta_1$, the loss decrease is:
%
\begin{align*}
   G(a_1,{\bf 0}) - G({\bf 0},{\bf 0}) & = \min_{a_1} 2a_1^2 +
  2\sum_{j=1}^k a_1 b_j\Phi_\epsilon({\bf 0},w_j) \\
 %
 & = -\frac{1}{2}\left(  \sum_{j=1}^k b_j
   \Phi_\epsilon({\bf 0},w_j)\right)^2 
\end{align*}

Because $w_j$ are random Gaussians with variance $O(d \log d)$, we have $\|w_j\| \leq O(d\log d)$ with high probability for all $j$. By Lemma~\ref{almostHarmReal}, our potential satisfies ${\Phi}_\epsilon({\bf 0}, w_j) \geq (d/\epsilon)^{ - O(d)}$. And since $b_j$ are uniformly chosen in $[-1,1]$, we conclude that with high probability over the choices of $b_j$, $-\frac{1}{2}\left(  \sum_{j=1}^k b_j\Phi(\theta_1,w_j)\right)^2 \geq (d/\epsilon)^{ - O(d)}$ by appealing to Chebyshev's inequality for independent uniform variables.

Therefore, we conclude that with high probability, $G(a_1, {\bf 0}) \leq G(\boldsymbol{0,0}) - \frac{1}{2}(d/\epsilon)^{ - O(d)}$. Let $\sqrt{G(a_1, {\bf 0})} = \sqrt{G(\boldsymbol{0,0})} - \Delta \geq 0$. Squaring and rearranging gives $\Delta \geq \frac{1}{4\sqrt{G({\bf 0, 0})}}(d/\epsilon)^{ - O(d)}$. Since $G(\boldsymbol{0,0}) \leq O(k) = O(\poly(d))$, we are done. 

%%Cramer-Rao type Argument
%Let $f(x) =  \sum_{j=1}^k b_j \Phi(x,w_j)$, then it suffices (***Justify later) to show that $\Var(f(X)) \geq 1/\poly(d) $, where $X$ is a multivariate standard normal. By appealing to Cramer-Rao type bounds given in \cite{cacoullos1982upper}, we see that
%%
%\begin{align*}
% \Var(f(X)) &\geq \frac{1}{d}\expt \left[\sum_{i=1}^d \frac{\partial}{\partial x_i} f(X)\right]^2 \\
% &= \frac{1}{d}\left( \sum_{j=1}^k b_j \sum_{i=1}^d \expt \frac{\partial}{\partial x_i}\Phi(X,w_j) \right)^2 \\
%\end{align*}
%
%By Stein's identity, $\expt[\frac{\partial}{\partial x_i}\Phi(X,w_j)] = \expt[X_i\Phi(X,w_j)]$. And since $\Phi$ is translationally symmetric, if we let $w_{ji}$ be the $i$-th coordinate of $w_j$, then $\expt[ X_i\Phi(X,w_j)] = \expt[(X_i - w_{ji})\Phi(X, {\bf 0})] = -w_{ji}\expt[\Phi(X,{\bf 0})] = -Cw_{ji}$, where $C$ is a positive constant and $C\geq 1/\poly(d)$ (justify later).
%
%Therefore, $\Var(f(X)) \geq \frac{C^2}{d} \left(\sum_{j=1}^k b_j \sum_{i=1}^d w_{ji}\right)^2$. However, note that $w_{ji}$ are independent Gaussians of variance 1, so we conclude that $\sum_{j=1}^k b_j \sum_{i=1}^d w_{ji}$ is a Gaussian of variance $d\|{\bf b}\|^2 \geq d$. So, with high probability over $w_1,...,w_j$, we conclude that $\Var(f(X)) \geq C^2 \geq 1/\poly(d)$.
\end{proof}


%%%%%%%%%%%Begin Node Convergence
\subsection{Node by Node Analysis}

\begin{lemma}\label{nodeConv}
Let $\mathcal{M} = \R^d$ for $d \equiv 3 \mod 4$ and let $L_1$ be the loss restricted to $(a_1,\theta_1)$ corresponding to the activation function $\sigma_\epsilon$ given by Lemma~\ref{almostHarmReal} with $\lambda = 1$. For any $\epsilon \in (0,1)$ and $\delta \in (0, 2d/\lambda)$, we can construct $\sigma_\epsilon$ such that if $\boldsymbol{(a_1,\theta_1)} \in \mathcal{M}_{L_1,\delta}$, then for all $i$, either 1) there exists $j$ such that $\|\theta_1 - w_j\| < \epsilon$ or 2) $a_1^2 < 2d\delta$.
\end{lemma}

\begin{proof}
The proof is similar to Lemma~\ref{almostHarmConv}. Let $\Phi_\epsilon$ be the realizable potential in \ref{almostHarmReal} such that $\Phi_\epsilon(r)$ is $\lambda$-harmonic when $r \geq \epsilon$. Note that $\Phi_\epsilon(0) = 1$ is normalized. And let $(a_1,\theta_1) \in \mathcal{M}_{L,\delta}$. Assume that there does exist $w_j$ such that $\|\theta_1 - w_j\| < \epsilon$. 
 
The optimality conditions on ${ a_1}$ are 
\begin{align*}
   \abs{\pd{L}{a_1}} & = \lvert 2a_1  + 2\sum_{j=1}^k b_j \Phi_\epsilon(\theta_1,w_j) \rvert \leq \delta
\end{align*}
%
Next, since $\Phi_\epsilon(r)$ is $\lambda$-harmonic for $r \geq \epsilon$, we may calculate the Laplacian of $L$ as
%
\begin{align*}
\Delta_{\theta_1} L & = \lambda \left(2\sum_{j=1}^k a_1b_j
  \Phi_\epsilon(\theta_1, w_j) \right) \\
& \leq  -2\lambda a_1^2 + \delta \lambda |a_1| \\
\end{align*} 
%
The second line follows from our optimality conditions. Since ${(a_1,\theta_1)} \in \mathcal{M}_{L,\delta}$, we have $\Delta_{\theta_1} L \geq - 2d\delta$. When $a_1^2 \geq \delta^2$, we see that $-\lambda a_1^2 \geq -2 \lambda a_1^2 + \delta\lambda |a_1| \geq -2d\delta$. Therefore, $a_1^2 \leq 2d\delta/\lambda$. We conclude that $a_1^2 \leq \max(\delta^2, 2d\delta/\lambda) \leq 2d\delta/\lambda$ for $\delta\leq 2d \leq 2d/\lambda$ since $\lambda = 1$. Therefore, $a_1^2 \leq 2d\delta$.
\end{proof}

The charges are big if we made progress
%
\begin{lemma}\label{nodeRes}
  Assume the conditions of Lemma. If
$\sqrt{L_1(a_1,\theta_1)} \leq \sqrt{L_1(0, 0)} - \delta$
  and $(a_1,\theta_1) \in \mathcal{M}_{G,\lambda \delta^2/(2d)}$,
  then there exists some $j$ such that $\|\theta_1 - w_j\| <\epsilon$.
\end{lemma}

\begin{proof}
The proof follows similarly from Lemma \ref{almostHarmRes}.
\end{proof}
 %
 Next, we guarantee progress. We first prove a lemma about gradients of the potential $\Phi_\epsilon$.

\begin{lemma}\label{nodeGradient}
Assume the conditions of Theorem~\ref{nodewiseSGD} and Lemma~\ref{almostHarmConv}. If $\|\theta_1 - w_j\| \leq d$ and $|b_j|\geq 1/\poly(d)$ and $|a_1 - a_1^*(\theta_1)| \leq (d/\epsilon)^{-O(d)}$ is almost optimal, then $-\nabla_{\theta_1}L_1 = \zeta \frac{w_j - \theta_1}{\|\theta_1 - w_j\|} + \xi$ and $\zeta \geq  \frac{1}{\poly(d)}(d/\epsilon)^{-8d}$ and $\xi \leq (d/\epsilon)^{-O(d)}$. 
\end{lemma}

\begin{proof}
Through the proof, we assume $k = \poly(d)$. Now, our gradient with respect to $\theta_1$ is
%
\begin{align*}
\nabla_{\theta_1} L_1 &= 2a_1b_j \Phi_\epsilon'(\|\theta_1 - w_j\|) \frac{\theta_1 - w_j}{\|\theta_1 - w_j\|}+ 2\sum_{i\neq j} a_1b_i\Phi_\epsilon'(\|\theta_1 - w_i\|) \frac{\theta_1 - w_i}{\|\theta_1 - w_i\|}
\end{align*}
%

Since $\|\theta_1 - w_j\| \leq d$, we may lower bound $|\Phi_\epsilon'(\|\theta_1 - w_j\|)| \geq e^{-\sqrt{\lambda}d}d^{1-d}(2d+\sqrt{\lambda})^{-2d}\epsilon^{2d}/3 \geq O((d/\epsilon)^{-4d})$. Similarly, $\Phi_\epsilon(\|\theta_1 - w_j\|) \geq O((d/\epsilon)^{-4d})$. On the other hand, for all $i \neq j$, we note that with high probability $\|w_i - w_j\| \geq \Omega(d \log d)$. Therefore, we may upper bound $|\Phi_{\epsilon}(\|\theta_1 - w_i\|)| \leq (d/\epsilon)^{-O(d)}$. 

Together, we conclude that $\nabla_{\theta_1} L_1 = 2a_1b_j \Phi_\epsilon'(\|\theta_1 - w_j\|) \frac{\theta_1 - w_j}{\|\theta_1 - w_j\|} + 2a_1\xi$, where $\|\xi\| \leq (d/\epsilon)^{-O(d)}$.

And since $|a_1 - a_1^*(\theta_1)| \leq (d/\epsilon)^{-O(d)}$, we know that
%
\begin{align*}
   \abs{\pd{L_1}{a_1}} & = \lvert 2a_1 + 2b_j \Phi_\epsilon(\|\theta_1 - w_j\|) + 2\sum_{i \neq j} b_i \Phi_\epsilon(\|\theta_1 - w_i\|) \rvert \leq (d/\epsilon)^{-O(d)}
\end{align*}

By a similar argument as on the derivative, we see that $a_1 = -2b_j \Phi_\epsilon(\|\theta_1 - w_j\|) + O(d/\epsilon)^{-O(d)}$. Therefore, the direction of $-\nabla_{\theta_1} L_1$ is moving $\theta_1$ closer to $w_j$ since 
%
\begin{align*}
-\nabla_{\theta_1} L_1 &=  b_j^2\Phi_\epsilon(\|\theta_1-w_j\|)\Phi_\epsilon'(\|\theta_1 - w_j\|) \frac{\theta_1 - w_j}{\|\theta_1 - w_j\|} + (d/\epsilon)^{-O(d)} 
\end{align*}

and $\Phi_\epsilon > 0$ and $\Phi_\epsilon' < 0$. Finally, with high probability, $-b_j^2\Phi_\epsilon(\|\theta_1-w_j\|)\Phi_\epsilon'(\|\theta_1 - w_j\|) \geq 1/\poly(d)(d/\epsilon)^{-8d}$.
\end{proof}

 
 %
 \begin{lemma}[Initialization]\label{nodeInitialize}
Assume the conditions of Theorem~\ref{nodewiseSGD} and Lemma. With high probability, we can initialize $(a_1^{(0)},\theta_1^{(0)})$ such that $\sqrt{L(a_1^{(0)},\theta_1^{(0)})} \leq \sqrt{L({0,0})} -\delta$ with $\delta = \frac{1}{\poly(d)}(d/\epsilon)^{ -18d}$ in time $\log(d)^{O(d)}$.
 \end{lemma}

\begin{proof}
By our conditions, there must exist some $|b_j|$ such that $|b_j| \geq 1/\poly(d)$. Note that if we randomly sample points in a ball of radius $O(d\log d)$, we will land in a $d$-neighborhood of $w_j$ with probability $\log(d)^{-O(d)}$ since $\|w_j\|\leq O(d\log d)$ with high probability. 

Let $\theta_1$ be such that $\|\theta_1 - w_j \| \leq d$ and then we can solve for $a_1 = a_1^*(\theta_1)$ since we are simply minimizing a quadratic in one variable. Then, by Lemma~\ref{nodeGradient}, we see that $\|\nabla_{\theta_1}L_1 \| \geq 1/\poly(d)(d/\epsilon)^{-8d}$. Finally, by Lemma~\ref{almostHarmReal}, we know that the Hessian is bounded by $\poly(d)(d/\epsilon)^{2d }$. So, by Lemma~\ref{GradDecrease}, we conclude by taking a stepsize of $\alpha = \frac{1}{\poly(d)}(d/\epsilon)^{-2d}$ to reach $(a_1',\theta_1')$, we can guarantee that $L_1(a_1', \theta_1') \leq L_1 (a_1^*(\theta_1), \theta_1) - \frac{1}{\poly(d)} (d/\epsilon)^{ -18d}$.

But since $L_1(a_1^*(\theta_1),\theta_1)\leq L_1(0,0)$, we conclude that $L_1(a_1', \theta_1') \leq L_1(0,0) -  \frac{1}{\poly(d)} (d/\epsilon)^{ -18d}$. Let $\sqrt{L_1(a_1', \theta_1')} = \sqrt{L_1(0,0)} - \Delta \geq 0$. Squaring and rearranging gives $\Delta \geq \frac{1}{4\sqrt{L_1( 0, 0)}}  \frac{1}{\poly(d)}(d/\epsilon)^{ -18d} $. Since $L_1(0,0) \leq O(k) = O(\poly(d))$, we are done. 

\end{proof}
%
\begin{lemma}\label{nodewiseSGD}
Let $\mathcal{M} = \R^{d}$ and $d\equiv 3 \mod 4$. For all $\epsilon \in (0,1),$ there exists an activation $\sigma_\epsilon$ such that if $w_1,...,w_k \in \R^d$ with $w_i$ randomly chosen from $w_i \sim  \mathcal{N}({\bf 0}, O(d\log d){\bf I_{d\times d}})$ and $b_1,...,b_k$ are any numbers in $[-1,1]$ and there exists some $|b_i| \geq 1/\poly(d)$, then with high probability, we can choose an initial point $(a_1^{(0)}, \theta_1^{(0)})$ such that after running SecondGD (Algorithm \ref{SecondGD}) on the restricted regularized objective $L_1(a_1,\theta_1)$ for at most $(d/\epsilon)^{O(d)}$ iterations, there exists some $w_j$ such that $\|\theta - w_j\| < \epsilon$. Furthermore, if $|b_j| \geq 1/\poly(d)$, then $\|\theta - w_j\| < (d/\epsilon)^{-O(d)}$ and $|a + b_j| < (d/\epsilon)^{-O(d)}$.
\end{lemma}


\begin{proof}
Let our potential $\Phi_{\epsilon/k}$ be the one as constructed in Lemma~\ref{almostHarmReal} that is $\lambda$-harmonic for all $r \geq \epsilon$ with $\lambda = 1$ and as always, $k = \poly(d)$. First, by Lemma~\ref{nodeInitialize},  we can initialize ${(a^{(0)},\theta^{(0)})}$ such that $L_1({a^{(0)},\theta^{(0)}}) \leq  L_1({\bf 0,0}) - 2\delta \sqrt{L_1({\bf 0,0})}$ for $ \delta = \frac{1}{\poly(d)}(d/\epsilon)^{ -18d}$. If we set $\alpha = (d/\epsilon)^{-O(d)}$ and $\eta = \gamma = \lambda \delta^2/(2d)$,  then running Algorithm~\ref{SecondGD} will terminate and return some $(a,\theta)$ in at most $(d/\epsilon)^{O(d)}$ iterations. This is because our algorithm ensures that our objective function decreases by at least $\min(\alpha \eta^2/2, \alpha^2\gamma^3/2)$ at each iteration and $G({\bf 0, 0})$ is bounded by $O(k)$ and $G \geq 0$ is non-negative.

Assume there does not exist $w_j$ such that $\|\theta - w_j\| < (d/\epsilon)^{-O(d)}$. Then, we claim that $(a,\theta) \in \mathcal{M}_{L,\lambda \delta^2/(2d)}$. For the sake of contradiction, assume otherwise. By our algorithm termination conditions, if $(a,\theta) \not\in \mathcal{M}_{L,\lambda \delta^2/(2d)}$, then it must be that after one step of gradient or Hessian descent from $(a,\theta)$, we reach some $(a',\theta')$ and $L(\boldsymbol{a',\theta'}) \geq G(\boldsymbol{a,\theta}) - \min(\alpha\eta^2/2,\alpha^2\gamma^3/2)$. Now, Lemma~\ref{almostHarmReal} ensures all first three derivatives of $\Phi$ are bounded by $(d/\epsilon)^{2d}$, except at $w_1,...,w_k$. Furthermore, since there does not exists $w_j$ such that $\|\theta - w_j\| < (d/\epsilon)^{-O(d)}$, $G$ is three-times continuously differentiable within a $\alpha (d/\epsilon)^{2d} = (d/\epsilon)^{-O(d)}$ neighborhood of $\boldsymbol{\theta}$. Therefore, by Lemma~\ref{GradDecrease} and ~\ref{HessianDecrease}, we know that $L(a',\theta') \leq L(a,\theta) - \min(\alpha\eta^2/2,\alpha^2\gamma^3/2)$, a contradiction. 

So, it must be $(a,\theta) \in \mathcal{M}_{L,\lambda \delta^2/(2d)}$. Since our algorithm maintains that our objective function is decreasing, so $\sqrt{L(\boldsymbol{a,\theta})} \leq \sqrt{L({\bf 0,0})} - \delta $. So, by Lemma~\ref{nodeRes}, there must be some $w_j$ such that $\|\theta- w_j\|\leq \epsilon$.

Now, if $|b_j| \geq 1/\poly(d)$, then since $(a,\theta) \in \mathcal{M}_{L,\lambda \delta^2/(2d)}$, by Lemma~\ref{nodeGradient}, since $\|\theta - w_j \| \leq \epsilon$, we see that $\|\nabla_{\theta_1}L_1 \| \geq 1/\poly(d)(d/\epsilon)^{-(6+2\sqrt{\lambda})d} > \delta^2/(2d)$, a contradiction. Therefore, we must conclude that our original assumption was false and $\|\theta - w_j\| < (d/\epsilon)^{-O(d)}$ for some $w_j$.

Finally, we see that the charges also converge since $a = -2b_j \Phi_\epsilon(\|\theta - w_j\|) + O(d/\epsilon)^{-O(d)}$ and $\|\theta - w_j\| = (d/\epsilon)^{-O(d)}$. By noting that $\Phi_\epsilon(0) = 1$ and $\Phi_\epsilon$ is $(d/\epsilon)^{2d}$-Lipschitz, we conclude. 
\end{proof}

Finally, we have our final theorem.

\begin{theorem}\label{nodewiseSGD}
Let $\mathcal{M} = \R^{d}$ and $d \equiv 3 \mod 4$ and let $L$ be as in \ref{errLoss}. For all $\epsilon \in (0,1),$ there exists an activation $\sigma_\epsilon$ such that if $w_1,...,w_k \in \R^d$ with $w_i$ randomly chosen from $w_i \sim  \mathcal{N}({\bf 0}, O(d\log k){\bf I_{d\times d}})$ and $b_1,...,b_k$ be randomly chosen at uniform from $[-1,1]$, then with high probability, after running nodewise SGD (Algorithm \ref{NodeGDOpt}) on the objective $L$ for at most $\poly(1/\beta,1/\lambda)d^{\sqrt{\lambda}}(d/\epsilon)^{O(d)}$ iterations, $\boldsymbol{(a,\theta)}$ is in a $(d/\epsilon)^{-O(d)}$ neighborhood of the global minima.
\end{theorem}

\begin{proof}
Let $(a_i, \theta_i)$ be the $i$-th node that is initialized and applied noisy gradient descent onto. We want to show that the nodes $(a_i, \theta_i)$ will converge, in a node-wise fashion, to some permutation of $\{(b_1,w_1),...,(b_k,w_k)\}$. By Lemma~\ref{nodewiseSGD}, we know that with high probability $(a_1,\theta_1)$ will converge to some $(d/\epsilon)^{-O(d)}$ neighborhood of $(b_{\pi(1)}, w_{\pi(1)})$ for some function $\pi: [k] \to [k]$. Now, since [condition holds], by Lemma~\ref{almostHarmInitialize}, we can initialize ${(a_2^{(0)},\theta_2^{(0)})}$ such that $L_2({a_2^{(0)},\theta_2^{(0)}}) \leq  L_2({\bf 0,0}) - 2\delta \sqrt{L_2({\bf 0,0})}$ for $\delta = d^{-O(d\sqrt{\lambda})}(d/\epsilon)^{-O(d)}$. Then, by Lemma~\ref{nodewiseSGD}, we know that if $x_i = (a_2^{(i)}, \theta_2^{(i)}$ are the noisy GD iterates, then there exists a minimal $l$ such that $x_l$ is in a $(d/\epsilon)^{-O(d)}$ neighborhood of $w_j$ for some $j$. [Needs to be argued further] By Azuma's inequality, $L_2({a_2^{(l)},\theta_2^{(l)}}) \leq  L_2({\bf 0,0}) - \delta \sqrt{L_2({\bf 0,0})}$, with high probability. 

Now, we claim that $x_l$ is not in a $(d/\epsilon)^{-O(d)}$ neighborhood of $(b_{\pi(1)}, w_{\pi(1)})$. Assume otherwise. First, we see that our decrease in objective function implies that $L_{a_1,\theta_1,a_2,\theta_2}(a_1,\theta_1, a_2^{(l)},\theta_2^{(l)}) \leq L_{a_1,\theta_1,a_2,\theta_2}(a_1,\theta_1, {\bf 0,0}) - \delta \sqrt{L_2({\bf 0,0})}$. However, notice that our assumption implies that $a_2^{(l)},\theta_2^{(l)}$ is in a $(d/\epsilon)^{-O(d)}$ neighborhood of $a_1,\theta_1$. By the smoothness of $L$, we see that $L_{a_1,\theta_1,a_2,\theta_2}(a_1,\theta_1, a_2^{(l)},\theta_2^{(l)}) \geq L_{a_1,\theta_1,a_2,\theta_2}(a_1,\theta_1, a_2^{(l)},\theta_1) - (d/\epsilon)^{-O(d)} = L_{a_1,\theta_1,a_2,\theta_2}(a_1 + a_2^{(l)},\theta_1, {\bf 0, 0}) - (d/\epsilon)^{-O(d)} $. Together, we know that  $L_{a_1,\theta_1,a_2,\theta_2}(a_1+a_2^{(l)},\theta_1,{\bf 0, 0}) \leq L_{a_1,\theta_1,a_2,\theta_2}(a_1,\theta_1, {\bf 0,0}) - (d/\epsilon)^{-O(d)}$

However, by Theorem \ref{quadConverge}, our gradient descent guarantees that $L_{a_1,\theta_1,a_2,\theta_2}(a_1,\theta_1, {\bf 0,0}) - L_{a_1,\theta_1,a_2,\theta_2}(a_1 + a_2^{(l)},\theta_1, {\bf 0, 0}) \leq O(1/T)$, where $T$ is the number of iterations of gradient descent. Since $T = (d/\epsilon)^{-O(d)}$ is large enough, we derive a contradiction. Therefore, our claim is done and by induction, $\pi$ is a permutation. Now, our theorem follows. 
\end{proof}

%%% Local Variables:
%%% mode: latex
%%% TeX-master: "icmlpaper2017.tex"
%%% End:

\subsection{Convergence of Sign Activation}

\signcon*

\begin{proof}
We will first argue that unless all the electrons and protons have matched up as a permutation it cannot be a strict local minimum and then argue that the global minimum is a strict local minimum.

First note that if some electron and proton have merged, we can remove such pairs and argue about the remaining configuration of charges. So WLOG we assume there are no such overlapping electron and proton.

First consider the case when there is an isolated electron $e$ and there is no charge diagonally opposite to it. In this case look at the two semicircles on the left and the right half of the circle around the isolated electron -- let $q_1$ and $q_2$ be the net charges in the left and the right semi-circles. Note that $q_1 \neq q_2$ since they are integers and $q_1 + q_2 = +1$ which is odd. So by moving the electron slightly to the side with the larger charge you decrease the potential.

If there is a proton opposite the isolated electron the argument becomes simpler as the proton benefits the motion of the electron in either the left or right direction. So  the only way the electron does not benefit by moving in either direction is that $q_1 = -1$ and $q_2 = -1$ which is impossible.

If there is an electron opposite the isolated electron then the combination of these two diagonally opposing electrons have a zero effect on every other charge. So it is possible rotate this pair jointly keeping them opposed in any way and not change the potential. So this is not a strict local minimum.

Next if there is a clump of isolated electrons with no charge on the diagonally opposite point then again as before if $q_1 \neq q_2$ we are done. If  $q_1 = q_2$ then the the electrons in the clump locally are unaffected by the remaining charges. So now by splitting the clump into two groups and moving them apart infinitesimally we will decrease the potential.

Now if there is only protons in the diagonally opposite position an isolated electron again we are done as in the case when there is one electron diagonally opposite one proton. 

Finally if there is only electrons diagonally opposite a clump of electrons again we are done as we have found at least one pair of opposing electrons that can be jointly rotated in any way.

Next we will argue that a permutation matching up is a strict local minumum. For this we will assume that no two protons are diagonally opposite each other (as they can be removed without affecting the function). Now given a perfect matching up of electrons and protons, if we perturb the electrons in any way infinitesimally, then any isolated clump of electrons can be moved slightly to  the left or right to improve the potential.
\end{proof}

\signconv*

\begin{proof}
In $S^1$, notice that the pairwise potential function is $\Phi(\theta,w) = 1 - 2\cos^{-1}(\theta^Tw)/\pi = 1 - 2\alpha/\pi$, where $\alpha$ is the angle between $\theta, w$. So, let us parameterize in polar coordinates, calling our true parameters as $\widetilde{w_1},...,\widetilde{w_k} \in [0,2\pi]$ and rewriting our loss as a function of $\widetilde{\theta} \in [0,2\pi]$. 

Since $\Phi$ is a linear function of the angle between $\theta, w_j$, each $w_j$ exerts a constant gradient on $\widetilde{\theta}$ towards $\widetilde{w_j}$, with discontinuities at $\widetilde{w_j},\pi+\widetilde{w_j}$. Almost surely over $b_1,..,b_k$, the gradient is non-zero almost everywhere, except at the discontinuities, which are at $\widetilde{w_j}, \pi+\widetilde{w_j}$ for some $j$. 
\end{proof}


\subsection{Convergence of Polynomial Potentials}

\polystrict*

\begin{proof}
WLOG, we can consider $w_1,...,w_d$ to be the basis vectors $e_1,...,e_d$. Note that this is a manifold optimization problem, so our optimality conditions are given by introducing a Lagrange multiplier $\lambda$, as in \cite{GeHJY15}.
\[\pd{L}{a} = 2\sum_{i=1}^d ab_i (\theta_i)^l + 2a = 0\]
\[ (\nabla_\theta L)_i = 2ab_il(\theta_i)^{l-1}  -2\lambda \theta_i = 0 \]
where $\lambda$ is chosen that minimizes 
\[\lambda = \arg \min_\lambda \sum_i (ab_i l (\theta_i)^{l-1} - \lambda\theta_i)^2 = \sum ab_i l (\theta_i)^l \]
Therefore, either $\theta_i = 0$ or $b_i (\theta_i)^{l-2} = \lambda/(al)$. From \cite{GeHJY15}, we consider the constrained Hessian, which is a diagonal matrix with diagonal entry: 
\[(\nabla^2 L)_{ii} = 2a b_i l(l-1)(\theta_i)^{l-2} - 2 \lambda\]
Assume that there exists $\theta_i, \theta_j \neq 0$, then we claim that $\theta$ is not a local minima. First, our optimality conditions imply $b_i(\theta_i)^{l-2} = b_j (\theta_j)^{l-2} = \lambda/(al)$. So,
\[(\nabla^2 L)_{ii} = (\nabla^2L)_{jj} = 2a b_i l(l-1)(\theta_i)^{l-2} - 2 \lambda\]
\[ = 2(l-2)\lambda = -2(l-2)la^2\]
Now, there must exist a vector $v \in S^{d-1}$ such that $v_k = 0$ for $k \neq i,j$ and $v^T\theta = 0$, so $v$ is in the tangent space at $\theta$. Finally, $v^T(\nabla^2 L) v  = -2(l-2)l a^2 < 0$, implying $\theta$ is not a local minima when $a \neq 0$. Note that $a = 0$ occurs with probability 0 since our objective function is non-increasing throughout the gradient descent algorithm and is almost surely initialized to be negative with $a$ optimized upon initialization, as by observed before.
\end{proof}

When the output weights are variable, notice that our convergence results often rely on the optimality of the output weights. For this reason, we will optimize the output weights at every gradient descent step, by carefully choosing the stepsize. As our loss function \eqref{errLoss} is quadratic in $a$, we know that gradient descent will find the optimal $a$ efficiently. Under a node-wise descent algorithm, we can show efficient convergence to global minima under orthogonality assumptions on $w_j$ for these polynomial activations/potentials.

\Qnote{Will be rewritten!!}
\begin{restatable}{theorem}{polyConv}
\label{PolyConv}
Let $\mathcal{M} = S^{d-1}$. Let $w_1,...,w_d$ be orthonormal vectors in $\R^d$ and $\Phi$ is of the form $\Phi(\theta,w) = (\theta^Tw)^l$ for some fixed integer $l \geq 3$. Furthermore, $1 \leq |b_i|\leq \poly(d)$. 

Then, with high probability, running Algorithm \ref{NodeGDOpt} on \eqref{errLoss} converges to an $\epsilon$-neighborhood of the global minima in $\poly(d,1/\epsilon)$ time. 
\end{restatable}


\begin{proof}
Without loss of generality let $w_1,...,w_k$ be the standard basis vectors $e_1,..,e_k$. and these basis vectors span the whole optimization space. Consider the algorithm on just the first node: $(a_1,\theta_1)$. For simplicity, we will drop the subscripts in the proof. 

First, consider the initialization of $\theta$, notice that upon initialization and optimization, if $\theta$ is drawn from a standard Gaussian, then by independence and $\expt[\theta_i] = 0$,
%
\[\expt[a(\theta)^2] = \expt\left[\left(\sum_{i=1}^k  b_i (\theta_i)^l\right)^2\right] = \sum_{i=1}^k b_i^2 \expt[\theta_i^{2l}]\]

There must exists $j$ such that $|b_j|\geq 1$ and since drawing $\theta_i$ uniformly on $S^{d-1}$ is just a $\poly(d)$ rescaling of a Gaussian, we conclude that $E[a^2]\geq1/\poly(d)$. By Theorem \ref{nonDecrease}, we conclude that $E[a^2] \geq 1/\poly(d)$ throughout the gradient descent algorithm. Next, to use Theorem \ref{strongConverge}, we check the regularity conditions. By assumption, we can choose $B, L, \rho$ to be $\poly(d)$ since $\Phi$ and the second and third partials of $\Phi$ are all bounded by $\poly(d)$ in $\Omega$. Furthermore, our activation function $\sigma$ and its derivatives are bounded in magnitude by $O(|x|^{l})$, where $l$ is fixed. By Theorem \ref{genErrBound}, with high probability, we can construct a stochastic oracle up to $\epsilon$ error with sample complexity $\poly(d,1/\epsilon)$.


Therefore, we conclude that we converge to $\theta \in \mathcal{M}_{\poly(\epsilon,1/d)}$ in $\poly(d,1/\epsilon)$ iterations. Note that this is a constrained optimization problem, so by introducing a Lagrange multiplier $\lambda$, the optimality conditions are:
%
\begin{align*}
|\pd{L}{a}| & = |2\sum_{i=1}^k b_i (\theta_i)^l + 2a| \leq \poly(\epsilon,1/d), \textrm{ and } \\
%
 |(\nabla_\theta L)_i| & = |\theta_i||2ab_il(\theta_i)^{l-2}  -2\lambda| \leq \poly(\epsilon,1/d) ,
\end{align*}
where $\lambda$ is chosen to minimize
\[\lambda = \arg \min_\lambda \sum_i (ab_i l (\theta_i)^{l-1} - \lambda\theta_i)^2 = \sum_i ab_i l (\theta_i)^l = -la^2 + \poly(\epsilon,1/d) \]

Therefore, either $|\theta_i| \leq \poly(\epsilon,1/d)$ or $|2ab_il(\theta_i)^{l-2} - 2 \lambda |\leq \poly(\epsilon,1/d)$. Next, the projected Hessian is a diagonal matrix with diagonal entry: 
%
\[(\nabla^2 L)_{ii} = 2a b_i l(l-1)(\theta_i)^{l-2} - 2 \lambda\]

Assume that there exists $\theta_i, \theta_j$ such that $|\theta_i|,|\theta_j| \geq \poly(\epsilon,1/d)$, then we conclude that the other inequality must hold for coordinates $i, j$. So,
%
\begin{align*}
(\nabla^2 L)_{ii} & = 2a b_i l(l-1)(\theta_i)^{l-2} - 2 \lambda \\
& \leq
  2(l-2)\lambda + (l-1)\epsilon \\
&  \leq -2(l-2)la^2 + \poly(\epsilon,1/d)
\end{align*}
%
Since $l-2 \geq 1$ and $a^2 \geq 1/\poly(d)$, we conclude that there
exists a vector with $v_k = 0$ for all $k\neq i, j$ such that
$v^T\theta = 0$ (in the tangent space) and
$v^T(\nabla^2 L) v = -2(l-2)l a^2 + \poly(\epsilon,1/d) <
-\poly(\epsilon,1/d)$.
This contradicts $\theta \in \mathcal{M}_{\poly(\epsilon,1/d)}$, so
$\theta$ is in some $\epsilon$ neighborhood of some
$w_j$. Furthermore, note that $|a_1| \leq \poly(d)$ and there exists
$b_j$ such that $|b_j| \geq 1$ as some $w_i$ has not yet been paired
with a $\theta$.

Now, we proceed with induction and repeat the same argument on $\theta_2$. We can simply treat $\theta_1$ as $w_{k+1}$ and so applying the same argument tells us that $\theta_2$ is close to some $w_j$ for some $j$. The issue is that $\theta_2$ could be in a $\poly(\epsilon,1/d)$-neighborhood of $w_{k+1} = \theta_1$ or $w_i$. We claim that this will not occur. First, since $w_i, w_{k+1}$ are in a $\poly(\epsilon,1/d)$-neighborhood of each other, we will assume WLOG that $\theta_2$ is close to $\theta_1$.

Now, by the optimality of $a_1$, we know that $L(a_1,\theta_1) \leq \min_{a \in \Omega} L(a,\theta_1) + O(\epsilon)$. We claim that if $\theta_2$ is close to $\theta_1$, then $L(a_1+a_2,\theta_1) \leq L(a_1,\theta_1) - \Omega(\epsilon)$. This, combined with the fact that $a_1 + a_2$ is bounded by $\poly(d)$, would lead to a contradiction.

First, notice that since $a_2$ is always optimal, we have
\begin{align*}
& L(a_1,\theta_1) - L(a_1,\theta_1,a_2,\theta_2) \\
& \qquad \qquad = a_2^2 + 2a_2 (\sum_{j < 2} \Phi(\theta_2,\theta_j) + \sum_{j=1}^k b_j \Phi(\theta_2,w_j)) \\
& \qquad \qquad = -a_2^2 = \Omega(\epsilon)
\end{align*}

Therefore, it suffices to show that $|L(a_1+a_2,\theta_1) - L(a_1,\theta_1,a_2,\theta_2) | \leq O(\epsilon)$. Since $L(a_1+a_2,\theta) = L(a_1,\theta_1,a_2,\theta_1)$, this follows immediately from the $\poly(d)$-Lipschitz of $\Phi$ and the fact that $\theta_2$ is in a $\poly(d,1/\epsilon)$-neighborhood of $\theta_1$. We conclude that $\theta_2$ cannot be in a $\poly(\epsilon,1/d)$-neighborhood of $\theta_1$ but converges to a point close to $w_{j}$, $j\neq i,k+1$. Therefore, the no two $\theta_i$ are matched to one $w_j$. 

By applying this logic to all $\theta_i$ through induction, we deduce that $\theta$ is within $\epsilon$ of the global minima.
\end{proof}



\section{Proof of Sign Uniqueness}
For the sign activation function, we can show a related result.
\begin{restatable}{theorem}{signUnique}
\label{SignUnique}
Let $\mathcal{M} = S^{d-1}$ and $\sigma$ be the sign activation function and $b_2,...,b_k = 0$. If the loss \eqref{errLoss} at $(\boldsymbol{a,\theta})$ is less than $O(1)$, then there must exist $\theta_i$ such that $w_1^T\theta_i > \Omega(1/\sqrt{k})$.
\end{restatable}
\begin{proof}
WLOG let $w_1 = e_1$. Notice that our loss can be bounded below by Jensen's:
%
\begin{align*}
& \expt_X \left[ \left( \sum_{i=1}^k a_i \sigma(\theta_i^TX) - \sigma(X_1)\right)^2 \right] \\
& \qquad 
\geq \expt_{X_1} \left[ \left( \EE{X_2...X_d}{\left[ \sum_{i=1}^k a_i \sigma(\theta_i^TX) \right]}- \sigma(X_1)\right)^2 \right],
\end{align*}
where $X$ is a standard Gaussian in $\R^d$. 
%
\begin{align*}
E_{X_2,..,X_d} \left[  \sum_{i=1}^k a_i \sigma(\theta_i^TX) \right] &= \sum_{i=1}^k a_i E_{X_2,...X_d}\left[  \sigma(\theta_{i1}X_1 + \sum_{j >1} \theta_{ij}X_{j})  \right]\\
&= \sum_{i=1}^k E_{Y} \left[   \sigma(\theta_{i1}X_1 + \sqrt{1-\theta_{i1}^2}Y)  \right]  \\
&= \sum_{i=1}^k a_i E_{Y} \left[   \sigma(\textstyle\frac{\theta_{i1}}{\sqrt{1-\theta_{i1}^2}}X_1 + Y)  \right] ,
\end{align*}
where $Y$ is an independent standard Gaussian and for any small $\delta$, if $p(y)$ is the standard Gaussian density, 
%
\[ E_Y[\sigma(\delta + Y)] = \int_{-\delta}^{\delta} p(y) \, dy = 2p(0)\delta + O(\delta^2) \]

If $w_1^T\theta_i = \theta_{i1} < \epsilon$ for all $i$, then notice that with high probability on $X_1$ (say condition on $|X_1| \leq 1$), 
%
\[\expt_{Y} \left[   \sigma(\textstyle\frac{\theta_{i1}}{\sqrt{1-\theta_{i1}^2}}X_1 + Y)  \right] = 2p(0)\textstyle\frac{\theta_{i1}}{\sqrt{1-\theta_{i1}^2}}X_1 + O(\epsilon^2X_1^2)\]

Therefore, since $\epsilon < O(1/\sqrt{k})$,
%
\begin{align*}
\expt_{X_2,..,X_d} \left[  \sum_{i=1}^k a_i \sigma(\theta_i^TX)
  \right]  & = X_1
  \sum_{i=1}^k2p(0)a_i\textstyle\frac{\theta_{i1}}{\sqrt{1-\theta_{i1}^2}}
  + O(k\epsilon^2X_1^2) \\
& = cX_1+O(1)
\end{align*}


Finally, our error bound is now
%
\begin{align*}
& \expt_{X_1} \left[ \left( \expt_{X_2...X_d}\left[ \sum_{i=1}^k a_i
      \sigma(\theta_i^TX) \right]- \sigma(X_1)\right)^2 \right] \\
& \qquad \geq
\expt_{|X_1| \leq 1}[(cX_1+O(1) - \sigma(X_1))^2]
\end{align*}

And the final expression is always larger than some constant, regardless of $c$.
\end{proof}
























%%%%%%%%%%%%%%%%%%%%Not in paper but could be useful
\if{1}

\subsection{Infinite Iteration Bounds} 
\label{InfIter}


\begin{theorem}\cite{lee2016gradient, PanageasP16}\label{convStrict}
  Let $f :\Omega \to \R$ be a twice differentiable function such that
  $\sup_{x \in \Omega} \|\nabla^2 f\| \leq L$. Let
  $\mathcal{S} \subseteq \Omega$ be the set of critical points of $f$
  that are not local minima. Also, if
  $g(x) = x - \frac{1}{2L} \nabla f(x)$, then
  $g(\Omega) \subseteq \Omega$.

  Then, running Algorithm \ref{GD} with gradient input $\nabla f$ and
  stepsize $\alpha = 1/(2L)$, as the iteration $T \to\infty$, will
  converge to a point $x_\infty$ outside of $S$ almost surely over
  randomly chosen initial points $x_0$.
\end{theorem}

\begin{corollary}
Assume all the assumptions of Theorem \ref{convStrict} and let $f$ admit a global minima in $\overline{\Omega}$. Assume all critical points of $f$ in $\Omega$ are not local minima, except at the global minima. Then, running Algorithm \ref{GD} with gradient input $\nabla f$ and stepsize $\alpha = 1/(2L)$ will converge to the global minima almost surely as the iteration count $T \to\infty$.
\end{corollary}
\fi


\if{1}
\begin{theorem}
For any $\epsilon < 1/\poly(d)$, we can construct a realizable potential $\Phi$ such that with high probability, running Algorithm \ref{NodeGDOpt} on \eqref{errLoss} with error $\delta = \poly(\epsilon,1/d)$, $\gamma = \epsilon$ and stepsize $\alpha = 1/\poly(d,1/\epsilon)$ converges in $T = \poly(d, 1/\epsilon)$ iterations to $(\boldsymbol{a,\theta})$ such that either  $\theta$ is within $\epsilon$-neighborhood of the global minima or there exists $i$ such that if $\theta_i$ is picked uniformly in $\mathcal{M}$
%
\[ \expt\left[\left( \sum_{j < i} a_j \Phi(\theta_i,\theta_j) + \sum_{j=1}^k b_j \Phi(\theta_i,w_j)\right)^2\right] < \epsilon\]

The sample complexity is $d^{O(\log(d)/\epsilon)}$.
\end{theorem}

\begin{proof}
Let $\Phi_m$ be the $(1,m)$-Harmonic potential in Theorem \ref{eigConv} with $m = \poly(d,1/\epsilon)$ and $m$ odd. We first consider the algorithm on node $\theta_1$ and claim that it will merge with some $w_j$ and then we will proceed with induction. 


If
%
\[ \expt\left[\left( \sum_{j < 1} a_j \Phi(\theta_i,\theta_j) +
    \sum_{j=1}^k b_j \Phi(\theta_i,w_j)\right)^2\right] < \epsilon,\]
then we are done. Otherwise, with high probability, Theorem
\ref{nonDecrease} allows us to deduce that throughout the SGD
algorithm applied on $\theta_1$, $a_1^2 = \Omega(\epsilon)$.

Now we want to apply Theorem \ref{strongConverge}, so we check the regularity conditions. Since $\mathcal{M} = S^{d-1}$, then we can choose $B, L, \rho$ to be $\poly(d)$ since $\Phi$ and the second and third partials of $\Phi$ are all bounded by $\poly(d)$. Furthermore, by our construction, our activation function $\sigma(x)$ and its derivatives are $O(|x|^{\poly(d,1/\epsilon)})$. By Theorem \ref{genErrBound}, with high probability, we can construct a stochastic oracle up to $\poly(\epsilon,1/d)$ error with sample complexity $d^{\poly(d,1/\epsilon)}$.


Therefore, by Theorem \ref{strongConverge} we conclude that we converge to $\theta_1 \in \mathcal{M}_{\poly(\epsilon,1/d)}$. By Theorem \ref{eigConv}, since $|a_i| = \Omega(\sqrt{\epsilon})$, this implies that it is in an $\poly(\epsilon,1/d)$-neighborhood of some $w_{i}$ in $\poly(d,1/\epsilon)$ time. Note that $\theta_1$ will close to with $\pm w_j$ for some $j$ but since $\Phi_m$ is odd, WLOG, it is close to $w_j$. 

Furthermore, note that $|a_1| \leq \poly(d)$ by using the explicit formula. And lastly, by Theorem \ref{quadConverge}, since the maximum eigenvalue of our matrix $A$ is bounded by \poly(d), our gradient descent steps on the quadratic loss $L_{a_1}$ will converge to the optimum in $\Omega = \{a \in \R^n | \|a\| \leq \poly(d)\}$ with $O(\epsilon)$ error in $T$ iterations.

Now, we proceed with induction and repeat the same argument on $\theta_2$. We can simply treat $\theta_1$ as $w_{k+1}$ and so applying the same argument tells us that $\theta_2$ is close to some $w_j$ for some $j$. The issue is that $\theta_2$ could be in a $\poly(\epsilon,1/d)$-neighborhood of $w_{k+1} = \theta_1$ or $w_i$. We claim that this will not occur. First, since $w_i, w_{k+1}$ are in a $\poly(\epsilon,1/d)$-neighborhood of each other, we will assume WLOG that $\theta_2$ is close to $\theta_1$.

Now, by the optimality of $a_1$, we know that $L(a_1,\theta_1) \leq \min_{a \in \Omega} L(a,\theta_1) + O(\epsilon)$. We claim that if $\theta_2$ is close to $\theta_1$, then $L(a_1+a_2,\theta_1) \leq L(a_1,\theta_1) - \Omega(\epsilon)$. This, combined with the fact that $a_1 + a_2$ is bounded by \poly(d), would lead to a contradiction.

First, notice that since $a_2$ is always optimal, we have
\begin{align*}
& L(a_1,\theta_1) - L(a_1,\theta_1,a_2,\theta_2) \\
& \qquad \qquad = a_2^2 + 2a_2 (\sum_{j < 2} \Phi(\theta_2,\theta_j) + \sum_{j=1}^k b_j \Phi(\theta_2,w_j)) \\
& \qquad \qquad = -a_2^2 = \Omega(\epsilon)
\end{align*}

Therefore, it suffices to show that $|L(a_1+a_2,\theta_1) - L(a_1,\theta_1,a_2,\theta_2) | \leq O(\epsilon)$. Since $L(a_1+a_2,\theta) = L(a_1,\theta_1,a_2,\theta_1)$, this follows immediately from the $\poly(d)$-Lipschitz of $\Phi$ and the fact that $\theta_2$ is in a $\poly(d,1/\epsilon)$-neighborhood of $\theta_1$. We conclude that $\theta_2$ cannot be in a $\poly(\epsilon,1/d)$-neighborhood of $\theta_1$ but converges to a point close to $w_{j}$, $j\neq i,k+1$. Therefore, the no two $\theta_i$ are matched to one $w_j$. 

By applying this logic to all $\theta_i$ through induction, we deduce that $\theta$ is within $\epsilon$ of the global minima.
\end{proof}
\fi

\end{document} 






% This document was modified from the file originally made available by
% Pat Langley and Andrea Danyluk for ICML-2K. This version was
% created by Lise Getoor and Tobias Scheffer, it was slightly modified  
% from the 2010 version by Thorsten Joachims & Johannes Fuernkranz, 
% slightly modified from the 2009 version by Kiri Wagstaff and 
% Sam Roweis's 2008 version, which is slightly modified from 
% Prasad Tadepalli's 2007 version which is a lightly 
% changed version of the previous year's version by Andrew Moore, 
% which was in turn edited from those of Kristian Kersting and 
% Codrina Lauth. Alex Smola contributed to the algorithmic style files.  
