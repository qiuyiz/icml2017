
\subsection {Related Work}



The main difficulty in analysis is the non-convexity of the loss objectives that deep learning present. Recent work has shown that SGD will efficiently converge to a local minimizer and escape saddle points, under modest assumptions \cite{GeHJY15}. Therefore, it suffices to analyze the local minima of the loss landscape and show that no spurious local minima exist. This direction has led to positive results in matrix sensing \cite{ParkKCS16a}, matrix completion \cite{GeLM16}, dictionary learning \cite{SunQW15}, phase retrieval \cite{SunQW16}, and orthogonal tensor decomposition \cite{GeHJY15}. 

A non-convex analysis of the deep learning has been largely elusive and more discouragingly, there has been hardness results for even simple networks. A neural network with one hidden unit and sigmoidal activation can admit exponentially many local minima \cite{Auer}. Backprogration has been proven to fail in a simple network due to the abundance of bad local minima \cite{brady1989back}. Training a 3-node neural network with one hidden layer is { NP}-complete \cite{BlumR88}.  But, these and many similar worst-case hardness results are based on worst case training data assumptions. However, by using a result in \cite{klivans2006cryptographic} that learning a neural network with threshold activation functions is equivalent to learning intersection of halfspaces, several authors showed that under certain cryptographic assumptions, depth-two neural networks are not efficiently learnable with smooth activation functions \cite{LivniSS14, ZhangLWJ15, ZhangLJ15}. 

Due to the difficulty of analysis, many have turned to improper learning and the study of non-gradient methods to train neural networks. Janzamin et. al use tensor decomposition methods to learn the shallow neural network weights, provided access to the score function of the training data distribution \cite{JanzaminSA15}. Eigenvector and tensor methods are also used to train shallow neural networks with quadratic activation functions in \cite{LivniSS14}. Combinatorial methods that exploit layerwise correlations in sparse networks have also been analyzed provably in \cite{AroraBGM13}. Kernel methods, ridge regression, and even boosting were explored for regularized neural networks with smooth activation functions in \cite{shalev2011learning, ZhangLWJ15, ZhangLJ15}. Non-smooth activation functions, such as the ReLU, can be approximated by polynomials and are also amenable to kernel methods\cite{GoelKKT16}. These methods require assumptions that lessen the relevance of these algorithms to practice and more importantly, come short of explaining the widespread success of simple SGD.

If the activation functions are linear or
if certain independence assumptions are made, Kawaguchi shows that the
only local minima are the global minima \cite{Kawaguchi16a}. Under the
spin-glass and other physical models, some have shown that the loss
landscape admit well-behaving local minima that occur usually when the
overall error is small
\cite{ChoromanskaHMAL14, DauphinPGCGB14}. When only training
error is considered, some have shown that a global minima can be
achieved if the neural network contains sufficiently many hidden nodes
\cite{SoudryC16}. Our research is largely inspired by
\cite{valiant2014learning}, in which the authors show that when the
target functions are polynomials, gradient descent on neural networks
with one hidden layer is shown to converge to low error, given a large
number of hidden nodes. And when complex perturbations are allowed,
there is no robust local minima.
