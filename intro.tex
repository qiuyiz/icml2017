\section{Introduction}

\subsection{Background}

Deep learning has resulted in major strides in industrial applications including speech recognition, image classification and ad-matching. The simple idea of using layers of nodes with a non-linear activation function at each node allows one to express any function.  And then to learn a certain target function we just use (stochastic) gradient descent to minimize the loss; this approach has resulted in significantly lower error rates for several real world functions. Naturally the question remains: how close are we to the optimal  values of the edge weight parameters? Are we stuck in some bad local minima? While there are serveral recent works \cite{} that have tried to study the presence of local minima the picture is far from clear.

In this work we will ask how well can deep learning learn certain types of synthetic functions. Simple examples of synthetic functions are polynomials \cite{}, decision tress. Here we will study this question for functions that can be represented by a deep network of a certain depth and width. If a target function is deep network with a certain shape and has randomly initialized edge weights then will gradient descent converge to the target function? Our experimental simulations show that for short depths (say 2) and with random edge weights the error rate drops to a small percentage.  To understand this better we theoretically investigate this question of convergence for networks of depth $2$ with certain simiplifying assumptions -- particularly we assume that the top node is a sum node instead of a non linear node and that the training data comes from a distribution. 

Our main contribution is that with these simplified depth $2$ networks the question of whether the gradient descent converges to the desired target function is equivalent to a certain question of convergence in electrodynamics: the electrodynamics question involved $n$ fixed protons in a certain $d$ dimensional space and $n$ electrons initialized at random positions with all the electrons moving under the influence of the electrical force of attraction from the protons and repulsion from the remaining electrons. The question of convergence then is whether at equilibrium all the electrons will be matched up all the protons upto some permutation. Here $n$ corresponds to the number of hidden units, $d$ is the number of inputs to the deep network, the positions of the protons correspond to the edge weights incident on each of the hidden units and the positions of the electrons are the random initilialized values of the edge weights for the hidden units at the beginning of the gradient descent. The motion of the electrons essentially tracks the gradient descent in deep learning. The force function between a pair of charges is given not by the standard electrical force of $1/r^2$ (where $r$ is the distance between unit charges) but by another function of the distance that depends on the activation function. Thus the question of convergence in these simplified depth $2$ networks can be resolved by studying the equivalent electrodynamics question with the corresponding force function. 

We show the force function for different possible activation functions. Further for a certain synthetic activation function, we prove that the electrodynamic force function results in convergence thus 
implying the desired covergence for simplified. depth 2 networks with those activation functions assuming the sample complexity is close to its infinite limit. We leave open the problem of convergence for forces corresponding to more realistic activation functions.

\subsection {Related Work}

Deep learning has been widely adopted to solve a variety of practical problems in artificial intelligence. Backpropagation, perhaps the most well-known algorithm in deep learning, applies stochastic gradient descent (SGD) to the squared loss to learn suitable parameters of a neural network that approximates some target function $f: \R^n \to \R$. However, are these discovered parameters provably good or even optimal? 

The main difficulty in analysis is the non-convexity of the loss objectives that deep learning present. Recent work has shown that SGD will efficiently converge to a local minimizer and escape saddle points, under modest assumptions \cite{GeHJY15}. Therefore, it suffices to analyze the local minima of the loss landscape and show that no spurious local minima exist. This direction has led to positive results in matrix sensing \cite{ParkKCS16a}, matrix completion \cite{GeLM16}, dictionary learning \cite{SunQW15}, phase retrieval \cite{SunQW16}, and orthogonal tensor decomposition \cite{GeHJY15}. 

A non-convex analysis of the deep learning has been largely elusive and more discouragingly, there has been hardness results for even simple networks. A neural network with one hidden unit and sigmoidal activation can admit exponentially many local minima \cite{Auer}. Backprogration has been proven to fail in a simple network due to the abundance of bad local minima \cite{brady1989back}. Training a 3-node neural network with one hidden layer is { NP}-complete \cite{BlumR88}.  But, these and many similar worst-case hardness results are based on worst case training data assumptions. However, by using a result in \cite{klivans2006cryptographic} that learning a neural network with threshold activation functions is equivalent to learning intersection of halfspaces, several authors showed that under certain cryptographic assumptions, depth-two neural networks are not efficiently learnable with smooth activation functions \cite{LivniSS14} \cite{ZhangLWJ15}\cite{ZhangLJ15}. 

Due to the difficulty of analysis, many have turned to improper learning and the study of non-gradient methods to train neural networks. Janzamin et. al use tensor decomposition methods to learn the shallow neural network weights, provided access to the score function of the training data distribution \cite{JanzaminSA15}. Eigenvector and tensor methods are also used to train shallow neural networks with quadratic activation functions in \cite{LivniSS14}. Combinatorial methods that exploit layerwise correlations in sparse networks have also been analyzed provably in \cite{AroraBGM13}. Kernel methods, ridge regression, and even boosting were explored for regularized neural networks with smooth activation functions in \cite{shalev2011learning}\cite{ZhangLWJ15}\cite{ZhangLJ15}. Non-smooth activation functions, such as the ReLU, can be approximated by polynomials and are also amenable to kernel methods\cite{GoelKKT16}. These methods require assumptions that lessen the relevance of these algorithms to practice and more importantly, come short of explaining the widespread success of simple SGD.

Therefore, we will only focus on gradient-based methods in the proper
learning framework and hope to find a theoretical justification of the
good convergence properties. If the activation functions are linear or
if certain independence assumptions are made, Kawaguchi shows that the
only local minima are the global minima \cite{Kawaguchi16a}. Under the
spin-glass and other physical models, some have shown that the loss
landscape admit well-behaving local minima that occur usually when the
overall error is small
\cite{ChoromanskaHMAL14},\cite{DauphinPGCGB14}. When only training
error is considered, some have shown that a global minima can be
achieved if the neural network contains sufficiently many hidden nodes
\cite{SoudryC16}. Our research is largely inspired by
\cite{valiant2014learning}, in which the authors show that when the
target functions are polynomials, gradient descent on neural networks
with one hidden layer is shown to converge to low error, given a large
number of hidden nodes. And when complex perturbations are allowed,
there is no robust local minima.

\subsection{Our Contribution}


In this work, we ask the question: Can we use neural networks to learn neural networks with gradient descent methods? That is, if the function to be learnt is a neural network, and we try to learn it with a network of the same shape and randomly initialized edge weights, then will the gradient descent converge to the right function? Our experimental simulations show that for different widths and heights functions represented by neural networks with random edge weights can be learnt by stochastic gradient descent (see section~\ref{experiments}).

Next, we provide a partial theoretical justification for this phenomena with simplifying assumptions. We will focus on learning depth two networks with a linear activation on the output node. If the neural network takes inputs  $x \in \R^d$ (say from some distribution $\mathcal{D}$) then the output $f(x) = \sum_{i=1}^k b_i\sigma(x,w_i)$ is a sum over $k = \poly(d)$ hidden units scaled by output weights $b_i \in \R$ where $\sigma(x,w):\R^d \times \R^d\to \R$ is the activation function that takes in a hidden weight vector $w \in \R^d$ and the input vector $x \in \R^d$.
given by the form $f(x) = \sum_{i=1}^k b_i\sigma(x,w_i)$. Under some assumptions, we will show that gradient descent can learn at least one of the  $w_i's$ of the target network for certain activation functions. The algorithm will try to learn a guess $\widetilde{f}(x_j) = \sum_{i=1}^k a_i \sigma(x_j,\theta_i)$ for $f$ and then running gradient descent over the parameters $a_i, \theta_i$ will move them to $b_i, w_i$. We will prove that at convergence, at least one $\theta_i$ is equal to (or close to) some $w_j$. Note that we may end up with a many to one mapping of the learned hidden weights to the true hidden weights, instead of a bijection.

\subsubsection{Electron-Proton Dynamics}
%
Our main observation is that the gradient descent dynamics of learning such to layer networks is equivalent to the dynamics of a set of proton-electron charges under a certain electrical attraction force function. Assume for now that the coefficients $b_i$ and $a_i$ are $1$. Thus we are only perform gradient descent over $\theta_i$ to minimize the expected square loss of $f-\widetilde{f}$.
%
\Anote{If you're going to have an informal theorem statement, they should be understandable without reading the main text. That's because a reader like me will ignore your intro text and just look at the theorem statements to gauge your paper. If the theorems look interesting, then they'll read the paper. This theorem refers to f, which is undefined. it also doesn't explain at all in what way the electrons and protons correspond to f. You should either make this explicit in the informal theorem's statement, or you should just blend this informal theorem into the rest of the section.}
\begin{theorem}(informal statement of Theorem~\ref{EPDyn})
Applying gradient descent for learning the output of a neural network
with one hidden layer and a linear activation function at the output
node, under standard Gaussian inputs and squared loss, 
% two layer
% neural network 
% The gradient descent process for learning our neural network $f$ 
is equivalent to the motion of k electrons in the presence of k fixed
protons where the force between any pair of charges is given by a
potential function that depends on $\sigma$.
\end{theorem}
%
The charges reside in $R^d$. The protons are fixed at locations $w_1,..,w_k$. The electrons are at positions $\theta_1,..,\theta_k$ and can move: the total force on each charge is the sum of the pairwise forces, determined by the gradient of the potential function $\Phi(\theta, w) = \expt_{X\sim \mathcal{D}}[ \sigma(X,\theta) \sigma(X,w)]$.  

\subsubsection{Convergence Analysis}
%
Note that for the standard electric potential function given by $\Phi = 1/r$ where $r$ is the distance between the charges, it is known that the electrons must converge with the protons, by Earnshaw's Theorem. However, there is no activation function $\sigma$ corresponding to this $\Phi$. 
%
We derive some sufficient conditions that characterize potentials that arise from activation functions $\sigma$. The different $\sigma$ that we study, their corresponding potentials, and their convergence results are summarized in Table ~\ref{table1}.
First we are show how to construct a complicated activation function, which we call almost $\lambda$-harmonic, for which we show that at convergence of the gradient descent, at least some $\theta_i$ coincides with some $w_j$.
%
\Anote{same problem as the other informal theorem. i don't know what a $\theta$ is. it has only been defined in the text. furthermore, it's far too informal. instead of "carefully" it should give this notion a name. again, better to just blend this content in the rest of the text without calling it an informal theorem.}
\begin{theorem}(informal statement of Theorem~\ref{eigConv}) 
There is
  a polynomial activation function such that upon running gradient
  descent for minimizing the squared loss along with $\ell_2$
  regularization for standard gaussian inputs, at convergence, 
% For some
%   carefully chosen activation function, with regularization, at
%   convergence of running the SGD algorithm, 
we learn at least one of
  the hidden weights of the target neural network.
\end{theorem}
%
We also show some weaker results for other more practical activation functions.
Specifically, for the sign activation, we show that if we consider the loss function with respect to a single variable $\theta_i$, then the only local minimas are at the true weights $w_j$, with high probability under random initializations of $b_i$. For the polynomial activation, we derive a similar result under the assumption that our weights $w_j$ are orthonormal and show convergence of a SGD algorithm to learn these weights in $\poly(d,1/\epsilon)$ time.  
%
\begin{table}[tb]
\caption{Activation, Potentials, and Convergence Results Summary}
\label{table1}
\noindent
\vskip 0.1in
\begin{center}
\begin{small}
\begin{sc}
\begin{tabular}{
  |p{\dimexpr.3\linewidth-2\tabcolsep-1.3333\arrayrulewidth}% column 1
  |p{\dimexpr.37\linewidth-2\tabcolsep-1.3333\arrayrulewidth}% column 2
  |p{\dimexpr.33\linewidth-2\tabcolsep-1.3333\arrayrulewidth}|% column 3
  }
   \hline 
        Name of Activation&  Potential  ($\Phi(\theta,w)$)    & Convergence? \\ \hline 
        Sign & $1 - \frac{2}{\pi}\cos^{-1}(\theta^Tw)$       & Yes d = 2, \ref{signCon} \ref{SignConv}\\ 
        Polynomial  & $(\theta^Tw)^m$       & Yes, $\poly(d,\frac{1}{\epsilon}) \ref{PolyConv}$  \\        
        Almost   $\lambda$-harmonic  & Poly($\theta^Tw$), \ref{AlmostHarmonic}  & Yes, $\poly(d,\frac{1}{\epsilon})$ \\
         % Bessel    &  $e^{-\|\theta-w\|_1}$        & Yes for $d=1$  \\   
        \hline
\end{tabular}
\end{sc}
\end{small}
\end{center}
\vskip -0.1in
\end{table} 
%
Our main tool for analysis is to derive second-order information about our dynamics by using the Laplacian of the Hessian (or a submatrix of the Hessian) of our loss function. Together with some generalization error bounds and discrete optimization results, we can finally translate these convergence results into finite time convergence rates. We remark that our algorithms learn and return a neural network with the same architecture and number of hidden nodes as the target network. This is a big contrast to the improper learning setting of many proposed algorithms. 

We acknowledge that there is still a large gap between our developed theory and practice. However, our work can offer theoretical explanations and guidelines for the design of better activation functions or gradient-based training algorithms. For example, better accuracy and training speed were reported when using the newly discovered exponential linear unit (ELU) activation function in \cite{ClevertUH15} \cite{ShahKSS16}. We hope for more theory-backed answers to these and many other questions in deep learning.

In section 2, we introduce our framework and assumptions, and derive
and derive the equivalence between gradient descent and
electron-proton dynamics under a suitable potential. In section 3, we show convergence results for certain potential functions that may not correspond to bounded activation functions. These convergence results are proven to simply illustrate our ideas. In section 4, we create a realizable potential function and address the steps necessary to prove finite convergence guarantees. In section 5, we consider more realistic activation functions, such as the sign and polynomial function. In section 6, experimental
results confirm that depth-2 neural networks can be learned by
gradient descent with common activation functions but seem to
discredit that claim for higher depth networks.

%%% Local Variables:
%%% mode: latex
%%% TeX-master: "icmlpaper2017.tex"
%%% End:
