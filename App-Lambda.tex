

\section{Convergence of Almost $\lambda$-Harmonic Potentials}\label{App:EigenFunc}

\almostHarmonic*

\begin{proof}
We start with the construction. Let $u,v$ be on the unit sphere, and WLOG let $v = (0,...,1)$. Then, we can reparametrize any function of the dot product $h(u^Tv) = h(\cos(\alpha))$, where $\alpha \in [0, \pi]$ is the angle between $u$ and the positive $z$-axis.
Using the Laplacian formula on $S^{d-1}$ gives: \cite{Laplacian}
\begin{align*}
\Delta h & = (\sin \alpha)^{2-d} \frac{\partial}{\partial \alpha}\left[ (\sin \alpha)^{d-2} \frac{\partial f(\cos \alpha)}{\partial \alpha}\right] \\
& = h''(\cos\alpha)(\sin \alpha)^{2} - (d-1)\cos(\alpha)h'(\cos\alpha)
\end{align*}

\Snote{IMP: Rina, can you please check that this Laplacian is equal to
  the trace of the hessian?}

We want $h$ to be approximately $1$-harmonic, so we want to approximately solve:
%
\[h''(x)(1-x^2) - (d-1)xh'(x) =  \lambda h(x) \]
%
Let us construct our approximate function $h$ as follows. Let $c_n$ are the Taylor coefficients of $h(x)$, then we want to solve the equation:
Therefore, we get the following recurrence: $c_n (n(n-1) + (d-1)*n + \lambda) = c_{n+2} (n+2)(n+1)$. Let $c_0 = 0, c_1 = 1$, then let the recurrence hold for $n=0,...,m-2$ for some $m$ odd. For $i > m$, we let $c_i = 0$. 
\begin{align*}
\Delta_\alpha h & = h''(\cos\alpha)(\sin \alpha)^{2} - (d-1)\cos(\alpha)h'(\cos\alpha) \\
&= h(\cos(\alpha)) + k_mc_m(\cos\alpha)^m
\end{align*}
Where $k_m = m^2 + (d-2)m + \lambda$. Notice that $h(x)$ has non-negative
Taylor coefficients and is a bounded polynomial, so it is a
realizable potential function by Theorem \ref{thm:rotReal}, with
corresponding activation $\sigma(x) = \sum_{i} \sqrt{c_i} H_i(x)$, where $H_i(x)$
is the i-th normalized Hermite polynomial. Now, by normalizing
$\Phi_m(\theta, w) = h(\theta^Tw)/h(1)$, have
%
\[|\Delta_\theta \Phi_m(\theta ,w) - \Phi_m(\theta,w) | \leq \frac{k_mc_m}{h(1)} |\theta^Tw|^m \leq k_m|\theta^Tw|^m\]
Furthermore, since $\Phi$ is a polynomial with non-negative coefficients, $|\Phi(\theta^Tw)| \leq \Phi(|\theta^Tw|) \leq \Phi(1) = 1$. Next, for any differential operator $D$, 

\[D\Phi_m(\theta, w) = \sum_{i=1}^m c_i D(\theta^T w)^i\]

Since $\theta, w \in S^{d-1}$ and $\sum_i c_i = 1$, we realize that the first, second, third partials of $\Phi_m$ are bounded by $\poly(m)$ as each $D(\theta^T w)^i$ is bounded  by $\poly(m)$. Thus we can construct a realizable $m$-almost $\lambda$-Harmonic potential for any $m$ odd with our desired properties. Finally, a $m$-almost $\lambda$-harmonic potential is also $m-1$-almost $\lambda$-harmonic, so our claim follows for all $m$.
 \end{proof}

\begin{restatable}{lemma}{bounded}
\label{bounded}
  Let $L$ be as in Eq.~\eqref{errLoss} and let $b_1,...,b_k$ be reals
  bounded by $\poly(d)$ and $|\Phi|\leq 1$ on $\mathcal{M}$. Let
  $G = L + \beta\|a\|^2$. Then, if
  $G(\boldsymbol{a,\theta}) \leq G(0,\boldsymbol{\theta})$, then
  $\|a\| \leq \poly(d)/\beta$.
\end{restatable}

\begin{proof}
Notice that $G(0, \bold{\theta}) = L(0,\bold{\theta}) \leq \poly(d)$ and so, $G(\boldsymbol{a,\theta}) = L(\boldsymbol{a,\theta}) + \beta \|{\bf a}\|^2 \leq
\poly(d)$. Since $L \geq 0$, our claim follows. 
\end{proof}
